\documentclass [4pt,a4paper,oneside,openany]{book} %Classe del documento, formato carta, singola facciata, apertura capitoli pag destra/sinista (ininfluente se impostato oneside). Formato libro
\usepackage{lmodern}
\usepackage[T1]{fontenc}
\usepackage[utf8]{inputenc}
\usepackage{cite}
\usepackage[italian]{babel} %Lingua Documento (da impostare per la sillabazione)
\usepackage{graphicx} %Pacchetto necessario per la gestione delle immagini
\usepackage{listings} %Per inserire codice
\usepackage[usenames]{color} %Per permettere la colorazione dei caratteri
\usepackage{subfigure}
\usepackage{Pacchetti/algpseudocode} %Pacchetto per l'immissione di pseudocodice
\usepackage[boxed]{Pacchetti/algorithm} %Pacchetto per l'immissione di pseudocodice
\usepackage{amsmath} %Simboli matematici
\usepackage{amsfonts} %Simboli matematici
\usepackage{amssymb} %Simboli matematici
%%\usepackage{Pacchetti/mcode}  %Pacchetto per l'immissione di codice matlab
%%\lstloadlanguages{C}
\usepackage {fancyhdr} %Pacchetto per la gestione accurata della pagina
\usepackage {setspace} %Pacchetto necessario per i comandi successivi onehalfspacing, singlespacing....
\usepackage{longtable} %Pacchetto per la gestione delle tabelle grandi
\usepackage[colorlinks=true]{hyperref} %Segnalibri nel pdf finale, sotto relativi parametri
\usepackage{eurosym}
\hypersetup{
	bookmarksnumbered=true,
	linkcolor=black,
	citecolor=black,
	urlcolor=black,
}
\lstset{
	language=C,
	basicstyle=\small\ttfamily,
	keywordstyle=\color{green}\bfseries,
	commentstyle=\color{red},
	stringstyle=\color{blue},
	showstringspaces=false
}
\usepackage{geometry} % Dimensione pagina
\geometry{a4paper} % Formato carta
\addtolength{\textheight}{75pt} %Margini
\oddsidemargin 30pt

\begin{document} %Inizio Documento

%%FRONTESPIZIO%%=========================================
\begin{titlepage}
 \begin{center}
     \includegraphics[width=6cm]{Immagini/logo.png}\\ %Logo dell'Università, cambiare il percorso, se necessario
     \vspace{1em}
     {\Large \textsc{Universit\`a degli studi di Perugia}}\\
     \vspace{1em}
     {\Large \textsc{Facolt\`a di Ingegneria}}\\
     \vspace{5em}
     {\LARGE \textbf{Appunti di Sicurezza Informatica}}\\
 \end{center}
 
\vskip 2.5cm
\begin{center}
{\normalsize Anno Accademico 2014/2015}
\end{center}
\end{titlepage}


%%INTESTAZIONI PAGINA%%====================================
\pagestyle{fancy}
\renewcommand{\chaptermark} [1]{\chaptername\ \thechapter.\ #1}{} 
\renewcommand{\chaptermark}[1]{\markboth{\thechapter.\ #1}{}} 
\renewcommand{\sectionmark}[1]{\markright{\thesection\ #1}}
\fancyhf{}
\fancyhead[LE,RO]{\bfseries\thepage} 
\fancyhead[LO,RE]{\bfseries\leftmark} 
\fancypagestyle{plain}{%
\fancyhead{} % toglie l'intestazione
\renewcommand{\headrulewidth}{0pt} % e la linea
}

\frontmatter
%%INDICE%%==============================================
\tableofcontents
\mainmatter

%%CAPITOLI===============================================
\chapter{Introduzione alla sicurezza informatica}


\section{Premessa}

\subsection{Definizione}
Che cosa significa sicuro? La parola sicurezza e l’aggettivo sicuro vengono sempre associati a beni che si desidera proteggere da possibili danni, danneggiamenti, perdita, e via dicendo. In particolare un bene è al sicuro se è ben protetto, e la sua messa in sicurezza non deve impedirne l'utilizzo. Esistono molteplici definizioni di sicurezza informatica, ad esempio:
\begin{itemize} 
  \item Security is the degree of protection against danger, damage, loss, and criminal activity ~\cite{wikiSecurity}.
  \item Security is a form of protection where a separation is created between the assets and the threat ~\cite{isecomSecurity}.
\end{itemize}
La parola sicurezza deriva dal latino sine cura: senza preoccupazione. La sicurezza di un sistema può essere definita come la conoscenza del fatto che l’evoluzione del sistema non produrrà stati indesiderati. Le cause che possono minare la sicurezza sono molteplici e spesso
non prevedibili, quindi non si può parlare di sicurezza in senso assoluto, ma solo relativo (\textit{"L'unico computer sicuro è un computer spento"}).

\textbf{Trasversalità della tematica:} Le problematiche di sicurezza interessano molteplici campi (attività lavorative, vita domestica, hobby, giochi, sport, etc.). Di fatto, ogni settore della vita moderna ha delle implicazioni relative alla sicurezza. Il livello di sicurezza di un'organizzazione dipende dai livelli di sicurezza di tutti i suoi comparti/settori. Il livello di sicurezza di un sistema è determinato dal livello di sicurezza dal suo comparto meno sicuro (principio dell'anello debole).


\section{Concetti fondamentali}

\subsection{Sistema informativo e informatico}
\begin{itemize} 
  \item Per sistema informativo (information system) di un’organizzazione si intende l’insieme delle informazioni prodotte ed elaborate e delle risorse umane, materiali e immateriali, coinvolte nel processo di elaborazione di tali informazioni
  \item Per sistema informatico (information and communication technology system) s'intende l'insieme delle varie tecnologie coinvolte nel sistema informativo (il sistema informativo è parte del sistema informatico).
\end{itemize}
Questo corso verterà sulla sicurezza dei sistemi informativi. Tuttavia, si approfondiranno maggiormente questioni inerenti la sicurezza
dei sistemi informatici. Per sicurezza di un sistema informativo si intende il grado di protezione contro qualsiasi minaccia ai suoi asset.
Richiede il soddisfacimento dei seguenti requisiti:
\begin{itemize} 
  \item \textbf{confidenzialità, integrità e disponibilità}
  \item \textbf{assicurazione, autenticità e anonimato}
\end{itemize}
Quando si parla di \textbf{attacco} solitamente si intende la violazione di uno o più di questi requisiti.

\subsection{Vulnerabilità, Minacce, Attacchi, Difesa}
\subsubsection{Vulnerabilità}
Una vulnerabilità (o falla o breccia) è una \textbf{debolezza intrinseca} di un sistema, che potrebbe essere sfruttata per provocare perdite o danni. Scaturisce spesso da errate procedure di sicurezza e/o da errori di progettazione/implementazione, e  in alcuni casi è intimamente legata alla natura del sistema. Ad esempio:
\begin{itemize} 
  \item un sistema potrebbe essere vulnerabile alla manipolazione non autorizzata dei dati causa un bug nella procedura di autenticazione dell’utente
  \item un calcolatore è vulnerabile all'acqua
\end{itemize}
Nel primo caso (vulnerabilità scaturite da errate procedure di sicurezza) l'insorgenza delle vulnerabilità può essere mitigata adottando adeguati standard e norme di qualità.

\subsubsection{Minacce (Threats)}
Per minaccia (threat) ad un sistema informatico/informativo si intende quell’insieme di circostanze che potrebbero arrecare danni ai suoi asset: eventi potenziali, accidentali o deliberati, che, nel caso accadessero, produrrebbero perdite e danni. il realizzarsi di una minaccia generalmente avviene sfruttando una o più vulnerabilità del sistema.  Si parla quindi di situazioni ipotetiche che potrebbero avvenire in determinate circostanze. Ad esempio:
\begin{itemize} 
  \item esecuzione di codice malevole che invia dati sensibili ad un'organizzazione criminale
  \item accesso a dati riservati da parte di entità non autorizzate
  \item perdita di dati a causa della rottura di un apparato hardware o al crash di uno specifico software
\end{itemize}

\subsubsection{Attacchi (Attacks)}
Un attacco (attack) è un atto deliberato teso ad arrecare un danno al sistema. Consiste, di fatto, nella realizzazione di una \textbf{minaccia}. Generalmente, un attacco viene perpetrato attraverso lo sfruttamento di una o più vulnerabilità. Spesso si classificano in base all’entità del danno:
\begin{itemize} 
  \item \textbf{attacco attivo (active attack)}: altera le risorse o ne modifica il processo di gestione/elaborazione
  \item \textbf{attacco passivo (passive attack)}: ottiene le informazioni/dati senza alterarli e senza modificare il relativo processo di
gestione/elaborazione
\end{itemize}
Un’altra importante classificazione è in base "al luogo" da cui viene iniziato l’attacco:
\begin{itemize} 
  \item \textbf{attacco dall’interno (inside attack)}: attacco iniziato da un'entità all’interno del perimetro di sicurezza di un sistema informativo di una data organizzazione
  \item \textbf{attacco dall’esterno (outside attack)}: attacco iniziato da un'entità all’esterno del perimetro di sicurezza
\end{itemize}
Ovviamente, è molto più difficile prevenire e rilevare gli attacchi interni di quelli esterni. Ciò anche a causa del fatto che le misure di prevenzione per questo tipo di attacchi limita notevolmente l'usabilità del sistema (si pensi, ad esempio, alla struttura gerachica in ambiente militare, in cui ogni risorsa conosce il minimo indispensabile per svolgere i propri compiti. In questo modo nel caso la risorsa venga compromessa, si limita il danno. Ovviamente tutto ciò rallenta il processo di funzionamento del sistema).

\subsubsection{Tecniche di difesa}
Diverse contromisure (o misure protettive) possono essere attuate per proteggere un sistema informativo da eventi accidentali e da
attacchi deliberati. Tali misure devono essere strutturate all’interno di un piano di sicurezza redatto dopo un’attenta analisi costi/benefici (textit{cost-effective solutions}).
Le tecniche di difesa possono essere di tipo:
\begin{itemize} 
  \item \textbf{preventivo}: effettuano una serie di \textbf{controlli} per evitare \textbf{a priori} che attacchi noti o immaginabili possano essere sferrati con successo (e.g. controlli aeroportuali, controllo di accessi e permessi negli OS).
  \item \textbf{a posteriori}: sono tese a ridurre gli effetti di un attacco che è riuscito a eludere le misure preventive di cui sopra; devono monitorare un sistema ed essere in grado di \textbf{rivelare} comportamenti anomali.
\end{itemize}

Un \textbf{meccanismo di sicurezza} è un qualsiasi metodo, strumento, o procedura teso a rilevare, prevenire o porre rimedio agli effetti di un attacco alla sicurezza del sistema. La strategia di difesa, qualunque essa sia, combina in modo opportuno
uno o più meccanismi di sicurezza. molti meccanismi di sicurezza consistono in controlli hardware/software.

\subsection{Sicurezza informatica: definizione classica e requisiti CIA}
La sicurezza informatica si fonda sulla protezione dei seguenti macro-requisiti di un sistema informativo (informatico):
\begin{itemize} 
  \item \textbf{Confidenzialità (Confidentiality)}
  \item \textbf{Integrità (Integrity)}
  \item \textbf{Disponibilità (Availability)}
\end{itemize}
Spesso si utilizza l’acronimo C.I.A. per denotarli in modo compatto.
\begin{figure}[htbp]
	\centering%
	\subfigure%
	{\includegraphics[height=7cm, width=7cm, keepaspectratio]{Immagini/Capitolo1/venn_security}}
	\caption{Diagramma di Venn dei macro-requisiti di sicurezza \label{fig:venn_security}} 	
\end{figure}
Ovviamente, come si può notare dal diagramma di Venn in \figurename ~\ref{fig:venn_security} spesso vengono richiesti contemporaneamente più di questi requisiti, cosi come un attacco può violare più requisiti.

\subsubsection{Confidenzialità (Confidentiality)}
Def. \textbf{Confidentiality}: the property that information is not made available or disclosed to unauthorized individuals, entities, or processes. Per confidenzialità si intende quindi la garanzia che alle risorse informatiche accedano solo le parti autorizzate ad accedervi. E' talvolta denominata segretezza, riservatezza o privacy. Nota bene: per accesso non si intende solo la lettura, ma anche la visualizzazione, la stampa o la semplice consapevolezza dell’esistenza di una data risorsa nel sistema.\newline

\textbf{Attacchi alla confidenzialità:} Si ha un attacco alla confidenzialità quando una entità (persona, processo o risorsa) tenta di accedere senza autorizzazione a informazioni protette (mentre queste sono memorizzate, oppure durante l’elaborazione, oppure ancora durante una comunicazione). La protezione della confidenzialità avviene utilizzando in modo appropriato i seguenti strumenti (meccanismi) di sicurezza informatica:
\begin{itemize} 
  \item \textbf{Cifratura (Encryption)}
  \item \textbf{Controllo degli accessi (Access Control)}
  \item \textbf{Sicurezza fisica (Physical security)}
\end{itemize}

\subsubsection{Integrità (Integrity)}
 Def. \textbf{Integrity}: safeguarding the accuracy and completeness of information and processing methods.\newline
 Def. \textbf{Integrity}: the property of safeguarding the accuracy and completeness of assets.\newline


Per integrità si intende quindi la garanzia che le risorse possano essere modificate solo dalle parti autorizzate e solo nei modi prestabiliti. Le modifiche comprendono la scrittura, la variazione, il cambiamento dello stato, l’eliminazione e la creazione. Nel caso di file, conviene includere anche i metadati associati (proprietario, ultimo utente ad averlo letto, data ultima modifica,data di creazione), in modo che un accesso non autorizzato al contenuto possa essere rivelato da un controllo di integrità applicato ai metadati.\newline

\textbf{Attacchi all'integrità:} Si ha un attacco all'integrità quando una entità (persona, processo o risorsa) tenta di modificare senza autorizzazione una o più risorse del sistema informativo.

 La protezione della confidenzialità avviene utilizzando in modo appropriato i seguenti strumenti (meccanismi) di sicurezza informatica (tutti basati su un uso corretto della ridondanza):
\begin{itemize} 
  \item \textbf{Backup}
  \item \textbf{Somma di controllo o Checksum}
  \item \textbf{Codici a correzione di errore (Corruzioni non deliberate ma accidentali, possono essere facilmente elusi da attacchi intelligenti)}
  \item \textbf{Codici di autenticazione dei messaggi o Message Authentication Code MAC} 
\end{itemize}


\subsubsection{Disponibilità (Availability)}
 Def. \textbf{Availability}: ensuring that authorized users have access to information and associated assets when required. \newline

Per disponibilità (availability) si intende quindi che le risorse siano accessibili, nei tempi e nei modi prestabiliti, alle parti autorizzate ogni volta che le richiedono: se una persona o un sistema dispone dei diritti di accesso ad una risorsa, l’accesso non deve essergli impedito. Spesso la disponibilità viene citata tramite il suo opposto: la \textbf{negazione di servizio} (\textbf{Denial of Service} o \textbf{DoS}). La disponibilità può assumere significati/sfumature diverse; una risorsa può trovarsi in uno stato intermedio tra i due opposti stati di piena disponibilità e di piena indisponibilità.

\subsubsection{Conflittualità requisiti CIA}
Spesso i requisiti di confidenzialità, integrità e disponibilità possono essere in conflitto tra loro. E' quindi importante trovare il compromesso ottimo per la garanzia di tutti e tre. E' facile garantire la confidenzialità di una risorsa impedendo a chiunque di accedervi (e.g. se chiudo la risorsa in un blocco di cemento, sicuramente resterà confidenziale. Tuttavia la disponibilità della stessa verrebbe compromessa in senso assoluto).

\subsection{Requisiti AAA}
In aggiunta ai concetti CIA, sovente viene richiesto il soddisfacimento di ulteriori requisiti che vanno sotto l’acronimo A.A.A.

\subsubsection{Assicurazione (Assurance)}
Per assicurazione (assurance) si intende come viene fornita e gestita la fiducia reciproca tra le varie parti coinvolte nel sistema informativo. Tale requisito sopperisce la mancanza di regolamentazione, da parte dei requisiti C.I.A., dell'uso delle risorse di un sistema. E' teso quindi ad evitare un uso non consono dei vari asset del sistema. Tale concetto è ovviamente bidirezionale: così come il sistema deve fidarsi degli utenti (del fatto che non lo usino in modo inappropriato), gli utenti devono fidarsi del sistema (e.g.:trattamento dei dati personali). Il concetto di fiducia è tuttavia difficile da quantificare ed è legato al grado di confidenza sul comportamento che ci si attende dal sistema. Il requisito di assicurazione viene regolato agendo sui seguenti strumenti:
\begin{itemize} 
  \item \textbf{Politiche (Policies):} specifiche comportamentali che regolano l’operato degli attori del sistema
  \item \textbf{Permessi (Permissions):} descrivono le operazioni ammesse/concesse e quelle proibite
  \item \textbf{Protezioni (Protections):} implementazione dei meccanismi di sicurezza tesi ad applicare e far rispettare le politiche e i permessi di cui sopra
  \item \textbf{Qualità del software:} lo sviluppo del software che rispetta standard di qualità rigorosi rende più difficile che il sistema si discosti dai comportamenti attesi 
\end{itemize}

\subsubsection{Autenticità (Authenticity)}
L’autenticità è la capacità di provare che dichiarazioni, politiche e autorizzazioni rilasciate da persone o sistemi siano veritiere. Se non è garantita persone/sistemi possono sostenere argomentazioni non vere senza essere contraddetti da prove oggettive. Se è garantita, si dice anche che è soddisfatto il requisito del textbf{non-ripudio (non-ripudiation)}. Per non-ripudio si intende che dichiarazioni autentiche rilasciate da persone e/o sistemi NON possono essere negate. La firma digitale è il principale meccanismo crittografico che permette di ottenerlo. Un contesto in cui tale requisito è importante è, ad esempio, la PEC.

\subsubsection{Anonimato (Anonymity)}
Per anonimato (anonymity) si intende che non è possibile ottenere o risalire all’dentità di un individuo pur avendo accesso a determinati record/transazioni di un sistema informativo. Se un'organizzazione deve rendere pubblici alcuni dati dei suoi clienti/membri senza violarne la privacy, può adottare alcuni dei seguenti strumenti:
\begin{itemize} 
  \item \textbf{Aggregazione (Aggregation):} combinare mediante somme e/o medie i dati di diversi individui
  \item \textbf{Miscelazione (Mixing):} permutare i dati dei singoli utenti in modo da preservare l’esito di predeterminate query
  \item \textbf{Mediazione (Proxies):} utilizzare degli agenti fidati che si interpongono tra l’individuo e i sistemi con i quali interagisce
  \item \textbf{Pseudonimi (Pseudonyms):} utilizzare delle identità fittizie eventualmente autenticate da una terza entità che funge da garante 
\end{itemize}

\subsection{Tipologie di Minacce e Attacchi}
E' possibile classificare diverse minacce e attacchi rispetto a quali requisiti violano. Vediamone alcune:

\subsubsection{Intercettazione (Eavesdropping)}
Si ha un’intercettazione quando un’entità non autorizzata ha ottenuto l’accesso ad una risorsa. Generalmente questo tipo di attacchi avviene nella fase di trasmissione dell'informazione, e rappresenta una violazione alla confidenzialità. Esempi:
\begin{itemize} 
  \item copia illecita di file di programma o dati
  \item intercettazione di una comunicazione telefonica
  \item sniffing di pacchetti di dati
\end{itemize}
Un’intercettazione potrebbe essere molto difficile da rilevare, poiché un intercettatore “silenzioso” potrebbe essere molto abile e non lasciare tracce o concellarle. 

\subsubsection{Alterazione (Alteration)}
Si ha una alterazione o modifica quando un’entità non autorizzata, oltre ad accedere a una risorsa, interferisce con essa modificandola. Rappresenta una violazione alla confidenzialità. Esempi:
\begin{itemize} 
  \item attacchi di tipo \textbf{man-in-the-middle}: un flusso di dati viene intercettato, modificato e ritrasmesso (rappresenta anche una minaccia alla confidenzialità)
  \item virus informatici che modificano file di sistema critici in modo da eseguire azioni maliziose
  \item cambiare i valori di un database o modificare un programma in modo che esegua dei calcoli diversi
\end{itemize}
Alcuni casi di alterazione possono essere facilmente rilevati, mentre altre modifiche, più sottili, possono essere estremamente difficili da individuare.

\subsubsection{Interruzione (Denial-of-service)}
In un’interruzione, una risorsa del sistema viene degradata o eliminata, diventando non disponibile o inutilizzabile. Rappresenta una minaccia alla disponibilità e/o all'integrità. Esempi:
\begin{itemize} 
  \item la distruzione o il sabotaggio di un dispositivo
  \item la cancellazione di un file
  \item la congestione di un Web server causata da un numero enorme di richieste artificiali
\end{itemize}

\subsubsection{Falsificazione (Masquerading)}
La falsificazione consiste nella contraffazione di risorse (dati/hardware) da parti non autorizzate. Costituisce principalmente una violazione di autenticità, e a volte anche alla confidenzialità e/o all’integrità. Esempi:
\begin{itemize} 
  \item {phishing}: creazione di siti Web apparentemente identici agli originali allo scopo di frodarne gli utenti
  \item textbf{spoofing}: spedizione di pacchetti dati con falsi indirizzi di ritorno
\end{itemize}
A volte le falsificazioni sono facilmente rilevabili, ma se attuate con abilità potrebbero essere del tutto indistinguibili da normali operazioni reali legittime.

\subsubsection{Ripudio (Ripudiation)}
Il ripudio consiste nel negare di aver effettuato una data azione. Ll’azione potrebbe essere la ricezione o la spedizione di un messaggio, così come l’esecuzione di una transazione. Spesso, riguarda il tentativo di recedere da un contratto (accordo) precedentemente assunto in cui era comunque previsto l’uso di ricevute (o simili) per dimostrare l’esecuzione di determinate operazioni. Si tratta di un attacco all’assicurazione.

\subsubsection{Inferenza (Inference), Correlation and traceback}
Per inferenza o attacco inferenziale si intendono un insieme di tecniche che utilizzando la statistica e l’algebra permettono di ricavare/stimare \textbf{informazioni sensibili} a partire da dati non sensibili. Rappresenta un attacco alla confidenzialità, e spesso è attuato nel contesto dei databases. Similmente, per correlation and traceback si intendono un insieme di tecniche basate sulla statistica e sull’algebra che permettono di determinare la sorgente di una particolare informazione o di un particolare flusso di dati.\newline \newline
\textbf{N.B.:} C'è una differenza, in senso giuridico, tra dati personali e dati sensibili. Tale distinzione dipende dalla giurisdizione dei paesi. In particolare in Italia il dato personale viene indicato come un'informazione che permette di identificare un individuo (anagrafica), mentre un dato sensibile rappresenta un'informazione su aspetti della vita privata dell'individuo (orientamento sessuale, opinione politica, credo religioso, etc.)

\subsection{Principi della Sicurezza Informatica}

\subsubsection{Mediazione completa (Complete mediation):} Ogni accesso ad una risorsa deve essere controllato verificando che sia conforme alle politiche di sicurezza stabilite; diffidare da miglioramenti nell’efficienza ottenuti salvando autorizzazioni precedentemente acquisite, poiché i permessi possono variare nel tempo.

\subsubsection{Struttura aperta (Open design):}  L’architettura, il progetto e l'implementazione dei meccanismi di sicurezza di un sistema devono essere resi pubblici.
\begin{itemize} 
  \item la sicurezza deve fondarsi sulla segretezza di pochi elementi chiave
  \item maggior feedback favoriscono l’individuazione di bug, falle e vulnerabilità, aumentando la robustezza e la sicurezza del sistema
  \item un meccanismo di protezione ritenuto sicuro da molti è preferibile ad uno noto solo a pochi. E' quindi bene evitare meccanismi di sicurezza basati sulla segretezza (security by obscurity).
\end{itemize}


\subsubsection{Separazione dei privilegi (Separation of privilege):} Più condizioni dovrebbero essere richieste per concedere l’accesso a risorse limitate o ottenere il permesso di effettuare una data azione. In genere questo principio comporta una separazione logico/funzionale delle componenti di un sistema.

\subsubsection{Minimo privilegio (Least privilege):} Ogni parte di un sistema deve avere i privilegi minimi necessari allo svolgimento dei propri compiti. Per attività inusuali che richiedono maggiori privilegi conviene assegnare autorizzazioni temporanee fortemente limitate nel tempo. In questo modo si riduce il rischio di attacchi basati sulla scalata di privilegi.

\subsubsection{Minimo meccanismo comune (Least common mechanism):} I meccanismi di sicurezza che per l’accesso e la gestione di risorse condivise non dovrebbero essere a loro volta condivisi o dovrebbero essere condivisi il meno possibile. E' quindi buona norma adottare tecniche di isolamento quali la virtualizzazione e il sandboxing (e.g. browser web: nessuna applicazione web può agire sul filesystem, eccetto attraverso i cookies). In questo modo vengono mitigati rischi derivanti da comportamenti malevoli di utenti cui spetta comunque l’accesso a una data risorsa condivisa.

\subsubsection{Usabilità (Usability, Psychological acceptability):} I meccanismi di sicurezza non devono rendere più difficile l’accesso alle risorse. Le interfacce utente devono essere ben progettate, intuitive e di facile utilizzo, mentre i parametri di configurazioni inerenti aspetti di sicurezza devono essere di semplice comprensione e facilmente modificabili.

\subsubsection{Fattore lavoro (Work factor):} Il costo necessario ad aggirare un meccanismo di sicurezza deve essere confrontabile alle risorse di cui dispone un potenziale attaccante. Un meccanismo di sicurezza deve avere un livello di sofisticazione, e pertanto un costo, che tenga conto del valore degli asset da proteggere e delle risorse a disposizione di potenziali attaccanti.

\subsubsection{Monitoraggio (Compromise recording):} A volte può convenire effettuare un monitoraggio dettagliato piuttosto che investire in sofisticati meccanismi di sicurezza di tipo preventivo.

\subsubsection{Penetrazione più semplice:} un attaccante utilizzerà qualsiasi mezzo di penetrazione disponibile: non necessariamente i mezzi più ovvi (prevedibili), non necessariamente i mezzi per i quali sono state installate le difese più solide. (e.g.: E' inutile avere un sistema informatico sicuro se il personale non viene formato e fornisce a chiunque credenziali di accesso). L’applicazione di tale principio presenta le seguenti difficoltà:
\begin{itemize} 
  \item saper anticipare l’avversario, cioè riuscire a prevederlo
  \item gestire in modo equilibrato la sicurezza delle diverse parti di un sistema; rafforzare le difese di una parte può indurre gli avversari ad attaccare un'altra parte (più debole) del sistema.
\end{itemize}

\subsubsection{Temporalità:}  le risorse di un sistema informativo devono essere protette solo fino a quando possiedono un valore, e in modo proporzionale al loro valore. Generalmente tale principio si riferisce ai dati: il loro valore può subire brusche variazioni; si consideri ad esempio il valore dei dati sull’andamento dei mercati prima e dopo la loro divulgazione.

\subsubsection{Anello più debole:}  la sicurezza di un sistema articolato NON può essere più forte del suo anello più debole. La gestione della sicurezza deve tener conto del sistema nel suo insieme. Un elevato grado di sicurezza delle singole parti non implica un elevato grado di sicurezza globale, ma è necessario prevedere anche una strategia oculata di coordinamento della varie parti che non introduca vulnerabilità  (l'insieme è più della somma delle parti).

\section{Fondamenti di Crittografia}
La parola Crittografia deriva dal greco, e significa "scrittura segreta". La crittografia è quindi l’arte e la scienza dello scrivere in modo segreto, ovvero l’arte di offuscare le informazioni in modo incomprensibile prevedendo una tecnica segreta per ricostruirle in modo esatto. L’applicazione storica della crittografia è la protezione della confidenzialità dei messaggi. Con \textbf{crittografia classica} si intende, infatti, l'insieme di tecniche che sopperiscono la necessità di nascondere il contenuto di una comunicazione a soggetti non autorizzati. Con l'avvento dei calcolatori e del digitale, si è passati alla crittografia moderna, che presenta molte applicazioni aggiuntive di non immediata comprensione, ma estremamente utili:
\begin{itemize} 
  \item firma digitale
  \item controllo di integrità sicuro
  \item autenticazione
\end{itemize}

\subsection{Terminologia e definizioni preliminari}
\begin{itemize} 
  \item \textbf{Testo in chiaro (plaintext o cleartext):} messaggio nella sua forma originale
  \item \textbf{Testo cifrato (ciphertext):} messaggio cifrato (criptato o crittografato)
  \item \textbf{Cifratura o crittazione o criptazione (encryption):} processo che produce il testo cifrato a partire dal testo in chiaro
  \item \textbf{Decifratura o decrittazione o decriptazione (decryption):} processo inverso della cifratura
  \item \textbf{Crittografo (cryptographer):} esperto di crittografia
  \item \textbf{Crittoanalisi o crittanalisi (cryptanalysis):} l’arte o la scienza di violare testi cifrati; ricostruire il testo in chiaro senza disporre del segreto necessario alla fase di decifratura
  \item \textbf{Crittoanalista:} esperto di crittoanalisi
  \item \textbf{Crittologia (cryptology):} l’arte o la scienza delle scritture nascoste nella sua accezione pi`u generale; include la crittografia e la crittoanalisi
  \item \textbf{Chiave segreta:} la segretezza dell'elaborazione nell'algoritmo è di solito concentrata nella segretezza di una chiave. E' stata introdotta perché è difficile concepire ogni volta nuovi algoritmi di cifratura ed è difficile spiegare rapidamente il funzionamento di un nuovo    algoritmo ad un soggetto con il quale si desidera comunicare in modo sicuro
\end{itemize}
\begin{figure}[htbp]
	\centering%
	\subfigure%
	{\includegraphics[height=7cm, width=12cm, keepaspectratio]{Immagini/Capitolo1/schema_blocchi_crittografia.png}}
	\caption{Schema a blocchi che illustra il processo cifratura-decifratura \label{fig:schema_blocchi_crittografia}} 	
\end{figure}

\subsection{Definizione formale di un sistema crittografico}: Formalmente, un sistema crittografico è costituito da sette componenti (a volte alcune di queste sono assenti o coincidono):
\begin{itemize} 
  \item l'insieme dei possibili input (plaintext)
  \item l'insieme dei possibili output (ciphertex)
  \item l'insieme delle possibili chiavi di cifratura
  \item l'insieme delle possibili chiavi di decifratura
  \item la corrispondenza tra chiavi di cifratura e di decifratura
  \item l'algoritmo di cifratura da usare
  \item l'algoritmo di decifratura da usare
\end{itemize}

\subsection{Complessità computazionale} 
Uno schema crittografico anche assumendo che sia privo di vulnerabilità intrinseche, può essere (quasi) sempre sottoposto ad attacchi a forza bruta. Affinché ciò possa avvenire è necessario disporre di un test affidabile che permetta di dire se una chiave scelta a caso è quella corretta; cioè, essere in grado di riconoscere il testo in chiaro. Se tale precondizione è verificata si può perseguire un approccio esaustivo in cui si tentano tutte le chiavi. Uno schema crittografico è allora tanto più sicuro quanto è maggiore il costo computazionale necessario a violarlo. Contemporaneamente deve garantire l'efficienza di cifratura/decifratura. Formalizzando, possiamo definire un sistema \textbf{computazionalmente sicuro} se:
\begin{itemize} 
  \item il costo per rendere inefficace il cifrario supera il valore dell’informazione cifrata
  \item il tempo richiesto per rendere inefficace il cifrario supera l’arco temporale in cui l’informazione ha una qualche utilità
\end{itemize}
Tuttavia, stimare lo sforzo computazionale richiesto per effettuare con successo la crittoanalisi del testo cifrato è molto difficile. Il tempo richiesto per un approccio a forza bruta fornisce soltanto un limite superiore: è realistico solo se l’algoritmo non presenta delle debolezze intrinseche di tipo matematico; in questo caso si possono effettuare stime ragionevoli su tempi e costi. In un approccio a forza bruta mediamente si devono provare metà di tutte le possibili chiavi. Ovviamente in questo processo prende parte, come variabile fondamentale, la lunghezza della chiave. Alcuni schemi crittografici prevedono una chiave avente lunghezza variabile: aumentandola aumenta la sicurezza, ma diminuisce l'efficienza. Altri schemi stabiliscono a priori la lunghezza della chiave, se si desidera aumentarla è necessario sviluppare algoritmi simili che utilizzano chiavi di lunghezza diversa, oppure è possibile combinare in modo opportuno tali schemi ottenendo uno schema risultante con una chiave globale più lunga (attenzione al modo in cui si combinano!).

\subsection{Algoritmo pubblico o segreto?}
Ottenere la sicurezza dalla segretezza cioè custodendo gelosamente e nascondendo la metodologia di cifratura/decifratura, \textbf{security by obscurity}, è assolutamente da evitare. Questo perché:
\begin{itemize} 
  \item mantenere la segretezza è difficile, poiché le tecniche di reverse engineering permettono di risalire al codice
  \item è una strategia in chiara opposizione al fundamental tenet of cryptography
\end{itemize}
Oggigiorno, nelle applicazioni di uso civile/commerciale si usano schemi crittografici di dominio pubblico. La segretezza, laddove prevista, è conseguenza di altre cause (e.g. protezione del segreto industriale, contesti militari).

\subsection{Attacchi a sistemi crittografici}
Si possono distinguere cinque categorie di attacchi in base al tipo di informazioni in possesso dell’attaccante (di cui gli ultimi due meno frequenti):

\subsubsection{Solo testo cifrato (ciphertext only)}
L’attaccante conosce l’algoritmo di cifratura e dispone soltanto di una certa quantità di testo cifrato. Nell’ipotesi che non vi siano vulnerabilità intrinseche nello schema crittografico, l’attaccante può tentare un attacco forza bruta, a patto che disponga di un test affidabile per il riconoscimento del testo in chiaro e di potenza di calcolo sufficiente. Tale attacco è anche noto come \textbf{testo in chiaro riconoscibile (recognizable plaintext)}. Riguardo la riconoscibilità del testo in chiaro valgono le seguenti osservazioni:
\begin{itemize} 
  \item nel caso di messaggi testuali, è molto improbabile che una chiave di decifratura errata permetta di ottenere un messaggio verosimile
  \item è essenziale avere una sufficiente quantit`a di testo in chiaro
\end{itemize}
\textbf{Esempio:} il messaggio $m$ è una frase di un qualche linguaggio naturale. Ipotizzando che i caratteri di $m$ siano codificati in ASCII a 8 bit e il testo cifrato $c$ abbia la stessa lunghezza di $m$ (quasi sempre è così). Siano:
\begin{itemize} 
  \item $t$: il numero di caratteri di $m$ e di $c$
  \item $n$ = $8t$: il numero di bit di $m$ e di $c$
  \item $2^{\alpha n}$: il numero totali di messaggi distinti, nel linguaggio naturale considerato, di lunghezza $n$ bit
\end{itemize}
Si osservi che nominalmente esistono $2^n$ stringhe binarie di $n$ bit, ma solo una piccolissima frazione di queste è la codifica di un messaggio nel linguaggio naturale, $0 < \alpha < 1$ è un fattore correttivo usato per modellare tale fenomeno, per la lingua inglese $\alpha = 0.16$.
\begin{figure}[htbp]
	\centering%
	\subfigure%
	{\includegraphics[height=12cm, width=12cm, keepaspectratio]{Immagini/Capitolo1/attacco_solo_tc.png}}
\end{figure}
\newline
Il seguente rapporto:
\begin{equation}
\frac{2^{\alpha n}}{2^n} = \frac{1}{2^{(1 - \alpha)n}}
\end{equation}
rappresenta la probabilità che un stringa random di $n$ bit costituisca la codifica di un messaggio nel linguaggio naturale considerato. Se la chiave di decifratura ha una lunghezza di $k$ bit, effettuando un attacco a forza bruta ($2^k$ tentativi), il numero atteso di messaggi verosimili (nel linguaggio naturale dato) associabili al testo cifrato $c$ è pari a:
\begin{equation}
\frac{2^k}{2^{(1 - \alpha)n}}
\end{equation}
Essendo k fissato, al crescere di n tale numero tende rapidamente a zero, quindi il test di riconoscibilità diventa quasi infallibile se si dispone di una sufficiente quantità di testo cifrato. In alcuni casi, si potrebbe tentare un approccio a forza bruta computazionalmente meno costoso se:
\begin{itemize} 
  \item la chiave di decifratura coincide con quella di cifratura, spesso è così
  \item la chiave di cifratura `e derivata da una password di utente segreta con una procedura nota (si può tentare un approccio a forza bruta sullo spazio delle password, sensibilmente più piccolo dello spazio delle chiavi)
\end{itemize}
IMPORTANTE: uno schema crittografico deve SEMPRE essere sicuro contro attacchi di tipo \textbf{ciphertext only}, poiché il testo cifrato è sempre facilmente intercettabile e ottenibile.

\subsubsection{Testo in chiaro conosciuto (known plaintext)}
In questo caso, il crittoanalista conosce:
\begin{itemize} 
  \item l’algoritmo di cifratura
  \item un certo insieme di coppie $\langle plaintext, ciphertext \rangle$, ma non ha la facoltà di decidere lui il tipo specifico di plaintext
  \item una certa quantità di testo cifrato
\end{itemize}
Alcuni schemi crittografici potrebbero essere molto resistenti ad attacchi di tipo \textbf{ciphertext only} ma rivelarsi vulnerabili ad attacchi di tipo \textbf{known plaintext}. In caso di impiego, è fondamentale prevenire che un attaccante ottenga coppie $\langle plaintext, ciphertext \rangle$.

\subsubsection{Testo in chiaro selezionato (chosen plaintext)}
In questo caso, il crittoanalista conosce:
\begin{itemize} 
  \item l’algoritmo di cifratura
  \item una certa quantità di testo cifrato (con chiave segreta K)
\end{itemize}
e può scegliere a piacimento il testo in chiaro ed ottenere il relativo testo cifrato (sempre con la chiave K). In altre parole può scegliere quali coppie $\langle plaintext, ciphertext \rangle$ conoscere. E' possibile che un sistema crittografico sia sicuro contro attacchi di tipo \textbf{ciphertext only} e \textbf{known plaintext}, ma sia vulnerabile ad attacchi di tipo \textbf{chosen plaintext}.

\subsubsection{Testo cifrato selezionato (chosen ciphertext)}
L'attaccante dispone (oltre che dell'algoritmo di cifratura e del testo cifrato da decifrare) di testo cifrato, con significato, scelto da lui, e corrispondente testo in chiaro.
\subsubsection{Testo selezionato (chosen text)}
L'attaccante dispone sia di testo in chiaro scelto da lui (e corrispondente testo cifrato) sia di testo cifrato scelto da lui (e corrispondente testo in chiaro).
\begin{figure}[htbp]
	\centering%
	\subfigure%
	{\includegraphics[height=14cm, width=15cm, keepaspectratio]{Immagini/Capitolo1/tab_attacchi_cifrari.png}}
	\caption{Tabella riassuntiva tipologie di attacchi \label{fig:tab_attacchi_cifrari}} 	
\end{figure}
\begin{figure}[htbp]
	\centering%
	\subfigure%
	{\includegraphics[height=14cm, width=16cm, keepaspectratio]{Immagini/Capitolo1/tab_tempi.png}}
	\caption{Tempo medio per una ricerca esaustiva \label{fig:tab_tempi}} 	
\end{figure}

\subsection{Tipi di funzioni crittografiche}
\begin{itemize} 
  \item funzioni a chiave pubblica (public key functions): richiedono l'uso di due chiavi, una pubblica e una privata (approfodire non ripudio su firma digitale)
  \item funzioni a chiave segreta (secret key functions): richiedono l'uso di una singola chiave
  \item funzioni hash (hash functions): non usano alcuna chiave 
\end{itemize}

\section{Modelli di controllo degli accessi (Access control models)}
Il controllo degli accessi alla varie risorse di un sistema informativo costituisce un fondamentale meccanismo di sicurezza di tipo \textbf{preventivo}. Previene attacchi alla confidenzialità, all'integrità e all'anonimato. L'idea di base è restringere l’accesso solo a coloro che hanno la necessità di accedere e/o modificare specifiche risorse, in accordo al principio del \textbf{minimo privilegio}.

\subsection{Matrice di controllo degli accessi} 
Una matrice di controllo degli accessi (Access Control Matrix ACM) è una tabella che definisce i permessi di accesso dei vari \textbf{soggetti} di un sistema informativo ai suoi \textbf{oggetti}.
\begin{itemize} 
  \item un \textbf{soggetto} è un utente, un gruppo o una generica entità attiva che desidera effettuare una data azione su un data risorsa
  \item un \textbf{oggetto} è un file, un documento, un dispositivo, una generica risorsa o più in generale un'entità passiva sulla quale si desidera compiere una data azione
\end{itemize}

Ogni riga della tabella è associata ad un soggetto, e ogni colonna è associata ad un oggetto. Ogni cella stabilisce quindi il tipo di azione consentita; il soggetto e l’oggetto sono implicitamente definiti dalla cella. Diverse modalità di accesso sono possibili: lettura, scrittura, copia,
esecuzione, cancellazione, annotazione. Una cella vuota stabilisce che non viene assegnato alcun tipo di accesso.
\begin{figure}[htbp]
	\centering%
	\subfigure%
	{\includegraphics[height=13cm, width=13cm, keepaspectratio]{Immagini/Capitolo1/access_control_matrix_ex.png}}
	\caption{Esempio access control matrix \label{fig:acm}} 	
\end{figure}

\subsubsection{Pro e Contro ACM}
L’adozione di una ACM offre, in linea di principio, i seguenti vantaggi:
\begin{itemize} 
  \item immediatezza nel valutare e modificare i diritti di accesso per una data coppia soggetto-oggetto
  \item facilità di visualizzazione e gestione: semplifica la vita all’amministratore
\end{itemize}
Purtroppo, il grande svantaggio delle ACM, che ne limita fortemente l'applicabilità, è la mancanza di scalabilità. E' del tutto ragionevole pensare che un computer server possa avere ordine di $10^3$ soggetti (per lo più utenti) e $10^6$ oggetti (files e directories), richiedendo una ACM di $10^9$ celle. In questi ordini di grandezza i precedenti vantaggi decadono (non scalabilità).

\subsection{Liste di controllo degli accessi (Access Control List ACL)}
Una lista di controllo degli accessi (access control list ACL) segue un approccio di tipo "object-centered" per garantire una buona scalabilità, in particolare:
\begin{itemize} 
  \item ad ogni oggetto \textbf{o} viene associata un lista \textbf{L}, detta la lista di accesso di \textbf{o}, che enumera tutti i soggetti che hanno un qualche diritto di accesso ad \textbf{o}, e
  \item per ciascuno di tali soggetti, \textbf{s}, sono specificati i tipi di azioni che \textbf{s} può compiere su textbf{o}
\end{itemize}
\begin{figure}[htbp]
	\centering%
	\subfigure%
	{\includegraphics[height=10cm, width=10cm, keepaspectratio]{Immagini/Capitolo1/access_control_list_ex.png}}
	\caption{Esempio access control list \label{fig:acl}} 	
\end{figure}
Nella pratica, il modello ACL comporta un significativo risparmio di memoria rispetto al modello ACM: ogni lista del modello ACL è ottenibile scandendo la relativa colonna del modello ACM ignorando però le celle vuote.

\subsubsection{Pro e Contro ACL}
Il principale vantaggio del modello ACL rispetto al modello ACM è la minor occupazione di memoria che si registra nella pratica (in teoria, nel caso peggiore l’occupazione è la medesima). La memoria occupata complessivamente nel modello ACL, $\sum_{O} size(L_{O})$, è proporzionale a $C_{nv} (ACM)$ cioè al numero di celle non vuote della ACM. Nella pratica $C_{nv} (ACM)$ è molto minore di $C (ACM)$ (che denota il numero totale di celle della ACM); cioè nella pratica $C_{nv} (ACM)<<C (ACM)$. Un altro vantaggio è la facilità di gestione delle ACL da parte del sistema operativo:
\begin{itemize} 
  \item sono incorporate nei metadati associati all’oggetto in questione (i.e. filesystem)
  \item per verificare i diritti di accesso ad un oggetto o non è necessario consultare una struttura dati centralizzata associata a tutti gli oggetti
  \item ma basta manipolare una struttura dati molto più snella, distribuita ed associata soltanto all’oggetto \textbf{o}
\end{itemize}

Il principale svantaggio delle ACL è che non consente di enumerare in modo efficiente i diritti di accesso di un dato soggetto. Tale operazione deve essere espletata ogni qual volta un utente viene rimosso da un sistema. I diritti di accesso di un soggetto \textbf{s} possono ottenersi soltanto effettuando una scansione di tutte le liste di accesso (associate a tutti gli oggetti \textbf{o}), e selezionando i diritti di accesso di \textbf{s} nelle liste in cui tale soggetto compare. Nel modello ACM, invece, ciò poteva banalmente ottenersi esaminando la
riga della matrice relativa al soggetto \textbf{s}.

\subsection{Controllod egli accessi basato su liste di capacità (Capabilities Access Control CAC)}
Il modello basato sulle liste di capacità segue un approccio “subject-centered”, complementare (ortogonale) a quello del modello ACL, per offrire una buona scalabilità. Ad ogni soggetto \textbf{s} viene associata una lista, detta \textbf{C-list} di \textbf{s}, contenente soltato gli oggetti per i quali \textbf{s} ha un qualche diritto di accesso. Per ciascun oggetto \textbf{o} della \textbf{C-list} di \textbf{s} viene specificato il tipo di azione che \textbf{s} può esercitare su \textbf{o}. Similmente al modello ACL, anche il modello CAC comporta un significativo risparmio di memoria rispetto al modello ACM nelle situazioni pratiche: ogni lista del modello CAC è ottenibile scandendo la relativa riga del modello ACM ignorando però le celle vuote. 
\begin{figure}[htbp]
	\centering%
	\subfigure%
	{\includegraphics[height=10cm, width=10cm, keepaspectratio]{Immagini/Capitolo1/CAC_ex.png}}
	\caption{Esempio CAC \label{fig:CAC}} 	
\end{figure}
\subsubsection{Pro e Contro CAC}
Rispetto al modello ACM, anche il modello CAC richiede una minor occupazione di memoria nella pratica. la memoria occupata complessivamente nel modello CAC,  $\sum_{s} size(L_{s})$, è proporzionale a $C_{nv} (ACM)$, cioè al numero di celle non vuote della ACM (vedi considerazioni su ACL). Facilita inoltre il compito dell'amministratore di sistema nell'analizzare e gestire i diritti di accesso di un dato soggetto. Inoltre, l'autorizzazione ad accedere ad un dato oggetto o da parte di un soggetto \textbf{s} richiede tempi ragionevolmente brevi qualora la \textbf{C-list} di \textbf{s} abbia una dimensione contenuta.\newline \newline
Il principale svantaggio del modello CAC è che le \textbf{C-list} non sono direttamente associate agli oggetti. Ciò comporta che:
\begin{itemize} 
  \item non è possibile implementarle in modo distribuito incorporandole nei metadati degli oggetti
  \item l'unico modo per determinare chi e come può accedere ad un dato oggetto o è effettuare una ricerca su tutte le C-list (di tutti i soggetti); nel modello ACM tale operazione richiede semplicemente di esaminare la colonna relativa ad o (problema ortogonale a quello delle ACL)
\end{itemize}

\subsection{Controllod egli accessi basato sui ruoli (Role-Based Access Control RBAC)}
Per controllo degli accessi basato sui ruoli si intende che l'assegnazione dei diritti di accesso avviene in modo indiretto ed è funzione del ruolo attribuito ad un dato soggetto \textbf{s}. L’amministratore definisce i ruoli e specifica i diritti di accesso per tali ruoli, anziché quelli per i singoli soggetti (utenti). Ogni ruolo dovrebbe rappresentare una classe di soggetti (utenti) con la medesima mansione. I soggetti vanno assegnati ai vari ruoli coerentemente alle loro mansioni, e un soggetto può ricevere più ruoli dato che può ricoprire più mansioni. Definiti i ruoli, i diritti di accesso vanno assegnati secondo una logica ruolo-oggetto e NON soggetto-oggetto. I diritti di accesso di un dato soggetto sono l’unione dei diritti di accesso dei suoi ruoli. Pertanto:
\begin{itemize} 
  \item textbf{il modello RBAC non `e alternativo ai modelli ACM, ACL e CAC, ma si pone ad un livello di astrazione superiore}: RBAC deve necessariamente sfruttare un modello per il controllo degli accessi, basato su \textbf{ACM},\textbf{ACL} o \textbf{CAC}, ove i soggetti saranno però sostituiti dai ruoli.
  \item RBAC deve inoltre mantenere/gestire l'elenco dei ruoli associati ai vari soggetti
\end{itemize}

\subsubsection{Architettura RBAC}
Il modello RBAC `e realizzabile utilizzando un generico framework per la gestione del controllo degli accessi (ACM, ACL, CAC), ove i soggetti sono, come già detto, sostituiti con i ruoli, più un layer superiore che gestisce l'elenco dei ruoli associati ai vari soggetti e che traduce le richieste di accesso per un dato soggetto in quelle relative ai suoi ruoli.
\begin{figure}[htbp]
	\centering%
	\subfigure%
	{\includegraphics[height=10cm, width=10cm, keepaspectratio]{Immagini/Capitolo1/RBAC_arch.png}}
	\caption{Architettura RBAC \label{fig:RBAC_arch}} 	
\end{figure}

\subsubsection{Gerarchia dei ruoli}
Generalmente è conveniente organizzare i ruoli in una struttura gerarchica, così da riflettere l’organigramma (gerarchico) di una data organizzazione. I diritti di accesso vengono più agevolmente gestiti e assegnati ai vari ruoli sfruttando l’ereditarietà.
\begin{figure}[htbp]
	\centering%
	\subfigure%
	{\includegraphics[height=10cm, width=10cm, keepaspectratio]{Immagini/Capitolo1/RBAC_ger.png}}
	\caption{Architettura RBAC \label{fig:RBAC_ger}} 	
\end{figure}

\subsubsection{Pro e Contro RBAC}
Indipendentemente dal framework di basso di livello utilizzato per il controllo degli accessi, i vantaggi nell’uso del modello RBAC sono:
\begin{itemize} 
  \item la riduzione drastica del numero totale di regole di accesso da gestire; nella pratica il numero dei ruoli è significativamente inferiore a quello dei soggetti, inoltre la loro organizzazione gerarchica ne semplifica ulteriormente la gestione.
  \item l'overhead per determinare se un dato soggetto ha un dato diritto è contenuto, è sufficiente consultare se almeno uno dei suoi ruoli ha tale diritto.
\end{itemize}
Lo svantaggio principale è che non viene implementato dagli attuali sistemi operativi!
 
\chapter{Secret Key Cryptography}

\section{Introduzione}
La crittografia a chiave segreta richiede l’uso di UNA sola chiave: dato un messaggio (il testo in chiaro) e la chiave, la cifratura produce
dati non intellegibili (il testo cifrato). Il testo cifrato ha circa la stessa lunghezza di quello in chiaro e la decifratura è l’inverso della cifratura, ed usa la stessa chiave. La crittografia a chiave segreta è talvolta chiamata crittografia \textbf{convenzionale} o crittografia \textbf{simmetrica}.
\begin{figure}[htbp]
	\centering%
	\subfigure%
	{\includegraphics[height=13cm, width=13cm, keepaspectratio]{Immagini/Capitolo2/segreta_schema_blocchi.png}}
	\caption{Schema a blocchi crittografia a chiave segreta \label{fig:segreta_schema_blocchi}} 	
\end{figure}

\subsection{Impieghi crittografia a chiave segreta}
La crittografia a chiave segreta è alla base di molti meccanismi di sicurezza, i suoi principali impieghi sono: 
\begin{itemize}
  \item protezione della \textbf{confidenzialità} (l'uso classico di queste tecniche è rappresentato dalle \textbf{comunicazioni su un canale insicuro}, uno degli usi più moderni invece dalla \textbf{memorizzazione sicura su supporto insicuro})
  \item protezione dell'\textbf{integrità}: tramite queste tecniche possono essere eseguiti test per rilevare eventuali modifiche non consentite.
  \item autenticazione: tramite l'implementazione di protocolli per verificare l'identità di persone e/o processi.
\end{itemize}

\subsubsection{Comunicazioni su un canale insicuro}
In molte circostanze, due entità devono comunicare attraverso un canale insicuro correndo il rischio di essere ascoltate da una terza parte. Questo contesto è molto comune: si pensi alle reti LAN, che trasmettono dati in broadcast. La crittografia a chiave segreta permette a due entità che condividono un segreto (la chiave) di comunicare attraverso un canale insicuro, ove non può essere garantita l’assenza di intercettazioni/ascoltatori (eavesdropper), avendo la garanzia che il contenuto della comunicazione rimarrà confidenziale.

\subsubsection{Memorizzazione sicura}
Si supponga di disporre di un supporto di memorizzazione non protetto (ad esempio accessibile a molti utenti). Se si desidera salvare i dati proteggendone la confidenzialità si può definire una chiave segreta, salvare i dati dopo averli crittografati con tale chiave e custodire la chiave segreta in un luogo protetto. Il rischio dell'uso di questa tecnica consiste nella possibilità di smarrimento della chiave. In tal caso i dati sarebbero irrevocabilmente persi.

\subsubsection{Autenticazione forte}
Per \textbf{autenticazione forte (strong authentication)} si intende che si è in grado di provare la conoscenza di un segreto, che contraddistingue l'identità di una data entità, senza rivelarlo. L'autenticazione forte è ottenibile utilizzando la crittografia a chiave segreta, ed è particolarmente utile quando due processi devono comunicare su una rete insicura. A rigore il segreto che contraddistingue l'identità che si desidera autenticare dovrebbe essere noto solo a quest'ultima. Nella crittografia chiave segreta tale requisito non può essere soddisfatto. Il segreto in questione è una chiave crittografica che deve essere nota anche all'entità autenticante.

\subsubsection{Esempio}
Si supponga che Alice e Bob condividano una chiave segreta $K_{AB}$ e che vogliano autenticarsi reciprocamente, cioè ciascuno vuole accertarsi dell'identità dell'altro. \newline \textbf{ipotesi:} $K_{AB}$ è nota solo ad Alice e Bob. Alice deve dimostrare a Bob di conoscere $K_{AB}$ senza rivelarla e viceversa. \newline \textbf{Strategia a sfida e risposta:} ciascuno dimostra di conoscere $K_{AB}$ rispondendo ad una sfida posta dall'altro. La \textbf{sfida} è un numero/stringa random \textbf{r} non prevedibile e sempre diversa. La \textbf{risposta} alla sfida è la sfida stessa cifrata \textbf{$E(K_{AB},R)$}. In \figurename ~\ref{fig:strong_auth_sec} è riportato un possibile schema di autenticazione a sfida e risposta (challenge-response) a chiave segreta. La procedura segue i seguenti passi:
\begin{itemize}
  \item Alice genera un numero random $r_{A}$ (la sfida) e la invia al presunto Bob
  \item il presunto Bob critta la sfida con la sua chiave segreta $K'_{AB}$ e restituisce ad Alice la risposta $E(K'_{AB}, r_{A})$
  \item Alice riceve la risposta del presunto Bob e la decritta con la chiave $K_{AB}$, cioè calcola $D(K_{AB},E(K'_{AB}, r_{A}))$. Se ottiene $r_{A}$, allora il presunto Bob è realmente Bob poiché con elevatissima probabilità, se $D(K_{AB},E(K'_{AB}, r_{A})) = r_{A}$, allora $K'_{AB} = K_{AB}$. In caso negativo deduce invece che il presunto Bob è un impostore. In modo analogo Bob verifica l'identità di Alice.
\end{itemize}
\begin{figure}[htbp]
	\centering%
	\subfigure%
	{\includegraphics[height=13cm, width=13cm, keepaspectratio]{Immagini/Capitolo2/strong_auth_secret.png}}
	\caption{Schema a blocchi crittografia a chiave segreta \label{fig:strong_auth_sec}} 	
\end{figure}
La sicurezza del precedente protocollo si fonda sulle seguenti condizioni che non devono venire meno:
\begin{itemize}
  \item solo e soltanto Alice e Bob devono conoscere la chiave segreta $K_{AB}$
  \item le sfide generate devono essere \textbf{randomiche}, e di conseguenza non prevedibili, e \textbf{non ripetibili}, cioè la probabilità che due sfide si ripetano deve tendere a zero. L'attaccante potrebbe infatti collezionare molte coppie testo in chiaro/testo cifrato.
  \item è importante quindi che il numero di bit di una sfida sia superiore ad una data soglia (almeno 64 bit)
\end{itemize}

\subsubsection{Controllo di integrità}
La verifica dell’integrità di un messaggio inviato o di un file è un problema ricorrente in telecomunicazioni e in informatica. Per rilevare eventuali modifiche accidentali si fa uso generalmente di codici (somme) di controllo, detti anche \textbf{checksum}. Un \textbf{checksum} associa ad un qualsiasi messaggio $\textbf{m} \in \{0,1\}^*$ un codice di lunghezza prefissato di $b$ bit (generalmente $b = 32,64,128$ bit): $ck(\textbf{m}) \in \{0,1\}^b$. Un buon \textbf{checksum} dovrebbe variare in modo significativo anche a fronte di iminime variazioni dell'input. \newline \newline 
La sorgente del messaggio $m_{s}$ rende pubblico/invia il corrispondente checksum $ck(m_{s})$. Chi riceve il messaggio $m_{r}$ calcola il checksum e verifica se vale l'uguaglianza $ck(m_{s}) = ck(m_{r})$. In caso affermativo, conclude che $m_{s} = m_{r}$. Si noti tuttavia che, seppur improbabile, è possibile ottenere dei falsi positivi, cioè $ck(m_{s}) = ck(m_{r} )$ anche se non è vero.

\subsubsection{Checksum segreti e non segreti}
I codici di controllo servono per proteggere l’hardware da difetti e da inevitabili errori/guasti. Esistono codici di controllo molto sofisticati come i \textbf{CRC (Cyclic Redundancy Check)} per i quali la probabilità di falsi positivi è estremamente ridotta. Esistono anche i codici \textbf{FEC (Forward Error Correction)} che permettono di correggere eventuali errori oltre che a rilevarli (aggiungendo ulteriore ridondanza). Tuttavia entrambe queste tecniche \textbf{non} sono utilizzabili per la protezione contro attacchi intelligenti. Essendo pubblici, infatti, un avversario intelligente che vuole cambiare un messagio potrebbe modificare anche il codice di controllo in maniera coerente.\newline \newline
Per la protezione contro modifiche maliziose ad un messaggio, è richiesto un codice di controllo (checksum) segreto. Se l'algoritmo non è noto, nessuno può calcolare il checksum corretto per il messaggio modificato. Chiaramente, come nel caso degli algoritmi di cifratura, anziché un
algoritmo segreto conviene avere un algoritmo noto a tutti che richiede la conoscenza di una chiave segreta per il calcolo di un codice di controllo (Vedi pagina ~\pageref{sec:openStruct}, principio \textbf{Open Structure}). \newline \newline
In ciò consiste appunto un checksum cifrato, detto anche \textbf{MIC (Message Integrity Code)}. Il funzionamento del MIC è il seguente:
\begin{itemize}
  \item l'algoritmo produce un codice di autenticazione di lunghezza fissa $MIC(K,m)$, denominato anche \textbf{MAC (Message Authentication Code)}
  \item il codice MIC MIC(K,m) viene trasmesso insieme al messaggio m stesso
  \item formalmente l’input è una coppia $(K,m) \in \{0, 1\}^k \times \{0, 1\}^*$, dove $k$ denota il numero di bit della chiave segreta K, mentre l'output è sempre una stringa binaria di lunghezza prefissata, cioè $MIC(K,m) \in \{0, 1\}^b$
\end{itemize}

\subsection{Cifrari a blocchi e cifrari a flusso}
Un \textbf{cifrario a blocchi} elabora un blocco di elementi in ingresso per volta, producendo un blocco di uscita per ciascun blocco di ingresso. Questa metodologia implica quindi che:
\begin{itemize}
  \item il testo in chiaro deve essere preliminarmente suddiviso in blocchi
  \item il testo cifrato si ottiene combinando i vari blocchi cifrati
\end{itemize}
DES, IDEA e AES sono esempi di cifrari a blocchi simmetrici. Un \textbf{cifrario a flusso}, invece, elabora continuamente gli elementi in ingresso, producendo in uscita un "flusso" di elementi cifrati. Gli elementi cifrati vengono prodotti singolarmente, uno alla volta, man mano che la cifratura procede.

\section{Cifratura a blocchi}

\subsection{Introduzione e concetti generali}
L’algoritmo di cifratura converte un blocco di testo in chiaro in un blocco di testo cifrato. La chiave K non deve essere troppo corta (e.g. se K ha lunghezza 4 bit, sono sufficienti $2^4 = 16$ tentativi per individuarla). Analogamente la lunghezza (fissata) di un blocco non deve essere troppo piccola (e.g. se un blocco ha lunghezza 8 bit, ottenendo delle coppie plaintext - ciphertext si potrebbe costruire una tabella di $2^8 = 256$ coppie utilizzabile per la decfiratura).\newline 
D'altra parte, avere blocchi esageratamente lunghi oltre a non essere necessario dal punto di vista della sicurezza, comporta una gestione più complicata e può degradare le prestazioni. \textbf{64 bit è una lunghezza ragionevole per un blocco}: 
\begin{itemize}
  \item è improbabile ottenere ordine di $2^{64}$ coppie $\langle plaintext, ciphertext \rangle$ per costruire una tabella di decifratura, e
  \item anche se fosse possibile, la sua memorizzazione richiederebbe una spazio enorme ($2^{64}$ record da 64 bit),
  \item come pure l’ordinamento per consentire ricerche efficienti
\end{itemize}
Il modo più generale per cifrare un blocco da 64 bit è definire una \textbf{biiezione} $\gamma :\{0,1\}^{64} \rightarrow \{0,1\}^{64}$. Tuttavia, memorizzare la definizione della \emph{biiezione} in una struttura dati è impraticabile: sarebbero richiesti $2^84 \times 64 = 2^{70}$ bit. Ad essere precisi ce ne vogliono un po' di meno, trattandosi di un \textbf{biiezione}, comunque almeno $2^{69}$ bit sono richiesti \textcolor{red}{(perché?!?!)}. Inoltre, così facendo la chiave è incorporata nella biiezione: per renderla parametrica rispetto alla chiave è necessario memorizzare una biiezione per ogni possibile chiave.\newline \newline
I sistemi di crittografia a chiave segreta sono concepiti per usare una chiave ragionevolmente lunga (ad esempio 64 bit e generare una \textbf{biiezione} che appare, a chi non conosce la chiave, completamente \underline{random}. Se la biiezione fosse realmente random due input $i$ e $i'$ nei quali cambia un solo bit (uno qualunque) sarebbero \textbf{statisticamente indipendenti}: non può succedere, ad esempio, che il terzo bit dell'output cambia \textbf{sempre} quando il dodicesimo bit dell'input cambia. Gli algoritmi crittografici sono pensati per \textbf{diffondere/spargere} in tutti i bit dell'output il valore di ogni bit dell'input: si cerca di far si che ogni bit dell'output dipenda allo stesso modo da tutti i bit dell'input.

\subsection{Sostituzione e Permutazione}
Sostituzioni e permutazioni sono due trasformazioni base applicabili ad un blocco di dati.\newline \newline
Si assuma di dover cifrare un blocco di $k$ bit. Una \textbf{sostituzione} specifica, per ciascuno dei $2^k$ possibili valori dell'input, i $k$ bit dell'output. Per specificare una sostituzione \textbf{"completamente random"} sono necessari circa $k2^k$ bit. E' quindi impraticabile implementare una sostituzione per blocchi di $64$ bit, mentre è fattibile per blocchi di lunghezza di $8$ bit. Esempio di crittografia per sostituzione è il \textbf{Cifrario di Cesare}.\newline \newline
Una \textbf{permutazione} specifica, per ciascuna delle $k$ posizioni dei bit in input, la posizione del corrispondente bit nell'output. Per specificare una permutazione \textbf{"completamente random"} per un blocco di
lunghezza $k$ bit sono necessari $k\log_2 k$ bit. Infatti per ciascuno dei k bit va specificata la sua posizione nell'output; ogni posizione richiede $k\log_2 k$. Ad esempio: essendo $2^6 = 64$, sono necessari $6 = \log_2 {64}$ bit per specificare la nuova posizione che l'i-esimo bit in input avrà in output. \newline \newline
Si noti che una \textbf{permutazione} è un caso particolare di \textbf{sostituzione} in cui ogni bit dell'output ottiene il suo valore da esattamente un bit dell'input.

\subsection{Cifrario a blocchi – schema generale}
Un algoritmo di cifratura a chiave segreta può funzionare come segue:
\begin{itemize}
  \item scompone il blocco in input in pezzi più piccoli (e.g. blocchi da 8 bit)
  \item applica una sostituzione (tramite una \textbf{rete combinatoria}) a ciascun pezzo da 8 bit (la sostituzione dipenderà dal valore della chiave)
  \item gli output delle sostituzioni vengono riuniti in un unico blocco (64 bit)
  \item tale blocco viene permutato in un permutatore a 64 bit (che ha il compito di diffondere le modifiche eseguite nelle sostituzioni)
  \item il processo viene ripetuto un certo numero di volte riportando l'output in ingresso
\end{itemize}
%%La scomposizione in blocchi e la ricomposizione vanno effettuate in maniera saggia, al fine di ottimizzare %%efficacia e efficienza del \textbf{cifrario}.
\subsection{Cifrario a blocchi – esempio}
Ogni attraversamento del cifrario viene detto \textbf{round}. In riferimento alla \figurename ~\ref{fig:block_chipher} si fanno le seguenti considerazioni. Con un solo round, un bit $b_x$ di input può influenzare soltanto 8 bit $b_{x1}, b_{x2}, …, b_{x8}$ dell'output, poiché $b_x$ ha attraversato soltanto un blocco di sostituzione. In generale i bit $b_{x1}, b_{x2}, …, b_{x8}$ non sono consecutivi essendo stati mescolati nel permutatore. Alla fine del secondo round, assumendo che i bit $b_{x1}, b_{x2}, …, b_{x8}$ siano smistati in blocchi di sostituzione distinti, il bit $b_x$ iniziale influenza tutti i bit in output.
\begin{figure}[htbp]
	\centering%
	\subfigure%
	{\includegraphics[height=13cm, width=13cm, keepaspectratio]{Immagini/Capitolo2/block_cipher.png}}
	\caption{Schema a blocchi crittografia a chiave segreta \label{fig:block_chipher}} 	
\end{figure}

\section{Data Encryption Standard - DES}
DES fu pubblicato nel 1977 dal \textbf{National Bureau of Standards}, ora rinominato \textbf{N}ational \textbf{I}nstitute of \textbf{S}tandards and \textbf{T}echnology (\textbf{NIST}), per usi commerciali e "altre" applicazioni del governo statunitense. Progettato da IBM, si basa sul precedente cifrario \textit{Lucifer} ed è frutto della collaborazione con consulenti della NSA. DES usa una chiave di 56 bit, e mappa un blocco di input da 64 bit in un blocco di output da 64 bit (l'algoritmo è quindi abbastanza rigido, al contrario di altri cifrari a blocchi). La \textbf{chiave} è in realtà costituita da una sequenza di 64 bit, ma un bit in ogni ottetto (il blocco da 64 bit è formato da 8 sequenze di 8 bit dette ottetti) è usato come \textbf{odd parity} su ciascun ottetto. Di fatto, soltanto 7 bit in ogni ottetto sono quindi veramente significativi come chiave. \newline \newline

DES è efficiente se realizzato in hardware, ma relativamente lento se implementato in software. Sebbene l'essere difficilmente implementabile come software non fosse un requisito specificato nel progetto molti sostengono che in realtà questa mancanza fosse voluta, forse per limitarne l'uso ad organizzazioni in grado di realizzare sistemi hardware, o forse perché rese più facile controllare l'accesso alla tecnologia. Ad ogni modo, l'aumento delle capacità di calcolo delle CPU rese possibile realizzare una versione software di DES. Una 500-MIPS (Million Instruction Per Second) CPU può infatti cifrare ad un tasso di circa 30 KB/s e forse più, a seconda dei dettagli architetturali della CPU e dell'intelligenza dell'implementazione. Un processore Intel Core i7 Extreme Edition i980EE ha una capacità di calcolo di circa 150 MIPS, quindi può cifrare ad un tasso di circa 9 KB/s (per cifrare 1 MB impiega circa 110 secondi). L’implementazione software è pertanto attualmente adeguata a molte applicazioni.\newline \newline

\subsubsection{Perché chiavi da 56 bit?}
La scelta di una chiave da 56 bit causò molte controversie. Prima che DES fu adottato, le persone al di fuori della \textit{intelligence community} lamentavano che 56 bit non offrivano una sicurezza adeguata. Perché solo 56 dei 64 bit di una chiave DES sono effettivamente usati nell'algoritmo? Lo svantaggio di usare 8 bit della chiave per un controllo di parità è che ciò rende DES molto meno sicuro (256 volte meno sicuro contro una ricerca esaustiva). Ma qual è il vantaggio di usare 8 bit per un controllo di parità? Una possibile risposta è che permette di verificare che la chiave non sia corrotta. Tuttavia questa spiegazione non regge. Se si considerassero infatti 64 bit a caso invece della chiave, c'è una probabilità su 256 che il controllo di parità dia esito positivo. La probabilità che la chiave sia comunque errata (nonostante il controllo di parità dia esito positivo) è troppo alta. Inoltre avere una chiave corrotta non comporta un problema di sicurezza, semplicemente la cifratura/decifratura non viene eseguita correttamente. Chiaramente è anche ridicolo sostenere che la scelta di 56 bit sia stata fatta per risparmiare memoria. \newline La risposta ormai condivisa è che il governo statunitense abbia deliberatamente indebolito la sicurezza di DES di una quantità appena sufficiente da consentire alla NSA di violarlo.

\subsection{Violabilità e sicurezza di DES}
Gli avanzamenti tecnologici dell'industria dei semiconduttori hanno reso ancora più critico il problema della lunghezza della chiave di DES. la velocità dei chip e un po' di furbizia permettono di violare (individuare) le chiavi DES con approcci a forza bruta in tempi ragionevoli. Il rapporto prestazioni/prezzo dell'hardware cresce del
$40\%$ per anno. La lunghezza delle chiavi dovrebbe aumentare di 1 bit ogni 2 anni. Assumendo che 56 bit erano appena sufficienti nel 1979 (quando DES fu standardizzato), 64 bit erano adeguati nel 1995, e 128 bit dovrebbero essere sufficienti fino al 2123. \newline \newline Ragioniamo ora sulla sicurezza di DES. Se si dispone di un singolo blocco $\langle plaintext,ciphertext \rangle$ quanto è difficile trovare la chiave? Un approccio a forza bruta dovrebbe provare ordine di $2^{56} \approx 10^{17}$ chiavi. Se ogni tentativo richiede una singola istruzione sono necessarie ordine di 1000 MIPS-year istruzioni.
\begin{itemize}
  \item 1 MIPS = 1 Milionef di Istruzioni Per Secondo
  \item 1 MIPS-year = numero di istruzioni eseguite in un anno ad un tasso pari a 1 MIPS
  \item 1 MIPS-year = 1 MIPS $\times(365 \times 86400)$ secondi in un anno = $3,1536 \times 10^{13}$ istruzioni
  \item $2^{56} \approx 10^{17} \approx 10^{3}$ MIPS-year
\end{itemize}
Questo vale tuttavia \textbf{SOLO} per effettuare una ricerca esaustiva. Dopo di che entra in gioco il fatto che io sappia riconoscere o meno il testo in chiaro.
\subsubsection{Violabilità DES - esempio}
Anche nell'ipotesi più scomoda per l'avversario di disporre solo di testo cifrato (in ragionevole quantità), un attacco a forza bruta è ancora possibile. Se ad esempio l'avversario sa soltanto che il testo in chiaro è ASCII a 7 bit ogni volta che prova una chiave deve verificare se sono nulli tutti i bit nelle posizioni $8, 16, 24, ..., n \times 8, ...$. Per ogni blocco di 64 bit, vanno esaminati solo 8 bit; i bit in posizione 8, 16, 24, 32, 40, 48, 56, 64. Se almeno uno di questi bit vale 1 la chiave è sicuramente errata. in caso contrario nulla si può dire; \textbf{la probabilità di errore è pertanto 1 su 256 ossia la probabilità che tutti questi bit valgano 0}. Se l'avversario esamina diversi blocchi, ad esempio 10, e verifica che si tratta sempre di ASCII a 7 bit la probabilità che la chiave scelta sia errata si riduce a 1 su 2560. Si noti che gli attuali chip commerciali che implementano DES non si prestano a ricerche esaustive della chiave: sono pensati per cifrare molti dati con una stessa chiave. Infatti il caricamento di una chiave è un'operazione lenta se confrontata con la velocità con la quale viene eseguita la cifratura dei dati. Tuttavia è sempre possibile costruire un chip ottimizzato ad eseguire ricerche esaustive della chiave.

\subsubsection{Violabilità DES - cifratura multipla}
Per ovviare a tali problemi di sicurezza è possibile cifrare più volte e con diverse chiavi lo stesso blocco di dati. Si parla di \textbf{cifratura multipla (multiple encryption)}. Si ritiene che una cifratura con un triplo DES sia
$2^{56}$ volte più difficile da violare.

\subsection{DES - Struttura base}
\begin{figure}[htbp]
	\centering%
	\subfigure%
	{\includegraphics[height=10cm, width=13cm, keepaspectratio]{Immagini/Capitolo2/des_structure.png}}
	\caption{Schema a blocchi DES \label{fig:des_struct}} 	
\end{figure}
La struttura di base dell'algoritmo DES è descritta in \figurename ~\ref{fig:des_struct}. In estrema sintesi:
\begin{itemize}
  \item L'input di 64 bit è sottoposto ad una permutazione iniziale;
  \item La chiave da 56 bit viene usata per generare 16 per-round chiavi da 48 bit; una chiave per ciascuno dei 16 round, prendendo 48 differenti sottoinsiemi dei 56 bit della chiave
  \item Ogni round riceve in input l'output di 64 bit del round precedente e la chiave da 48 bit di quel round
  \item Ogni round restituisce un output di 64 bit
  \item Dopo il 16-esimo round, le due metà dell'output di 64 bit vengono scambiate e il risultato viene sottoposto ad un’altra permutazione (inversa a quella iniziale)
  \item La decifratura consiste semplicemente nell’eseguire la cifratura DES all’indietro
  \item Per decifrare un blocco è necessario applicare la permutazione iniziale (ciò annulla l’effetto della permutazione finale), generare le 16 chiavi di round, che andranno usate in ordine inverso (prima $K_{16}$, l'ultima chiave generata), seguire 16 round esattamente come nella cifratura, scambiare le due metà dell'output e sottoporle a un'altra permutazione (che annulla l'effetto della permutazione iniziale)
\end{itemize}
Per descrivere DES è quindi sufficiente discutere le permutazioni iniziale e finale, come le chiavi di round sono generate, e cosa succede durante un round.


\subsection{DES - Permutazioni dei dati}
L'operazione di permutazione è molto importante in quanto diffonde la dipendenza dei bit in/out. Nel caso di DES però le permutazioni finale e iniziale non hanno effetto. Dimostriamo tale affermazione. \newline
\begin{figure}[htbp]
	\centering%
	\subfigure%
	{\includegraphics[height=10cm, width=8cm, keepaspectratio]{Immagini/Capitolo2/des_perm_3.png}}
	\caption{Inutilità delle permutazioni iniziale e finale \label{fig:des_perm_3}} 	
\end{figure}
Supponiamo di non averle. Se assumiamo per assurdo che il cifrario cosi ottenuto sia facile da violare, vogliamo dimostrare che anche DES è facile da violare. Chiamiamo il cifrario senza permutazione EDS. Se posso violare EDS (disponendo, ad esempio, di una coppia $\langle plaintext, ciphertext \rangle$), basta permutare ciò che ottengo e violo DES. Infatti, sia $\langle m, c \rangle$ una coppia $\langle plaintext, ciphertext \rangle$ di DES. Si consideri allora la coppia $\langle m', c' \rangle$, dove m' e c' sono ottenuti applicando la permutazione iniziale ad m e c. Allora, la chiave $K_{EDS}$ che si ottiene violando EDS per la
coppia $\langle m', c' \rangle$ coincide con la chiave $K_{DES}$ di DES per la
coppia $\langle m, c \rangle$ (vedi \figurename ~\ref{fig:des_perm_3}). L'ipotesi è che tali permutazioni siano state introdotte per rendere più complicata una possibile implementazione software.
\begin{figure}[htbp]
	\centering%
	\subfigure%
	{\includegraphics[height=10cm, width=13cm, keepaspectratio]{Immagini/Capitolo2/des_perm.png}}
	\caption{Tabella permutazioni DES \label{fig:des_perm}} 	
\end{figure}
Come sono state definite quindi le permutazioni iniziale e finale? E' di fatto una permutazione regolare, come è mostrato in \figurename ~\ref{fig:des_perm}. Le tabelle vanno interpretate nel seguente modo: i numeri, da 1 a 64, riportati in tabella rappresentano le posizioni dei bit in input alla permutazione, mentre l’ordine (per righe da sx verso dx) dei numeri nella tabella rappresenta la corrispondente posizione dei bit in output. Ad esempio, la permutazione iniziale sposta il 58-esimo bit in input nel primo bit in output, e il 50-esimo bit in input nel secondo bit in output. Non si tratta di permutazioni generate in modo random, in quanto presenta delle evidenti regolarità, come è evidenziato in \figurename ~\ref{fig:des_perm_2}.
\begin{figure}[htbp]
	\centering%
	\subfigure%
	{\includegraphics[height=10cm, width=10cm, keepaspectratio]{Immagini/Capitolo2/des_perm_2.png}}
	\caption{Regolarità tabella permutazioni DES \label{fig:des_perm_2}} 	
\end{figure}
\subsection{DES - Generazione chiavi di round}
DES genera sedici chiavi di round da 48 bit a partire dalla chiave principale K di 64 bit nominali, di cui solo 56 effettivi. I bit di parità di K non vengono infatti considerati. Denoteremo con $K_{1}$,$K_{2}$,...,$K_{16}$ le sedici chiavi di round. Il procedimento è il seguente:
\begin{itemize}
  \item viene prima effettuata una permutazione iniziale sui 56 bit effettivi di K
  \item i 56 bit in output vengono divisi in due metà $C_{0}$ e $D_{0}$
  \item Le sedici chiavi vengono generate in sedici round, i.e. nell'i-esimo round viene generata la chiave di round $K_{i}$
\end{itemize}
\subsubsection{Permutazione iniziale}
\begin{figure}[htbp]
	\centering%
	\subfigure%
	{\includegraphics[height=10cm, width=10cm, keepaspectratio]{Immagini/Capitolo2/des_perm_keys.png}}
	\caption{Permutazione iniziale della chiave \label{fig:des_perm_keys}} 	
\end{figure}
\begin{figure}[htbp]
	\centering%
	\subfigure%
	{\includegraphics[height=10cm, width=10cm, keepaspectratio]{Immagini/Capitolo2/des_perm_keys_2.png}}
	\caption{Regolarità tabella permutazioni delle chiavi DES \label{fig:des_perm_keys_2}} 	
\end{figure}
La struttura della permutazione iniziale della chiave è illustrata in \figurename ~\ref{fig:des_perm}. Il valore numerico di un elemento della tabella rappresenta la posizione del bit in input, mentre l’ordine nella tabella rappresenta la posizione del bit in output (come per le permutazioni dei dati). Anche in questo caso non è random, in quanto presenta delle evidenti regolarità, come è evidenziato in \figurename ~\ref{fig:des_perm_keys_2}. Come nel caso delle permutazioni iniziale e finale dei dati, non apportano alcun miglioramento alla sicurezza.
\newpage
\subsubsection{Generazione $K_{i}$ nell'i-esimo round}
\begin{figure}[htbp]
	\centering%
	\subfigure%
	{\includegraphics[height=10cm, width=10cm, keepaspectratio]{Immagini/Capitolo2/round_i.png}}
	\caption{generazione di Ki: round i \label{fig:round_i}} 	
\end{figure}
Trattiamo ora la generazione della chiave $K_{i}$ che avviene nell'i-esimo round. La procedura avviene seguendo questi passi (come mostrato in \figurename ~\ref{fig:round_i}):
\begin{itemize}
  \item i bit delle due metà $C_{i-1}$ e $D_{i-1}$ della (i-1)-esima chiave $K_{i-1}$ vengono ruotati (traslazione ciclica) a sinistra. L'entità della traslazione a dipende dal round: nei round 1, 2, 9 e 16 si ha una rotazione a sinistra di un solo bit (i.e. il primo bit diventa l’ultimo bit a destra), mentre negli altri round si ha una rotazione a sinistra di due bit.
  \item la permutazione di $C_{i}$ che produce la metà sinistra di $K_{i}$ è illustrata sotto. Si noti che i bit in posizione 9, 18, 22, e 25 sono scartati.
  	\begin{table}[h]
  	\centering
	\begin{tabular}{llllll}
	14 & 17 & 11 & 24 & 1  & 5  \\
	3  & 28 & 15 & 6  & 21 & 10 \\
	23 & 19 & 12 & 4  & 26 & 8  \\
	16 & 7  & 27 & 20 & 13 & 2 
	\end{tabular}
	\end{table}
  \item la permutazione di $D_{i-1}$ ruotato, cioè di $D_{i}$ , che produce la metà destra di $K_{i}$ è illustrata sotto. Si noti che i bit in posizione 35, 38, 43, e 54 sono scartati; rimangono così 24 bit anziché 28.
  	\begin{table}[h]
  	\centering
	\begin{tabular}{llllll}
	41 & 52 & 31 & 37 & 47 & 55  \\
	30 & 40 & 51 & 45 & 33 & 48 \\
	44 & 49 & 39 & 56 & 34 & 53  \\
	46 & 42 & 50 & 36 & 29 & 32 
	\end{tabular}
	\end{table}
\end{itemize}
\subsection{Un round di DES}
\begin{figure}[htbp]
	\centering%
	\subfigure%
	{\includegraphics[height=10cm, width=9cm, keepaspectratio]{Immagini/Capitolo2/round_des.png}}
	\caption{Struttura di un round di DES \label{fig:round_des}} 	
\end{figure}
La struttura di un round di des è illustrata in \figurename ~\ref{fig:round_des}). I 64 bit in input sono divisi in due metà da 32 bit: $L_{n}$, la metà sinistra dopo l'(n-1)-esimo round, e $R_{n}$, la metà destra dopo l'(n-1)-esimo round. L'output di 64 bit del round si ottiene concatenando le due metà: $L_{n + 1}$, la metà sinistra dopo l'n-esimo round, e $R_{n + 1}$, la metà destra dopo l'n-esimo round. $L_{n+1}$ è semplicemente $R_{n}$, mentre $R_{n + 1}$ è ottenuto come segue:
\begin{itemize}
  \item $R_{n}$ e $K_{n}$ sono posti in input alla mangler function; la mangler function viene anche detta funzione di Feistel
  \item l'output della mangler function, $mangler(R_{n}, K_{n})$, è una quantità di 32 bit; si noti che $R_{n}, K_{n}$ sono composti da
32 e 48 bit rispettivamente
  \item l'output $mangler(R_{n}, K_{n})$ viene poi sommato (XOR) con $L_{n}$
  \item il risultato ottenuto è $R_{n+1} = mangler(R_{n}, K_{n}) \oplus L_{n}$
\end{itemize}
Si noti che ogni round di DES è facilmente invertibile: noti $L_{n+1}$ , $R_{n+1}$ e $k_{n}$ è facile ottenere $L_{n}$ e $R_{n}$. Infatti $R_{n} = L_{n+1}$ e, poiché $R_{n+1} = mangler(R_{n}, K_{n}) \oplus L_{n}$, risulta $R_{n+1} \oplus mangler(R_{n}, K_{n}) = L_{n}$ (in virtù della proprietà dello XOR secondo cui $x \oplus y \oplus y = x$). La mangler function non è quindi mai utilizzata in senso inverso. DES è elegantemento progettato in modo da essere facilmente invertibile senza richiedere l'invertibilità della mangler function. Esaminando attentamente la \figurename ~\ref{fig:round_des}), si evince che la decifratura è di fatto identica alla cifratura, tranne per il fatto che le due metà da 32 bit sono invertite. In altre parole, fornendo $R_{n+1} \mid L_{n+1}$ in input al round n si ottiene $R_{n} \mid L_{n}$ in uscita
\subsection{Mangler Function}
\begin{figure}[htbp]
	\centering%
	\subfigure%
	{\includegraphics[height=11cm, width=11cm, keepaspectratio]{Immagini/Capitolo2/mangler.png}}
	\caption{Mangler function \label{fig:mangler}} 	
\end{figure}
La mangler function prende in input i 32 bit di $R_{n}$, o R per semplificare, e i 48 bit della chiave $K_{n}$ , o K per semplificare. Produce un output da 32 bit che sommato (XOR) con $L_{n}$ permette di ottenere $R_{n+1}$ (il prossimo R).\newline
La prima operazione è l’espansione di R, da 32 bit a 48 bit. R è scomposto in otto pezzi da 4 bit $R = \lbrace r_{1} , r_{2}, ..., r_{8} \rbrace$. L’i-esimo pezzo $r_{i}$ viene espanso a 6 bit aggiungendo in testa e in coda rispettivamente l’ultimo bit di $r_{i-1}$ e il primo bit di $r_{i+1}$. $r_{1}$ e $r_{8}$ sono considerati adiacenti, cioè $r_{0} = r_{8}$ e $r_{9} = r_{1}$. (Fare riferimento alla \figurename ~\ref{fig:mangler_exp})).
\begin{figure}[htbp]
	\centering%
	\subfigure%
	{\includegraphics[height=11cm, width=11cm, keepaspectratio]{Immagini/Capitolo2/mangler_exp.png}}
	\caption{Mangler function - espansione \label{fig:mangler_exp}} 	
\end{figure}
\newline
la chiave K($K_{n}$) viene scomposta in otto chunk (pezzi) da 6 bit. L'i-esimo chunk di R($R_{n}$) espanso viene sommato (XOR) all’i-esimo chunk di K. L'output a 6 bit ottenuto viene sottoposto ad una sostituzione \textbf{S-box} che produce un output di 4 bit. Vi sono in tutto otto S-box distinte; l’i-esima S-box elabora la somma dell’i-esimo chunk di K e di R. Ogni S-box ha 64 possibili input e 16 possibili output. Ovviamente input diversi possono essere mappati nello stesso output (sto riducendo lo spazio). Le S-box sono definite in modo tale che esattamente quattro
input distinti sono mappati in ciascun output possibile. Ogni S-box può riguardarsi come quattro S-box separate aventi 4 bit sia in input che in output: i quattro bit in input corrispondono ai bit interni (dal secondo al quinto) dell'input globale, e i due bit esterni (primo e sesto) dell’input globale servono a selezionare quale dei quattro output ottenuti rappresenta l'output globale.
\begin{figure}[htbp]
	\centering%
	\subfigure%
	{\includegraphics[height=7cm, width=7cm, keepaspectratio]{Immagini/Capitolo2/sbox.png}}
	\caption{S-Box \label{fig:sbox}} 	
\end{figure}
Infine gli otto output, da 4 bit, delle otto S-box sono riuniti in un unico output a 32 bit che viene sottoposto ad una permutazione. Tale permutazione aumenta il livello di sicurezza perché le sostituzioni fatte in ciascuna S-box in un round di DES vengono diffuse negli input di più S-box nel round seguente. Senza tale permutazione, un bit nella parte sinistra dell'input influenzerebbe principalmente alcuni bit della parte sinistra dell'output. Tale permutazione è mostrata in \figurename ~\ref{fig:sbox_perm}).
\begin{figure}[htbp]
	\centering%
	\subfigure%
	{\includegraphics[height=2cm, width=10cm, keepaspectratio]{Immagini/Capitolo2/sbox_perm.png}}
	\caption{Permutazione output S-Box \label{fig:sbox_perm}} 	
\end{figure}
	
\section{Advanced Encryption Standard - AES}
AES fu introdotto perché DES aveva una chiave troppo corta, 3DES (triplo DES) era troppo lento e IDEA era protetto da un brevetto, ed era in parte sospetto e lento. Il NIST si impegnò quindi a sviluppare un nuovo standard. Non si trattava di un problema solo tecnico, ma anche di un problema politico, poiché alcuni rami del governo avevano ostacolato il più possibile la diffusione e l'esportazione della crittografia sicura, e il fatto che il governo appoggiasse il NIST era visto con scetticismo, e molti non si fidavano. Il NIST voleva realmente creare un nuovo standard di sicurezza eccellente, cioè efficiente, flessibile, sicuro e free (non protetto). \newline

Nel 1997 il NIST annunciò una gara per la selezione di un nuovo standard di cifratura destinato a proteggere informazioni governative sensibili. Dopo molti anni di studio e discussioni, il NIST scelse l'algoritmo \textbf{Rijndael}, proposto da due crittografi belgi Joan Daemen e Vincet Rijmen. Rijndael prevede diverse lunghezze per i blocchi e
per la chiave: 128, 160, 192, 224, e 256 bit. La lunghezza di un blocco e quella della chiave possono differire. Il 26 Novembre 2001, viene emanato \textbf{AES}, una standardizzazione di Rijndael. AES quindi consiste in un'implementazione di \textbf{Rijndael} con determinati parametri. Lo standard AES, infatti:
\begin{itemize}
  \item fissa la lunghezza dei blocchi a 128 bit
  \item la chiave può essere di 128, di 192 o di 256 bit. Si parla di \textbf{AES-128}, \textbf{AES-192} e \textbf{AES-256}
\end{itemize}

Nel seguito si descriverà Rijndael, specificando di volta in volta i parametri di AES. A grandi linee, Rijndael somiglia a DES e a IDEA. Ci sono più round che "strapazzano" un blocco di testo in chiaro per ottenere il corrispondente cifrato, e c'è un algoritmo per l'espansione della chiave; a partire dalla chiave segreta genera le chiavi da usare nei vari round.

\subsubsection{Rijndael/AES - parametri}
Rijndael ha una struttura flessibile grazie all'uso di due parametri indipendenti, e di un terzo parametro derivato dai primi due:
\begin{itemize}
  \item $N_{b}$ \textbf{dimensione di un blocco}: numero di parole (word) da 32 bit (colonne da 4 ottetti) in un blocco da cifrare. In AES $N_{b = 4}$, ovvero un blocco ha lunghezza 128 bit, ovvero 4 parole da 32 bit (4 colonne da 4 ottetti).
  \item $N_{k}$ \textbf{dimensione della chiave}: numero di parole (word) da 32 bit (colonne da 4 ottetti) in una chiave di cifratura. In AES-128 $N_{k} = 4$, In AES-192 $N_{k} = 6$, In AES-256 $N_{k} = 8$. In Rijndael $N_{k}$ può essere un qualsiasi intero tra 4 e 8.
  \item $N_{r}$ \textbf{numero di round}: questo parametro dipende da $N_{b}$ e da $N_{k}$. Il numero di round deve aumentare all'aumentare della lunghezza di un blocco (e della chiave); ogni bit del testo in chiaro (e della chiave) deve influenzare (in modo complesso) ciascun bit del testo cifrato. Rijndael specifica che $N_{r}$ = 6 + max($N_{b}$ ,$N_{k}$). In AES-128 $N_{r} = 10$, in AES-192 $N_{r} = 12$, in AES-256 $N_{r} = 14$.
\end{itemize}
  
\subsubsection{Array di stato}
Rijndael mantiene un array di stato rettangolare durante il funzionamento. Ogni elemento dell'array è un ottetto. Complessivamente ci sono $N_{b}$ colonne da 4 ottetti. Lo stato iniziale è ottenuto popolando l’array, colonna per colonna, mediante le $N_{b}$ colonne da 4 ottetti che costituiscono il blocco di input. 

 
\chapter{Modes of Operation}

\section{Introduzione}
Si illustrerà come usare gli algoritmi di crittografia a chiave segreta, DES, IDEA e AES, in applicazioni reali: è stato visto soltanto come usare tali algoritmi per cifrare blocchi di lunghezza prefissata (64 bit per DES e IDEA, 128 bit per AES), come si procede se è necessario cifrare dei messaggi di lunghezza arbitraria/diversa? \newline \newline
Si vedrà inoltre come si generano dei MAC (\textbf{M}essage \textbf{A}uthentication \textbf{C}ode) sfruttando la crittografia a chiave segreta.

\section{Cifrare messaggi di grandi dimensioni}
Come è possibile cifrare messaggi di dimensioni superiori a 64 bit? \newline
Sono state proposte diverse \textbf{modalità operative dei cifrari a blocchi}, cioè ci si è interrogati su come utilizzare i cifrari a blocchi nel caso di messaggi di lunghezza maggiore di quella di un singolo blocco.\newline \newline
Le modalità più conosciute e che verranno descritte di seguito sono:
\begin{itemize}
  \item \textbf{E}lectronic \textbf{C}ode \textbf{B}ook (ECB)
  \item \textbf{C}ipher \textbf{B}lock \textbf{C}haining (CBC)
  \item k-Bit \textbf{C}ipher \textbf{F}eed\textbf{B}ack Mode (CFB)
  \item k-Bit \textbf{O}utput \textbf{F}eed\textbf{B}ack Mode (OFB)
  \item \textbf{C}oun\textbf{T}e\textbf{R} Mode (CTR)
\end{itemize} 
Si noti che nel seguito si farà riferimento a cifrari a blocchi con blocchi di 64 bit (tutte le considerazioni valgono anche nel casi di cifrari con blocchi di dimensione diversa da 64 bit).
\subsection{Electronic Code Book (ECB)}
Questa modalità consiste nel fare la cosa più ovvia, ma corrisponde, in genere, alla soluzione peggiore, cioè:
\begin{itemize}
\item il messaggio viene decomposto in blocchi da 64 bit (inserendo eventualmente dei bit di padding nell'ultimo blocco al fine di riempirlo)
\item ciascun blocco da 64 bit viene cifrato con la chiave segreta
\item ciascun blocco cifrato viene decifrato
\item il messaggio viene ricomposto a partire dai singoli blocchi decifrati
\end{itemize}
Come illustrato nella figure \figurename ~\ref{fig:ECB_enc} e \figurename ~\ref{fig:ECB_dec}
\begin{figure}[htbp]
	\centering%
	\subfigure%
	{\includegraphics[height=4cm, width=12cm, keepaspectratio]{Immagini/Capitolo3/ECB_enc.png}}
	\caption{Schema di crittografia ECB \label{fig:ECB_enc}} 	
	\subfigure%
	{\includegraphics[height=4cm, width=12cm, keepaspectratio]{Immagini/Capitolo3/ECB_dec.png}}
	\caption{Schema di decrittografia ECB \label{fig:ECB_dec}} 
\end{figure}
\subsubsection{Problemi di sicurezza in ECB}
La modalità operativa ECB introduce una serie di problemi non presenti nel cifrario a blocchi: se il messaggio contiene due blocchi di 64 bit identici, allora anche i corrispondenti blocchi cifrati saranno identici e ciò fornisce delle informazioni aggiuntive sul testo in chiaro che un ascoltatore può sfruttare.
\begin{figure}
\centering%
	\subfigure%
	{\includegraphics[height=6cm, width=10cm, keepaspectratio]{Immagini/Capitolo3/File_salari.png}}
	\caption{File con i salari, oggetto di attacco \label{fig:File_salari}} 	
\end{figure}
Si consideri ad esempio lo scenario della \figurename ~\ref{fig:File_salari}: supponiamo che l'ascoltatore sappia che il testo in chiaro contiene l'elenco, ordinato alfabeticamente, degli impiegati e dei relativi salari inviato dall'amministrazione all'ufficio paghe, supponiamo inoltre che ogni riga del file sia lunga esattamente 64 byte (8 blocchi da 8 byte) e che i vari blocchi risultino suddivisi in modo tale che alcuni contengono la codifica della cifra decimale più significativa del campo salario (migliaia di dollari/euro). Comparando i testi cifrati, l'ascoltatore, oltre a dedurre quanti dipendenti hanno lo stesso salario, può anche dedurre quanti dipendenti hanno uno stipendio nello stesso range (ordine di 10 euro/dollari); se ci sono complessivamente pochi range salariali, l'ascoltatore può dedurre a quale categoria di dipendente corrisponda un dato blocco cifrato; inoltre, se l'ascoltatore è un impiegato, può sostituire il blocco cifrato di un altro dipendente (un manager) al suo blocco cifrato (dedotto in base all'ordine e alla
numerosità della sua classe salariale).\newline \newline
ECB quindi ha due serie debolezze, legate al fatto che qualcuno, analizzando diversi blocchi cifrati potrebbe:
\begin{itemize}
\item dedurre (inferire) informazioni sfruttando le ripetizioni di alcuni blocchi
\item riarrangiare/modificare i blocchi cifrati a proprio vantaggio
\end{itemize}
Per tali ragioni ECB è raramente usato.

\subsection{Cipher Block Chaining (CBC)}
CBC non presenta i problemi di ECB: a due blocchi in chiaro identici non corrispondono
due blocchi cifrati identici.
\subsubsection{Idea base}
Per comprendere CBC conviene prima considerare il seguente esempio, in riferimento alla \figurename ~\ref{fig:rand_ele_cb_enc} che ne condivide l’idea base:
\begin{itemize}
\item per ogni blocco di testo in chiaro mi viene generato un numero random a 64 bit $r_{i}$
\item $m_{i}$ e $r_{i}$ vengono sommati ($\oplus$ XOR)
\item il risultato viene cifrato con la chiave segreta
\item i blocchi cifrati $c_{i}$ e i numeri random, in chiaro, $r_{i}$ vengono trasmessi
\end{itemize}
E per riottenere il testo in chiaro:
\begin{itemize}
\item vengono decifrati i blocchi $c_{i}$ con la chiave segreta 
\item i blocchi risultanti vengono sommati ($\oplus$ XOR) con i numeri random $r_{i}$
\end{itemize}
\begin{figure}
\centering%
	\subfigure%
	{\includegraphics[height=5cm, width=10cm, keepaspectratio]{Immagini/Capitolo3/rand_ele_cb_enc.png}}
	\caption{Randomized Electronic Code Book Encryption\label{fig:rand_ele_cb_enc}} 	
\end{figure}
L'esempio appena visto è molto inefficiente, infatti l'informazione da trasmettere è duplicata perché per ogni blocco va trasmesso il corrispondente numero random. \newline
Un altro problema è che un avversario può riarrangiare i blocchi in modo da ottenere un effetto predittivo sul testo in chiaro, ad esempio:
\begin{itemize}
\item se la coppia $r_{2}|c_{2}$ fosse rimossa il corrispondente blocco in chiaro $m_{2}$ scomparirebbe 
\item se la coppia $r_{2}|c_{2}$ fosse scambiata con la coppia $r_{7}|c_{7} \Rightarrow$ $m_{2}$ e $m_{7}$ risulterebbero scambiati
\item se l’avversario conosce ciascun  $m_{i}$, può modificare  $m_{i}$ in modo predittivo cambiando il corrispondente numero random $r_{i}$
\end{itemize}

\subsubsection{Funzionamento}
CBC genera i “propri” numeri random usando $c_{i}$ come numero random $r_{i+1}$, cioè usa il precedente blocco cifrato come numero random da sommare ($\oplus$ XOR) al blocco di testo in chiaro successivo.\newline
Per evitare che due testi in chiaro inizialmente identici diano luogo a dei blocchi cifrati inizialmente identici CBC genera un singolo numero random, detto vettore di inizializzazione (\textbf{Initialization Vector $IV$}), che viene sommato ($\oplus$ XOR) con il primo blocco di testo in chiaro. \newline
Il risultato viene trasmesso dopo la cifratura a chiave segreta. \newline
La decifratura è semplice essendo l'or-esclusivo un'operazione che coincide con la propria inversa.\newline
Quanto detto, rappresentato nelle figure \figurename ~\ref{fig:CBC_enc} e \figurename ~\ref{fig:CBC_dec}, può essere espresso anche algebricamente:
\begin{itemize}
\item CIFRATURA\newline $c_{1} = E(K, (IV \oplus m_{1}))$\newline $c_{i} = E(K, (c_{i-1} \oplus m_{i})) \forall i > 1$
\item DECIFRATURA\newline $m_{1} = E(K,c_{1})\oplus IV$\newline $m_{i} = E(K,c_{i})\oplus c_{i-1}$
\end{itemize}
\begin{figure}[htbp]
	\centering%
	\subfigure%
	{\includegraphics[height=4cm, width=12cm, keepaspectratio]{Immagini/Capitolo3/CBC_enc.png}}
	\caption{Schema di crittografia CBC \label{fig:CBC_enc}} 	
	\subfigure%
	{\includegraphics[height=4cm, width=12cm, keepaspectratio]{Immagini/Capitolo3/CBC_dec.png}}
	\caption{Schema di decrittografia CBC \label{fig:CBC_dec}} 
\end{figure}
Si noti che, essendo il costo della somma ($\oplus$ XOR) trascurabile rispetto al costo della cifratura a chiave segreta, la cifratura con CBC ha le stesse prestazioni della cifratura con ECB eccetto il costo delle generazione e trasmissione di $IV$. In molti casi, tuttavia, la sicurezza di CBC non dipende dalla
scelta del vettore di inizializzazione $IV$ (cioè si possono anche porre tutte le cifre di $IV$ pari a 0).\newline \newline
In alcuni casi, tuttavia, l'assenza di $IV$ riduce la sicurezza. Ad esempio, si supponga che il file cifrato contenente i salari dei dipendenti sia trasmesso settimanalmente; in assenza di $IV$, un ascoltatore potrebbe verificare se il testo cifrato differisce da quelle della precedente settimana, e potrebbe determinare la prima persona il cui salario è cambiato. Un altro esempio è quello di un generale che invia
giornalmente delle informazioni segrete dicendo “continue holding your position”; il testo cifrato sarebbe ogni giorno lo stesso, finché il generale decide di cambiare ordine, inviando il messaggio “start bombing”: il testo cifrato cambierebbe immediatamente, allertando il nemico.\newline
Un vettore di inizializzazione scelto randomicamente garantisce che, anche se lo stesso messaggio è inviato ripetutamente, il corrispondente testo cifrato risulta ogni volta differente, e previene attacchi all'algoritmo di cifratura di tipo testo in chiaro selezionato anche quando un avversario può fornire del testo in chiaro al CBC.
\subsubsection{CBC minaccia 1 – Modifica dei blocchi cifrati}
L’uso di CBC non elimina il problema che qualcuno possa modificare il messaggio in transito, semplicemente cambia la natura della minaccia: un avversario non può più vedere ripetizioni di blocchi cifrati, e non può più copiare/spostare blocchi cifrati (ad esempio per scambiare il salario di due dipendenti) ma può ancora modificare il testo cifrato in modo predittivo. \newline
Cosa potrebbe succedere se modificasse un blocco di testo cifrato, ad esempio $c_{n}$?\newline
Da $m_{n+1} = E(K,c_{n+1})\oplus c_{n}$ si evince che una modifica di $c_{n}$ può avere un effetto prevedibile su $m_{n+1}$ (ad esempio, cambiando il terzo bit di $c_{n}$ cambia il terzo bit di $m_{n+1}$);
chiaramente essendo anche $m_{n} = E(K,c_{n})\oplus c_{n-1}$,  l'avversario non può prevedere quale possa essere il nuovo valore di $m_{n}$, molto probabilmente la modifica di $c_{n}$ corrompe completamente il blocco in chiaro $m_{n}$.\newline
Vediamo a tal proposito l'esempio seguente, illustrato in \figurename ~\ref{fig:modifica_blk_cifrati}:
\begin{figure}[htbp]
	\centering%
	\subfigure%
	{\includegraphics[height=4cm, width=12cm, keepaspectratio]{Immagini/Capitolo3/modifica_blk_cifrati.png}}
	\caption{Modifica dei blocchi cifrati \label{fig:modifica_blk_cifrati}}	
\end{figure}
\newline Supponiamo che un avversario (Trudy) sappia che una data sequenza di blocchi cifrati, del file
dei salari, corrispondano alla riga contenente i suoi dati personali; se Trudy vuole aumentare il suo salario di 10K, e se sa che l'ultimo byte di $m_{7}$ corrisponde alle decine di migliaia nella codifica decimale (00000010), per darsi 10K in più deve semplicemente cambiare il bit meno significativo di $c_{6}$; da $m_{7} = D(K, c_{7}) \oplus c_{6}$ tuttavia, Trudy non sarà più in grado di predire cosa apparirà nella voce “Posizione”, infatti, da $m_{6} = D(K, c_{6}) \oplus c_{5}$, si vede che è impraticabile
prevedere l'effetto della modifica di $c_{6}$ su $m_{6}$. Se il file decifrato fosse letto da una persona umana, questa potrebbe insospettirsi della presenza di simboli strani nel campo “Posizione”, se invece il file decifrato viene elaborato da un programma l'attacco potrebbe non essere rilevato.\newline \newline
Ricapitolando: Trudy è stato in grado di modificare un blocco in modo predittivo con l'effetto collaterale di modificare il blocco precedente senza poter prevedere il risultato finale.

\subsubsection{CBC minaccia 2 – Riarrangiamento blocchi cifrati}
Con riferimento alla \figurename ~\ref{fig:riarrangiamento_blk_cifrati}, si supponga che Trudy conosca il testo in chiaro e, il corrispondente testo cifrato di qualche messaggio, cioè $m_{1}$, $m_{2}$, ..., $m_{n}$ e $IV$, $c_{1}$, $c_{2}$, ..., $c_{n}$; in questo modo Trudy conosce automaticamente anche il blocco decifrato di ciascun $c_{i}$, da $D(K, c_{1}) = c_{i-1} \oplus m_{i}$.
\begin{figure}[htbp]
	\centering%
	\subfigure%
	{\includegraphics[height=3cm, width=12cm, keepaspectratio]{Immagini/Capitolo3/riarrangiamento_blk_cifrati.png}}
	\caption{Riarrangiamento dei blocchi cifrati \label{fig:riarrangiamento_blk_cifrati}}	
\end{figure}
Da queste informazioni, Trudy può considerare ciascun $c_{i}$ come un "building block" e costruire un flusso cifrato usando ogni combinazione di $c_{i}$ ed essere in grado di calcolare quale sarà il corrispondente testo in chiaro. \newline \newline
Per capire a cosa potrebbe servire questo tipo di attacco si tenga presente che uno dei modi di combattere la minaccia di modifica di blocchi cifrati è includere un \textbf{CRC} (\textbf{C}yclic \textbf{R}edundancy \textbf{C}heck) al testo in chiaro prima di cifrarlo con un CBC, perciò se Trudy modifica qualche blocco cifrato, il CRC consentirà ad un computer di rilevare prontamente l'alterazione del messaggio (e.g. se si fosse scelto un CRC a 32 bit ci sarebbe una possibilità su $2^32$ che il CRC coincida con quello corretto); supponiamo che a Trudy non interessi quale possa essere il nuovo messaggio di testo in chiaro (che potrebbe essere completamente indecifrabile), ma interessi solamente che il nuovo messaggio manomesso sia accettato dal computer ricevente sapendo che viene eseguito un controllo di tipo CRC, Trudy può provare a costruire molti flussi cifrati combinando in modi diversi i blocchi $c_{1}$, $c_{2}$, ..., $c_{n}$ e può calcolare il risultante testo in chiaro per ciascuno di essi, per poi testare se il testo in chiaro risultante ha un CRC corretto (mediamente serviranno $2^31$ tentativi).\newline 
Che male potrebbe fare Trudy modificando un messaggio, senza controllarne il contenuto, in modo
tale che sia accettato dal computer ricevente? Forse Trudy è soltanto maliziosa, e vuole distruggere
alcuni dati che vengono caricati attraverso la rete, ma in realtà, c'è un modo sottile di controllare, seppur in misura ridotta, il contenuto del messaggio modificato: supponiamo che Trudy sposti blocchi contigui, ad esempio, se $c_{n}$ e $c_{n+1}$ vengono spostati in qualche altro posto, allora il blocco originale $m_{n+1}$ apparirà in un'altra posizione; se $m_{n+1}$ contiene il salario del presidente, Trudy
potrebbe scambiare i blocchi in modo da cambiarlo con il suo, ma poi dovrà modificare molto probabilmente gran parte del messaggio restante per garantire che il CRC risulti invariato.\newline \newline
Per prevenire attacchi di questo tipo, basati sul riarrangiamento dei blocchi cifrati e tali da preservare il CRC originario, potrebbe essere usato un CRC a 64 bit: ciò è sicuramente sufficiente se l'attacco al CRC, nell'ambito di un CBC, è di tipo a forza bruta. \newline
Per chi progetta protocolli crittografici sicuri, una modalità di cifratura ad un singolo step che protegga sia la confidenzialità che l'autenticità di un messaggio è stata per molti anni una sorta di "Sacro Graal" da ricercare!

\subsection{Output FeedBack Mode (OFB)}
L’OFB è un cifrario a flusso: la cifratura consiste nel sommare ($\oplus$ XOR) il messaggio con il keystream (o one-time pad) generato da OFB stesso. Supponiamo che il keystream sia ottenuto generando singoli blocchi di 64 bit alla volta; un possibile modo per generarlo è il seguente:
\begin{itemize}
\item viene generato un numero random $IV$ (Initialization Vector) di 64 bit
\item il primo blocco del keystream coincide con $IV$: $b_{0} = IV$
\item i blocchi seguenti $b_{i}$ si ottengono cifrando $b_{i-1}$ con la chiave segreta: $b_{i} = E(K,b_{i-1})$
\end{itemize}
Il one-time pad (keystream) risultante è dato dalla sequenza $OTP = OTP(K, IV) = b_{0}|b_{1}|b_{2}|…|b_{i}|b_{i+1}|…$.
La cifratura con OFB (\figurename ~\ref{fig:OFB_enc}) consiste nel sommare ($\oplus$ XOR) il messaggio $m$ con OTP, se $m$ ha lunghezza $l_{m}$ bit si considereranno soltanto $l_{m}$ bit di OTP.
\begin{figure}[htbp]
	\centering%
	\subfigure%
	{\includegraphics[height=4cm, width=12cm, keepaspectratio]{Immagini/Capitolo3/OFB_enc.png}}
	\caption{Schema di crittografia OFB \label{fig:OFB_enc}} 	
\end{figure}
Il risultato della cifratura $c = OTP \oplus m$ viene trasmesso insieme a $IV$ (la lunghezza $l_{c}$ di c coincide con $l_{m}$).\newline
In decifratura (\figurename ~\ref{fig:OFB_dec}) il destinatario riceve $IV$ e conoscendo $K$ calcola lo stesso onetime pad $OTP = OTP(K, IV)$  il messaggio $m$ si ottiene sommando ($\oplus$ XOR) il flusso cifrato $c$ con $l_{c}$ bit di OTP.
\begin{figure}[htbp]
	\centering%
	\subfigure%
	{\includegraphics[height=4cm, width=12cm, keepaspectratio]{Immagini/Capitolo3/OFB_dec.png}}
	\caption{Schema di decrittografia OFB \label{fig:OFB_dec}} 	
\end{figure}
\subsubsection{Vantaggi e svantaggi di OFB}
OFB presenta i seguenti vantaggi:
\begin{itemize}
\item visto che il one-time pad OTP può essere generato in anticipo, prima che sia noto il messaggio m da cifrare, una volta ottenuto m è necessario soltanto effettuare la somma ($\oplus$ XOR) con il one-time pad (lo XOR è eseguibile in modo estremamente veloce)
\item se qualche bit del testo cifrato dovesse corrompersi, soltanto i corrispondenti bit del testo in chiaro sarebbero corrotti, diversamente dalla modalità CBC, dove se $c_{n}$ fosse corrotto allora $m_{n}$ sarebbe completamente corrotto e $m_{n+1}$ sarebbe corrotto in corrispondenza dei medesimi bit di $c_{n}$
\item un messaggio m può arrivare a pezzi di lunghezza arbitraria, e, ogni volta che arriva un pezzo, il corrispondente testo cifrato può essere immediatamente trasmesso; in CBC invece, se il messaggio arriva un byte alla volta, per la cifratura è comunque necessario attendere che un blocco di 64 bit (o un multiplo intero di 8 byte) sia completo (ciò può comportare l'attesa di altri 7 byte o l'aggiunta di bit di riempimento, cosa che aumenta la quantità di dati da trasmettere)
\end{itemize}
OTP ha però anche il seguente svantaggio:
\begin{itemize}
\item se un avversario conoscesse il testo in chiaro m e quello cifrato c, potrebbe modificare il testo in
chiaro a piacimento semplicemente sommando il testo cifrato con il testo in chiaro noto, e sommando il risultato con un qualsiasi messaggio $m'$ che desidera sostituire ad m; cioè l'avversario dovrebbe modificare il testo cifrato come $c' = c \oplus m \oplus m'$ e verificare che decifrando $c'$ anziché $c$ si ottiene $m'$ anziché $m$.
\end{itemize}

\subsection{k-bit Output FeedBack Mode (k-OFB)}
In generale la modalità OFB consente di generare flussi di "pezzi" da $k$ bit (quanto visto prima corrisponde al caso in cui $k$ = 64 bit).
La modalità k-bit OFB funziona nel seguente modo (si descriverà la versione data in [DES81]) con riferimento alla \figurename ~\ref{fig:k-bit_OFB}:
\begin{figure}[htbp]
	\centering%
	\subfigure%
	{\includegraphics[height=4cm, width=12cm, keepaspectratio]{Immagini/Capitolo3/k-bit_OFB.png}}
	\caption{k-bit OFB \label{fig:k-bit_OFB}} 	
\end{figure}
\begin{itemize}
\item l'input $I_{0}$ al modulo di cifratura DES è inizializzato a $IV$, cioè $I_{0}=IV$: se $IV$ ha meno di 64 bit, vengono inseriti degli 0 di riempimento a sinistra (cifre più significative)
\item il primo pezzo $b_{0}$ di OTP si ottiene selezionando $k$ bit dall’output $O_{0} = E(K, I_{0})$ di DES (una quantità a 64 bit); da un punto di vista crittografico non ha importanza come siano scelti tali bit da $O_{0}$; [DES81] specifica che devono essere i $k$ bit più significativi
\item l’i-esimo pezzo $b_{i}$ si ottiene selezionando i $k$ bit più significativi dell'output $O_{i} = E(K, I_{i})$ di DES, ove l'input $I_{i}$ è stato ottenuto da $I_{i-1}$ eseguendo una traslazione a sinistra di $k$ bit, e un inserimento di $b_{i-1}$ nei $k$ bit meno significativi di $I_{1}$ ($k$ bit più a destra)
\end{itemize}
\subsection{Cipher FeedBack Mode (CFB)}
La modalità CFB è molto simile a OFB, con riferimento alla \figurename ~\ref{fig:k-bit_CFB}:
\begin{figure}[htbp]
	\centering%
	\subfigure%
	{\includegraphics[height=4cm, width=12cm, keepaspectratio]{Immagini/Capitolo3/k-bit_CFB.png}}
	\caption{k-bit CFB \label{fig:k-bit_CFB}} 	
\end{figure}
\begin{itemize}
\item viene prodotto un one-time pad generando, uno alla volta, singoli pezzi di $k$ bit
\item il one-time pad viene sommato ($\oplus$ XOR) con pezzi di $k$ bit del messaggio
\end{itemize}
Si noti che in OFB i $k$ bit meno significativi dell'input $I_{i}$ del modulo di cifratura DES sono i $k$ bit di $b_{i-1}$ (sono parte dell'output $O_{i-1}$ della cifratura DES del blocco precedente; invece, in CFB i $k$ bit di $I_{i}$ sono i $k$ bit di testo cifrato del blocco precedente, cioè i $k$ bit di $c_{i-1}$ (in CFB il one-time pad non può essere generato prima che il messaggio e noto, a differenza di OFB), nella modalità a k-bit è ragionevole assegnare a $k$ un valore diverso da 64 bit (una scelta sensata è $k$ = 8 bit).
\subsubsection{Vantaggi e svantaggi  di CFB}
CBF presenta i seguenti vantaggi:
\begin{itemize}
\item Con OFB o CBC, se si ha una perdita di caratteri in trasmissione (testo cifrato), i.e. se nel flusso cifrato $c_{1}, c_{2}, c_{3}, ... , c_{n}, ...$ si perde il carattere $c_{k}$, allora a destinazione si ottiene la sequenza $c_{1}, c_{2}, c_{3}, ... , c_{k'}, ... , c_{n-1'}$ ove $c_{k'} = c_{k+1}, c_{k+1'} = c_{k+2}, ... , c_{k+i'} = c_{k+i+1}$; oppure, con OFB o CBC, se extra caratteri sono aggiunti al flusso cifrato, i.e. se nel flusso cifrato$c_{1}, c_{2}, c_{3}, ... , c_{n}, ...$ si aggiunge il carattere $c*$ dopo di $c_{k-1}$, allora a destinazione si ottiene la sequenza $c_{1}, c_{2}, c_{3}, ... , c_{k'}, ... , c_{n+1'}$ ove $c_{k'} = c*, c_{k+1'} = c_{k}, ... , c_{k+i'} = c_{k+i-1}$. Quindi l'intera parte restante della trasmissione risulta indecifrabile poiché $m_{i’} = c_{i’} \oplus b_{i}$ e $b_{i} = b_{i}(K, IV)$ cioè bi non dipende dalla sequenza cifrata. Invece, con 8-bit CFB, si ha un effetto risincronizzante: se un byte ci è perso in trasmissione allora corrispondente testo in chiaro mi è perso, e i successivi 8 byte $m_{i+1}, …, m_{i+8}$ risulteranno indecifrabili, ma dal byte $m_{i+9}$ in poi il testo in chiaro sarà corretto, questo perché $b_{i} = b_{i}(K, c_{i-1})$, cioè $b_{i}$ è derivato dalla sequenza di caratteri cifrati. Discorsi analoghi valgono nel caso dell'aggiunta di un byte al flusso cifrato.
\item i messaggi cifrati con CFB offrono più protezione di CBC e di OFB rispetto ad eventuali manomissioni; infatti, nel caso di 8-bit CFB un avversario può modificare ogni singolo byte in modo predittivo, ma con l'effetto collaterale di non poter prevedere/controllare i successivi 8 byte discorsi simili valgono per 64-bit CFB.
\item a differenza di CBC, non sono possibili attacchi basati sul riarrangiamento di blocchi; tuttavia intere sezioni del messaggio possono essere riarrangiate rendendo indecifrabili le parti corrispondenti ai "punti di giuntura".
\end{itemize}
CBF ha anche i seguenti svantaggi:
\begin{itemize}
\item 8-bit CFB ha lo svantaggio che ogni byte di input richiede un'operazione DES. Inizialmente CFB fu concepito per essere utilizzato con un numero arbitrario $k$ di bit per "pezzo", con $k$ minore della dimensione di un blocco completo (64 bit per DES); nella pratica tuttavia $k$ è pari a 1 byte oppure coincide con la dimensione piena (full-block) dei blocchi del modulo di cifratura. Quando utilizzato in modalità full-block le prestazioni di CFB sono comparabili a quelle di ECB, CBC, e OFB.
\item come OFB consente di cifrare ed inviare ciascun byte del messaggio non appena è noto tuttavia, a differenza di OFB non è in grado di anticipare il calcolo del one-time pad in fine, è in grado di rilevare delle alterazioni meglio di OFB, ma non bene quanto CBC.
\end{itemize}
\subsection{CounTeR Mode (CTR)}
CTR (\figurename ~\ref{fig:CTR}) è simile a OFB perché un one-time pad viene generato e sommato ($\oplus$ XOR) con i dati; tuttavia differisce da OFB perché non concatena ciascun blocco di one-time-pad con il precedente, ma incrementa $IV$ e poi cifra quanto ottenuto per ottenere il prossimo blocco di one-time pad. 
\begin{figure}[htbp]
	\centering%
	\subfigure%
	{\includegraphics[height=4cm, width=12cm, keepaspectratio]{Immagini/Capitolo3/CTR.png}}
	\caption{CTR \label{fig:CTR}} 	
\end{figure}
\subsection{Vantaggi e svantaggi di CTR}
Il vantaggio principale di CTR è che, come OFB, il one-time pad può essere pre-calcolato, e la cifratura consiste in un semplice XOR; inoltre, come in CBC, la decifratura di un messaggio può iniziare da un qualunque blocco (non e obbligata ad iniziare dal primo blocco). Per questo CRT è l'ideale in applicazioni che richiedono la cifratura di file/memorie ad accesso casuale (sottoinsiemi di dati prelevati ed ordine non prevedibili).\newline \newline
Come in OFB (e in tutti i cifrari a flusso), nella modalità CTR si ha una perdita di sicurezza se messaggi diversi sono cifrati con la stessa coppia $\langle K, IV \rangle$, poichè un avversario potrebbe ottenere la somma ($\oplus$ XOR) dei testi in chiaro se somma ($\oplus$ XOR) due testi cifrati ottenuti con la stessa coppia $\langle K, IV \rangle$.
\section{Generare message authentication code(MAC)}
Un sistema di cifratura a chiave segreta può essere usato per generare un MAC cioè un checksum cifrato: \textbf{MAC} sta per \textbf{M}essage \textbf{A}uthentication \textbf{C}ode. \\
Un sinonimo di MAC è \textbf{MIC} (\textbf{M}essage \textbf{I}ntegrity \textbf{C}ode) e, anche se il termine MAC è più popolare; nella \textbf{PEM} (\textbf{P}rivacy \textbf{E}nhanced \textbf{M}ail) viene usato il termine MIC.\\
\\

Le modalità operative CBC, CFB, OFB, e CTR offrono una buona protezione della confidenzialità, i.e. un messaggio intercettato è difficilmente decifrabile, ma non offrono una buona protezione dell'integrità/autenticità di un messaggio, i.e. non proteggono da ascoltatori che lo modificano in modo non rilevabile.\\
Nel seguito useremo i termini integrità e autenticità in modo intercambiabile visto che se il messaggio é integro allora non è stato modificato dal momento in cui è stato generato, i.e. il messaggio è autentico.
\subsubsection{Residuo CBC}
Un modo standard per assicurare l'autenticità di un messaggio $m$ (cioè per proteggersi da modifiche di $m$ non rilevabili) è, in riferimento alla \figurename ~\ref{fig:residuo_CBC} :
\begin{itemize}
\item calcolare il CBC di $m$
\item inviare soltanto l'ultimo blocco cifrato (64 bit) e il messaggio $m$ in chiaro; l'ultimo blocco cifrato è detto \textbf{residuo CBC}.
\end{itemize}
\begin{figure}[htbp]
	\centering%
	\subfigure%
	{\includegraphics[height=4cm, width=12cm, keepaspectratio]{Immagini/Capitolo3/residuo_CBC.png}}
	\caption{Residuo CBC \label{fig:residuo_CBC}} 	
\end{figure}
Il calcolo del residuo CBC richiede la conoscenza della chiave segreta $K$, quindi se un avversario modifica $m$ in $m’$ allora $res_{CBC}(m’)$ sarà diverso da $res_{CBC}(m)$ (c'è solo 1 possibilità su $2^64$ che siano uguali) perchè l'avversario non è in grado di calcolare $res_{CBC}(m’)$ senza conoscere la chiave segreta $K$. \\
Il destinatario del messaggio calcola il residuo del messaggio in chiaro ricevuto, e verifica che sia uguale al residuo ricevuto; se i residui coincidono deduce che (con elevata probabilità) il residuo ricevuto è stato calcolato da qualcuno che conosce la chiave segreta, i.e. il mittente è autentico.\\
\\
In molte applicazioni non è necessario proteggere la confidenzialità, ma solo l'autenticità; in questi casi si può trasmettere il testo in chiaro più il residuo. Tuttavia, è assai frequente la necessità di
proteggere contemporaneamente confidenzialità e autenticità: se il messaggio $m$ è un singolo blocco, ciò può ottenersi con una semplice cifratura a chiave segreta. Nel caso di un messaggio multi blocco qual è la
trasformazione equivalente?
\textcolor{red}{... 62-72}
\subsubsection{Cifratura CBC con hash crittografico\textcolor{red}{le altre escluse}}

\section{Cifratura multipla DES}
\subsection{Cifratura multipla EDE o 3DES}
In generale, ogni schema di cifratura può essere reso più sicuro ricorrendo alla cifratura multipla.
Nel caso di DES, è universalmente ritenuta sicura la procedura nota come EDE.
\textbf{E}ncrypt-\textbf{D}ecrypt-\textbf{E}ncrypt (o 3DES, \textbf{triplo DES}):
%\begin{itemize}
%\item $ c = E(K_{1}, D(K_{2}, E(K_{1}, m))) = (E_{1} o D_{2} o E_{1})(m) $
%\item $ m = D(K_{1}, E(K_{2}, D(K_{1}, c))) = (D_{1} o E_{2} o D_{1})(c) $
%\end{itemize}

 
\chapter{Funzioni Crittografiche di Hash}
\section{Introduzione}
Una funzione di hash (o semplicemente hash o message digest) è una funzione unidirezionale (o one-way), del tipo h: $\{0, 1\}^{*} \rightarrow \{0, 1\}^{b}$, dove:

\begin{itemize}
	\item $\{0, 1\}^{*}$: spazio delle stringhe binarie di lunghezza qualsiasi
	\item $\{0, 1\}^{b}$: spazio delle stringhe binarie di lunghezza b bit
\end{itemize}

La funzione di hash è considerata unidirezionale (one-way) perché in genere è impraticabile capire quale input corrisponda ad un dato output. Sia \textbf{h()} una funzione di hash, allora \textbf{h()} è una \textbf{funzione di hash sicura} se rispetta le seguenti proprietà:

\begin{itemize}
	\item \textbf{resistenza alla preimmagine}: fissato un hash $\hbar$ è computazionalmente impraticabile trovare un messaggio m tale che $h(m) = \hbar$
	\item \textbf{resistenza alle collisioni}: è computazionalmente	impraticabile trovare due messaggi m1 e m2 aventi lo 	stesso digest h(m1) = h(m2), le proprietà precedenti implicano la seguente
	\item \textbf{resistenza alla seconda preimmagine}: dato un messaggio m, è computazionalmente impraticabile trovare un	messaggio m’ avente lo stesso digest $h(m) = h(m’)$
\end{itemize}


Si useranno i termini hash e message digest in modo intercambiabile; la funzione di hash del NIST è chiamata \textbf{SHA-1}: Secure Hash Algorithm mentre l'acronimo MD degli algoritmi \textbf{MD2}, \textbf{MD4} e \textbf{MD5} sta per Message Digest. Tutti gli algoritmi di digest/hash di base fanno la stessa cosa: prendono in input un messaggio di lunghezza variabile, e restituiscono in output una quantità avente lunghezza prefissata. \newline \newline

\subsubsection{Randomicità della funzione}
Dato un messaggio \textbf{m}, il digest \textbf{h(m)} viene calcolato in modo \textbf{deterministico}. Tuttavia, l'output della funzione di hash dovrebbe apparire il più possibile casuale. Dovrebbe essere impossibile, senza applicare la funzione di hash, predire una qualsiasi porzione dell'output: per ogni sottoinsieme (di posizioni) di bit nel digest \textbf{h(m)} dovrebbe essere possibile ottenere due messaggi m1 e m2 tali che h(m1) e h(m2) presentino gli stessi bit in quelle posizioni \textbf{soltanto procedendo in modo esaustivo}. \newline \newline

Chiaramente, ci sono molti messaggi distinti che sono mappati in uno stesso digest h(m); m ha lunghezza arbitraria, mentre il digest h(m) ha una lunghezza prefissata, ad esempio 128 bit. Se m ha una lunghezza di 1000 bit e h(m) di 128 bit ci sono in media 2872 messaggi che sono mappati in uno stesso digest  dopo molti tentativi, due messaggi aventi lo stesso digest si trovano sicuramente. Tuttavia, per "molti tentativi" si intende un numero talmente grande che è di fatto impossibile. Considerando una buona funzione di digest a 128 bit, è necessario provare approssimativamente $2^{128}$ possibili messaggi prima di ottenere un messaggio avente un particolare digest, o $2^{64}$ messaggi prima di trovarne due aventi lo stesso digest (trovare cioè due messaggi che collidono).

\section{Esempio di Applicazioni}
Un’applicazione delle funzioni di hash crittografiche è il calcolo dell’impronta digitale di
un programma o di un documento di cui si desiderano monitorare eventuali modifiche :
se il message digest h(p) del programma p è noto e se h(p) è memorizzato in modo sicuro (cioè non può essere
modificato da utenti non autorizzati) allora nessun utente non autorizzato può modificare p
senza essere scoperto perché non sarà in grado di trovare un diverso programma p’ tale che h(p’) = h(p).
Quindi sia h:$\{0, 1\}^{*} \rightarrow \{0, 1\}^{b}$ una funzione di hash siano $m_{1}, m_{2}, ..., m_{N}$ N messaggi arbitrariamente
scelti in ${0, 1}^{*}$. \\
\textbf{DOMANDA}: quanto deve valere N per avere una probabilità di 0.5 che due messaggi $m_{i}, m_{j}$ abbiano lo stesso hash?

Una prima stima di N (affinché ci sia 0.5 di possibilità di collisione), per difetto, è la seguente :
\begin{itemize}
\item $k = 2^{b}$: numero totale di possibili hash
\item $\frac{1}{k}$ : probabilità che una coppia di messaggi collida
\item $\textbf{Ipotesi}$: gli eventi considerati sono mutuamente esclusivi 
	\begin{itemize}
		\item $\Pr\{h(mi) = h(mj) \textit{OR} h(mp) = h(mq)\} = \Pr\{h(mi) = h(mj)\} + \Pr\{ h(mp) = h(mq)\}$
	\end{itemize}
\end{itemize}
 
A rigore tale ipotesi non è soddisfatta, molti eventi hanno intersezione non nulla, perciò la stima ottenuta della probabilità è per eccesso e ciò si traduce in una stima per difetto di N $(\Pr\{E1 OR E2\} = \Pr\{E1\} + \Pr\{E2\} – \Pr\{E1 \bigcap E2\})$.
Si ha una probabilità di 0.5 se si considerano $k/2$ coppie, perciò $N(N – 1)/2 = k/2$ e ipotizzando $N >> 1$ si ha $N = k^{1/2} = 2^{b/2}$.

Una seconda stima, più accurata di N è la seguente:
\begin{itemize}
\item P: probabilità che almeno una coppia di messaggi collida
\item $P^{*}$: probabilità che tutte le coppie di messaggi abbiano digest diversi, quindi $P = 1 - P^{*}$
\item \textbf{Ipotesi}: gli eventi considerati sono indipendenti
	\begin{itemize}
		\item $\Pr\{h(mi) \neq h(mj) \textit{AND} h(mp) \neq h(mq)\} = \Pr\{h(mi) \neq h(mj)\} * \Pr\{ h(mp) \neq h(mq)\}$
	\end{itemize}
\end{itemize}

A rigore tale ipotesi non è soddisfatta, gli eventi non sono del tutto indipendenti quindi la stima ottenuta della probabilità P è per difetto e ciò si traduce in una stima per eccesso di N ($\Pr\{E1 AND E2\} = \Pr\{E1\} * \Pr\{E1|E2\}$).
Nella ipotesi di eventi indipendenti si ha $P = 1 – P^{*} = 1 – (1 – 1/k)^{\frac{N(N – 1)}{2}}$, e ipotizzando che k >> 1 si ottiene che $P \approx 1 – e^{– N(N – 1)/2k}$, ponendo pertanto $P\geq1/2 si ottiene che e^{– N(N – 1)/2k} \leqslant 1/2 = e^{–ln2}$; da cui si ottiene che $N(N – 1)/2 \geq (ln2)k$ e, ipotizzando che N >> 1, $N \geqslant (2ln2)^{1/2} \cdot k^{1/2} = (2ln2)^{1/2} \cdot 2^{b/2}$.

Una terza stima ancor più corretta si può ottenere utilizzando la modalità di calcolo utilizzata anche nel paradosso del compleanno :
\begin{itemize}
\item P: probabilità che almeno una coppia di messaggi collida
\item $P^{*}$: probabilità che tutte le coppie di messaggi abbiano digest diversi, quindi $P = 1 - P^{*}$
\item $k = 2^{b}$: numero di possibili hash
\end{itemize}
Si consideri inoltre la seguente notazione:
\begin{itemize}
\item $P_{2^{*}}$: la probabilità che h(m2) sia diverso da h(m1)
\item $P_{3^{*}}$: la probabilità che h(m3) sia diverso da h(m2) e h(m1)
\item ...
\item $P_{i^{*}}$: la probabilità che h(mi) sia diverso da h(mj), $\forall 1\leq j \leq i-1$
\item ...
\item $P_{N^{*}}$: la probabilità che h(mN) sia diverso da h(mj), $\forall 1\leq j \leq N-1$
\end{itemize}
Da tutto ciò segue che $P^{*}=P_{2^{*}}*P_{3^{*}}*...*P_{i^{*}}*...*P_{N^{*}}=
(k – 1)/k*(k – 2)/k*...*(k – N + 1)/k = \frac{k!}{(k^{N}*(k – N)!)}\Rightarrow P^{*} = \frac{k!}{(k^{N}*(k – N)!)}$.
Questa è la stima esatta di $P^{*}$, si nota che il precedente valore di $P^{*}$ si può approssimare con $1 – e^{– N(N – 1)/2k}$.

\section{Lunghezza di un messaggio di Digest}
Quanti bit deve avere l’output di una funzione di hash in modo tale che nessuno sia in grado di trovare due messaggi aventi lo stesso digest?
Se il digest ha b bit, per trovare due messaggi aventi lo stesso digest, è necessario considerare circa $2^{b/2}$ messaggi, se il digest è lungo 64 bit, la ricerca esaustiva in uno spazio di circa $2^{32}$ elementi può essere fattibile mentre se il digest è lungo 128 bit si ritiene che una ricerca in uno spazio di $2^{64}$ elementi sia impraticabile dato l’attuale stato dell’arte.
Per quale ragione è importante che una funzione crittografica di hash sia resistente alle collisioni?
Il fatto che debba essere resistente alla preeimmagine è scontato, ma la resistenza alle collisioni è veramente necessaria?? Si, dato che in alcune circostanze riuscire a trovare due messaggi
con lo stesso digest può comportare dei seri problemi di sicurezza!!
Ad esempio, Alice genera una informazione $x_{A}$ e incarica Bob di calcolare un messaggio $m_{A}$ il cui contenuto deve dipendere da $x_{A}$ secondo criteri prestabiliti.
Una volta che Bob ha calcolato $m_{A}$ lo sottopone ad Alice che ne verifica l'integrità e calcola l'impronta $h(m_{A})$, la firma con la sua chiave privata e invia a Trudy il messaggio in chiaro $m_{A}$ e la sua firma.
Trudy legge il messaggio $m_{A}$ e la firma, controlla che la firma sia quella di Alice e verifica che tutto $m_{A}$ coincida con $h(m_{A})$.
\begin{figure}
	\begin{center}
	{\includegraphics[height=13cm, width=13cm, keepaspectratio]{Immagini/Capitolo4/schema_hash_collisioni.JPG}}
	\end{center}
\end{figure}
Se la funzione non è resistente alla collisioni, Bob potrebbe trovare $m_{A1}$ tale che risulti $h(m_{A})= h(m_{A1})$.
Una volta che Alice ha firmato il messaggio, sostituisce $m_{A}$ con $m_{A1}$ e Trudy non si potrà mai accorgere di nulla.
\begin{figure}
	\begin{center}
	{\includegraphics[height=13cm, width=13cm, keepaspectratio]{Immagini/Capitolo4/schema_hash_collisioni_1.JPG}}
	\end{center}
\end{figure}
Ma, per calcolare due messaggi con lo stesso hash, Bob può seguire il seguente approccio a forza bruta:
\begin{itemize}
\item dato xA, calcola prima il messaggio mA come desiderato da Alice;
\item genera un messaggio falsato $mA^{'}$;
\item esegue il test $h(mA)==h(mA^{'})$;
\item in caso negativo ritorna al punto 2;
\end{itemize}
con questo approccio a forza bruta, però, sono necessari circa $2^{b}$ tentativi, e non $2^{b/2}$!

\section{Impieghi degli Algoritmi di Hash}
Disponendo di un segreto condiviso, l'algoritmo di hash può "sostituire" un algoritmo di crittografia a chiave segreta in ogni suo impiego. (Autenticazione a chiave segreta, Calcolo di MAC, Cifratura e Decifratura).
\subsection{Message Digest per MAC}
Un possibile schema di autenticazione a chiave segreta (composta da "sfide") è il seguente
\begin{figure}
	\begin{center}
	{\includegraphics[height=13cm, width=13cm, keepaspectratio]{Immagini/Capitolo4/schema_autenticazione.JPG}}
	\end{center}
\end{figure}
Questo schema, purtroppo, presenta delle vulnerabilità...così come il seguente:
\begin{figure}
	\begin{center}
	{\includegraphics[height=13cm, width=13cm, keepaspectratio]{Immagini/Capitolo4/schema_autenticazione_md.JPG}}
	\end{center}
\end{figure}
Le vulnerabilità degli algoritmi di digest MD, sono note e facilmente attuabili in un possibile attacco.
Infatti, MD(m) è calcolabile da tutti coloro che conoscono m e l’algoritmo di digest MD senza conoscere la chiave 
segreta!!
Poiché, dall’input m si ottiene un messaggio mp avente lunghezza pari ad un multiplo intero di 512 bit ()mediante un opportuno padding che include, tra l’altro, la lunghezza originaria di m), mp viene decomposto in chunk (pezzi) da 512 bit il digest viene ottenuto mediante una procedura iterativa, quindi il digest all’n-esima iterazione dipende esclusivamente dall’n-esimo chunk e dal digest ottenuto all’(n – 1)-esima iterazione...il digest risultante di m è il digest ottenuto all’ultima iterazione.
Se un attaccante intercetta $\langle m, MD(K_{AB}|m) \rangle$ tra Alice e Bob, l'attaccante senza conoscere la chiave segreta $K_{AB}$ può calcolare il MAC di $\langle m^{+}, MD(K_{AB}|m^{*}) \rangle$ (dato che l'algoritmo di hash è noto) tramite l'ultimo messaggio intercettato. \\
Una possibile soluzione è che il segreto $K_{AB}$ sia messo in coda, a patto che l'algoritmo di hash sia molto resistente alle collisioni!! Un altra possibilità è quella di utilizzare come MAC un sottoinsieme arbitrario del digest MD, in questo caso, se il mio MAC è di soli 64 bit (invece di 128), l'attaccante potrebbe sempre ricostruire un messaggio da accodare...ma ha una possibilità su $2^{64}$ che il MAC ottenuto sia quello corretto.
\\
Una terza soluzione è quella di inserire il segreto $K_{AB}$ sia all'inizio che in coda del messaggio da mandare al digest MD, così che $K_{AB}$ in testa fornisca resistenza alle collisioni e $K_{AB}$ rende innocua la vulnerabilità insita negli algoritmi di hash.
Dato che ognuna delle soluzioni è equivalentemente corretta per ovviare al problema della vulnerabilità della funzione MD, per dare una piattaforma condivisa...una soluzione condivisibile da tutti, è stato introdotto il framework \textbf{HMAC} (keyed-Hash Message Authentication Code); l'algoritmo HMAC calcola due volte il digest del messaggio, e ogni volta inserisce in testa il segreto $K_{AB}$.
\subsection{Cifratura/Decifratura} 
Come faccio a cifrare con un algoritmo di hash? Devo utilizzare la funzione di hash potendo rendere tutto il processo invertibile, quindi introducendo lo XOR tra un chunk del messaggio e l' hash della chiave (genero un keypad) e ottengo un cifrario a flusso!!
\begin{figure}
	\begin{center}
	{\includegraphics[height=13cm, width=13cm, keepaspectratio]{Immagini/Capitolo4/schema_cifratura_flusso.JPG}}
	\end{center}
\end{figure}

\subsection{Algoritmi di cifratura come algoritmi di hash}
Se un algoritmo di hash/digest, può essere utilizzato come un algoritmo di cifratura...può accadere il contrario?
In genere, questo principio è utilizzato (ad esempio) per la memorizzazione degli hash delle password in UNIX.
Ma come si modifica un algoritmo di cifratura che sia resistente alle collisioni?\\
Lo schema di funzionamento (minimale) è questa
\begin{figure}
	\begin{center}
	{\includegraphics[height=13cm, width=13cm, keepaspectratio]{Immagini/Capitolo4/schema_des_come_hash.JPG}}
	\end{center}
\end{figure}
Ma questo schema funziona per soli messaggi corti e non è troppo resistente alle collisioni, sebbene $K_{PWD}$ sia ottenuta dalla pwd considerandone i primi 8 caratteri ed espansa con con i bit di parità per ogni gruppo di 8 bit.
Dato s il \textit{salt}, invero il numero di 12 bit ottenuto in modo pseudo randomico dalla pwd di ogni utente, la funzione h(pwd) si ottiene dal DES modificato che dipende dal valore di s (che determina quali bit  devono essere replicati nella fase di espansione di R da 32 a 48 bit). Infine il DES modificato è utilizzato con $K_{PWD}$ pr cifrare una costante 0 a 64 bit. Il risultato della cifratura e il salt, sono memorizzati per poter recuperare pwd.

\subsection{Hash di grandi messaggi con algoritmi di cifratura come hash}
Per l'algoritmo precedente, era stato sottolineato che era valido esclusivamente per messaggi corti.
Come si deve operare per estendere questi algoritmi a messaggi di lunghezza arbitraria?
Si può pensare di suddividere il messaggio $m = m_{1},m_{2},m_{3},...,m_{n}$, ogni $m_{i}$ viene utilizzato come input a un blocco in cascata di cifratura, alla fine della cascata ho realizzato un algoritmo di hash tramite cifratura...per un messaggio di lunghezza qualsiasi.\\
\begin{figure}
	\begin{center}
	{\includegraphics[height=10cm, width=10cm, keepaspectratio]{Immagini/Capitolo4/schema_des_come_hash_1.JPG}}
	\end{center}
\end{figure}
Purtroppo questo sistema soffre degli stessi problemi di cui soffriva il DES doppio, per ovviare ciò, basta rendere l'output di ogni blocco differente diverso dall'input del successivo, una possibile implementazione è la seguente
\begin{figure}
	\begin{center}
	{\includegraphics[height=10cm, width=10cm, keepaspectratio]{Immagini/Capitolo4/schema_des_come_hash_2.JPG}}
	\end{center}
\end{figure}

\subsection{Rainbow Tables (Opzionale)}
Le Rainbow tables sono un meccanismo per l'attacco di reverse hash. Le tables bilanciano memoria e sforzo computazionale, se fosse utilizzata solo memoria si avrebbero dei database troppo grandi per poter mappare tutte le immagini degli hash e se fosse utilizzata solo potenza computazionale servirebbe troppo tempo per poter trovare l'inverso di un hash.\\
Le Rainbow tables utilizzano delle funzioni di riduzione r(), la funzione di riduzione è una applicazione dallo spazio degli Hash allo spazio dei Plaintext, r() non è l'inversa di un algoritmo di hash; per questo, prima di avvicinarmi a una buona inversione dovrò concatenare fino a centinaia di iterazioni di $h(m) = k \longrightarrow r(k) = m^{'}$.
\begin{figure}
	\begin{center}
	{\includegraphics[height=10cm, width=10cm, keepaspectratio]{Immagini/Capitolo4/riduzione.JPG}}
	\end{center}
\end{figure}
Per popolare la table, si memorizzano solo il messaggio iniziale e il messaggio finale della catena. In questo modo, con una catena di 100 iterazioni (dato che le funzioni sono deterministiche) si risparmia uno spazio in memoria di 100 volte la grandezza della table, in confronto a memorizzare solo coppie $\langle plaintext, hash \rangle$.
Per recuperare un plaintext di un hash, si controlla se questo sia un endpoint di una qualsiasi delle catene (fase di lookup), se non lo fosse si controllano gli endpoint intermedi di ogni catena.
Appena recuperato l'hash giusto, si utilizza la catena determinata dall'inizio finchè non si trova l'hash di inizio; il messaggio che ha generato quell'hash sarà ciò che si cercava.
\begin{figure}
	\begin{center}
	{\includegraphics[height=10cm, width=10cm, keepaspectratio]{Immagini/Capitolo4/riduzione.JPG}}
	\end{center}
\end{figure}
\chapter{Crittografia a chiave pubblica}
<<<<<<< HEAD

La crittografia a chiave pubblica si basa su alcuni risultati nell'ambito della teoria dei numeri. Si esamineranno i seguenti schemi di cifratura a chiave pubblica: \begin{itemize}
||||||| merged common ancestors

%%Scrivere normalmente, SENZA inserire begin e end document, che sono gi� compresi nel file principale da compilare
La crittografia a chiave pubblica si basa su alcuni risultati nell'ambito della teoria dei numeri. Si esamineranno i seguenti schemi di cifratura a chiave pubblica: \begin{itemize}
=======
La crittografia a chiave pubblica si basa su alcuni risultati nell'ambito della teoria dei numeri. Si esamineranno i seguenti schemi di cifratura a chiave pubblica: 
\begin{itemize}
>>>>>>> 97170125928e81d41229ce7a4b1ea69a00151de4
\item RSA usata per cifrare e per calcolare la firma digitale
\item ElGamal e DSS, usati per la firma digitale
\item Diffie-Hellman, permette di stabilire un segreto condiviso, ma non fornisce alcun algoritmo che usa effettivamente tale segreto
\end{itemize}
<<<<<<< HEAD
L'unico aspetto comune a tutti gli algoritmi di crittografia a chiave pubblica è la presenza di due quantità correlate: una chiave segreta e una chiave pubblica.
||||||| merged common ancestors
L'unico aspetto comune a tutti gli algoritmi di crittografia a chiave pubblica � la presenza di due quantit� correlate: una chiave segreta e una chiave pubblica.
=======

L'unico aspetto comune a tutti gli algoritmi di crittografia a chiave pubblica è la presenza di due quantità correlate: una chiave segreta e una chiave pubblica.
>>>>>>> 97170125928e81d41229ce7a4b1ea69a00151de4
\section{Aritmetica modulare}
La maggior parte degli algoritmi a chiave pubblica si basano sull'aritmetica modulare: fissato un intero $ n>1 $, l'aritmetica modulare considera l'insieme degli interi non negativi minori di
$n: $\{$0, 1, 2,..., n-1$\}, effettua operazioni ordinarie come l'addizione e la moltiplicazione, e sostituisce il risultato $x$ con il resto $r$ della divisione intera di $x$ per $n$. Il risultato finale viene detto modulo n o $mod \, n$. \\
<<<<<<< HEAD
Definiamo l'inverso moltiplicativo di $k$, indicato con $k^{-1}$, come quel numero che moltiplicato per $k$ dà 1, cioè $kk^{-1} \, mod \, n = 1 \, mod \, n$. Fissato n, non tutti i numeri hanno un inverso moltiplicativo $mod \, n$. Si osservi inoltre che, se $k$ ammette un inverso moltiplicativo $mod \, n $, esiste un unico inverso moltiplicativo $k^{-1} < n$. La moltiplicazione $mod \, n$ di per sé non costituisce un cifrario sicuro,ma funziona, nel senso che la moltiplicazione per $k$ produce un mescolamento dell'input; la decifratura può ottenersi moltiplicando per $k^{-1}$.\\ Trovare un inverso moltiplicativo $k^{-1}$ nella aritmetica $mod \, n$, non è affatto banale se $n$ è molto grande. Esiste un modo efficiente per risolvere tale problema, noto come algoritmo di Euclide: \begin{itemize}
\item dati $x$ ed $n$, con $x<n$, l'algoritmo di Euclide trova il
numero $y<n$ tale che $xy \, mod \, n = 1$, ammesso che un siffatto $y$ esista.\end{itemize} 
Quali sono dunque gli inversi moltiplicativi $mod \, n$? E' sufficiente trovare i numeri \textit{relativamente primi} con $n$, cioè tali che se $x$ è uno di questi numeri, si ha che $MCD(x,n) = 1$. Se $n$ è un numero primo, tutti gli interi positivi $x<n$ ammettono un inverso moltiplicativo $mod \, n$, che indichiamo con $x^{-1}$.\\ \\
||||||| merged common ancestors
Definiamo l'inverso moltiplicativo di $k$, indicato con $k^{-1}$, come quel numero che moltiplicato per $k$ d� 1, cio� $kk^{-1} \, mod \, n = 1 \, mod \, n$. Fissato n, non tutti i numeri hanno un inverso moltiplicativo $mod \, n$. Si osservi inoltre che, se $k$ ammette un inverso moltiplicativo $mod \, n $, esiste un unico inverso moltiplicativo $k^{-1} < n$. La moltiplicazione $mod \, n$ di per s� non costituisce un cifrario sicuro,ma funziona, nel senso che la moltiplicazione per $k$ produce un mescolamento dell'input; la decifratura pu� ottenersi moltiplicando per $k^{-1}$.\\ Trovare un inverso moltiplicativo $k^{-1}$ nella aritmetica $mod \, n$, non � affatto banale se $n$ � molto grande. Esiste un modo efficiente per risolvere tale problema, noto come algoritmo di Euclide: \begin{itemize}
\item dati $x$ ed $n$, con $x<n$, l'algoritmo di Euclide trova il
numero $y<n$ tale che $xy \, mod \, n = 1$, ammesso che un siffatto $y$ esista.\end{itemize} 
Quali sono dunque gli inversi moltiplicativi $mod \, n$? E' sufficiente trovare i numeri \textit{relativamente primi} con $n$, cio� tali che se $x$ � uno di questi numeri, si ha che $MCD(x,n) = 1$. Se $n$ � un numero primo, tutti gli interi positivi $x<n$ ammettono un inverso moltiplicativo $mod \, n$, che indichiamo con $x^{-1}$.\\ \\
=======
Definiamo l'inverso moltiplicativo di $k$, indicato con $k^{-1}$, come quel numero che moltiplicato per $k$ dà 1, cioè $kk^{-1} \, mod \, n = 1 \, mod \, n$. Fissato n, non tutti i numeri hanno un inverso moltiplicativo $mod \, n$. Si osservi inoltre che, se $k$ ammette un inverso moltiplicativo $mod \, n $, esiste un unico inverso moltiplicativo $k^{-1} < n$. La moltiplicazione $mod \, n$ di per sé non costituisce un cifrario sicuro,ma funziona, nel senso che la moltiplicazione per $k$ produce un mescolamento dell'input; la decifratura può ottenersi moltiplicando per $k^{-1}$.\\ Trovare un inverso moltiplicativo $k^{-1}$ nella aritmetica $mod \, n$, non è affatto banale se $n$ è molto grande. Esiste un modo efficiente per risolvere tale problema, noto come algoritmo di Euclide: 

\begin{itemize}
\item dati $x$ ed $n$, con $x<n$, l'algoritmo di Euclide trova il numero $y<n$ tale che $xy \, mod \, n = 1$, ammesso che un siffatto $y$ esista.
\end{itemize} 
Quali sono dunque gli inversi moltiplicativi $mod \, n$? E' sufficiente trovare i numeri \textit{relativamente primi} con $n$, cioè tali che se $x$ è uno di questi numeri, si ha che $MCD(x,n) = 1$. Se $n$ è un numero primo, tutti gli interi positivi $x<n$ ammettono un inverso moltiplicativo $mod \, n$, che indichiamo con $x^{-1}$.\\ \\
>>>>>>> 97170125928e81d41229ce7a4b1ea69a00151de4
\textbf{Funzione di Eulero o totiente:} \{$i \in Z, 0 < i < n: MCD(i,n) = 1$\} \\

Se $n$ è primo, tutti gli interi da 1 a $n-1$ sono relativamente primi con $n$, per cui $\phi(x) = n - 1$. Se $n = pq$, dove p e q sono numeri primi maggiori di 1, allora $\phi(n) = (p-1)(q-1)$. \\
Sia adesso $n > 1$ un intero privo di quadrati(cioè dove non compaiono fattori al quadrato nella sua scomposizione in fattori primi), detto anche square free, allora per ogni $y > 0$, si ha: \begin{center}
$x^y \, mod \, n = x^{(y + \phi(n) )} \, mod \, n$
\end{center} Ne segue che, se $ y = 1 \, mod \, \phi(n) $, ovvero se $y = 1 + k \phi(n) $, con $k \in Z$, si ha $x^y \, mod \, n = x \, mod \, n $. Tale risultato è sfruttato dall'algoritmo RSA.

\section{RSA}

RSA è un algoritmo di cifratura(a blocchi) a chiave pubblica, la cui lunghezza è variabile(solitamente si considerano chiavi di lunghezza pari ad almeno 512 bit). Anche la lunghezza dei blocchi è variabile: un blocco di testo in chiaro deve avere lunghezza minore di quella della chiave, mentre un blocco di testo cifrato è lungo come la chiave. \\
RSA è computazionalmente molto più lento degli algoritmi a chiave segreta più popolari come DES, IDEA e AES, per cui difficilmente viene usato per cifrare messaggi lunghi. Generalmente viene usato per cifrare una chiave segreta $K$, utilizzata per cifrare un messaggio usando un algoritmo a chiave segreta. RSA può essere usato dunque sia per cifrare/decifrare messaggi sia per la firma digitale di messaggi. In entrambi i casi bisogna disporre della coppia $<chiave \, pubblica, \,  chiave \, privata>(<PU,PR>)$. \\
I passi da seguire per generare la chiave pubblica e la chiave privata sono: \begin{enumerate}
\item Scegliere due numeri primi $p$ e $q$ molto grandi(circa 256 bit ciascuno) tali che $n = pq$. E' fondamentale che $p$ e $q$ rimangano segreti, cosicchè fattorizzare $n$ sia computazionalmente impraticabile. 
\item Scegliere un numero $e$ che sia relativamente primo rispetto a $\phi(n) = (p-1)(q-1)$.
\item Calcolare l'inverso moltiplicativo $d$ di $e \, mod \, \phi(n)$, cioè tale che sia $(d \cdot e ) \, mod \, \phi(n) \, = 1$.
\item La chiave pubblica è $PU = <e,n>$, mentre la chiave privata è $PR = <d,n>$.
\end{enumerate}
Per quanto riguarda la \textbf{cifratura}/\textbf{decifratura}, siano $PU$ e $PR$ la chiave pubblica e la chiave privata del destinatario, $m$ il messaggio da cifrare. La procedura da seguire è la seguente: \begin{itemize}
\item il mittente, utilizzando la chiave pubblica $PU$ del destinatario, cifra il messaggio ottenendo $c = m^e \, mod \, n$
\item il destinatario, usando la propria chiave privata $PR$, decifra $c$ calcolando $m = c^d \, mod \, n$. \end{itemize}
\textbf{Dimostrazione.} Poste le seguenti proprietà: \begin{itemize}
\item se $m<n$, $ \, m \, mod \, n\,=\,n$
\item $(x^a \, mod \, n)^b \, mod \, n \, = \, x^{ab} \, mod \, n$
\item $(e \cdot d) \, mod \, \phi(n) = 1$
\end{itemize}
Se $c=E(PU,m) = m^e \, mod \, n \, \Rightarrow m=D(PR,c) = c^d \, mod \, n = (m^e \, mod \, n)^d \, mod \, n = m^{ed} \, mod \, n = m^{1 \, mod \, \phi(n)} \, mod \, n = m \, mod \, n = m$. \\ \\
Per la \textbf{firma digitale} invece sia $PR$ la chiave privata del firmatario del messaggio e sia $PU$ la sua chiave pubblica: \begin{itemize}
\item il firmatario, usando la propria chiave privata $PR$, calcola la firma digitale $s = m^d \, mod \, n$;
\item chiunque desideri verificare l'autenticità della firma, può farlo usando la chiave pubblica $PU$ del firmatario e calcolando $m = s^e \, mod \, n$.
\end{itemize}
La dimostrazione è per la firma è analoga alla cifratura. \\

La sicurezza di RSA deriva dal fatto che fattorizzare interi molto grandi è impraticabile. Infatti, identificando i numeri primi $p$ e $q$ tali che $n = pq$, si ottiene $\phi(n) = (p-1)(q-1)$ e quindi si può calcolare $d$ come l'inverso moltiplicativo di $e \, mod \, \phi(n)$, ottenendo la chiave privata $PR = <d,n>$ dalla chiave pubblica $PU = <e,n>$. \\ 
Tuttavia è possibile violare RSA senza ricorrere alla fattorizzazione, se lo si usa in modo improprio. \\
In base al tipo di impiego, RSA svolge le seguenti operazioni molto frequentemente (ad ogni sessione di lavoro): \begin{itemize}
\item cifratura/decifratura
\item generazione/verifica di una firma digitale
\end{itemize}
E' necessario pertanto che tali operazioni siano svolte nel modo più efficiente possibile. Invece, l'operazione di generazione delle chiavi viene eseguita meno frequentemente e quindi si può tollerare una minore efficienza.\\
Le operazioni di cifratura, decifratura, firma e verifica della firma richiedono tutte di dover considerare un intero molto grande, elevarlo ad un esponente(intero) molto grande e trovare il resto della divisione intera per un numero molto grande. Considerando la dimensione dei numeri interi per i quali RSA è ritenuto sicuro, tali operazioni
risulterebbero proibitive se eseguite nel modo più ovvio. \\
\subsection{Generazione delle chiavi RSA}
La generazione delle chiavi RSA è un'operazione poco frequente: in gran parte delle applicazioni della tecnologia a chiave pubblica deve essere eseguita soltanto una volta e non è richiesta la stessa efficienza delle altre operazioni RSA; deve comunque essere garantita un'efficienza ragionevole. \\
Per generare una coppia di chiavi $<PU,PR>$ è necessario trovare due numeri primi $p$ e $q$ molto grandi e trovare due interi $d$ ed $e$ con le proprietà precedentemente descritte. \\ \\
\textbf{Trovare due numeri primi grandi p e q.} Esistono infiniti numeri primi, che diminuiscono all'aumentare di $n$: estraendo un numero a caso, si ha che $Pr \{ n \, primo \} \approx 1/{ln \, n} \approx 1/N_{b}$, dove $N_{b}$ è il numero di bit utilizzato per rappresentare $n$. La densità dei numeri primi è inversamente proporzionale alla loro lunghezza in bit(o in cifre decimali). Ad esempio, per un numero $n$ a cento cifre decimali (dimensione usata in RSA), c'è una possibilità su 230 che esso sia primo. \\ Pertanto, i passi da seguire per generare $p$ e $q$ sono i seguenti: \begin{enumerate}
\item estrai un numero dispari molto grande
\item verifica se tale numero è primo, in caso negativo ritenta(in media, sono necessari 230 tentativi per ottenere un numero primo)
\end{enumerate}
Tale strategia va bene se si dispone di un test di primalità efficiente: come è possibile testare se un intero $n$ è primo? Un metodo banale consiste nel dividere $n$ per tutti gli interi $ \le n^{1/2}$ e verificare che non ci sono divisori $> 1$, ma ciò richiederebbe diverse vite dell'universo! \\ RSA utilizza un test di primalità probabilistico, cioè non si può affermare con certezza che l'esito del test sia corretto. Tuttavia, la probabilità di errore può essere resa
arbitrariamente piccola aumentando il tempo di test. Il test si basa sul teorema di Fermat che fornisce una condizione necessaria affinché un intero $n$ sia primo: \begin{itemize}
\item se $n$ è primo $ \Rightarrow$ per ogni intero $a$ risulta $a^{n-1} = 1 \, mod \, n $;
\item non vale però il viceversa: esistono degli interi $a$ per i quali, l'uguaglianza è verificata anche se $n$ è non primo \end{itemize} 
\textbf{Test di primalità probabilistico}. Dato un intero $n$, un possibile test di primalità può consistere dei seguenti passi: \begin{enumerate}
\item scegliere un intero $a<n$;
\item calcolare $a^{n-1} \, mod \,n$;
\item \begin{enumerate} \item [a.] se il risultato è diverso da 1 $ \Rightarrow n$ è certamente non primo.
\item [b.] se il risultato è pari a 1 $\Rightarrow $ $n$ potrebbe essere primo, anche se non è sicuro(è stato dimostrato che, se $n$ è un intero random di circa cento cifre decimali, la probabilità di un falso positivo è $10^{-13}$).
\end{enumerate}
\end{enumerate}
Si osservi che un errore nel test di primalità può  rendere impossibile la decifratura RSA di un messaggio, più facile l'identificazione della chiave privata. \\Se una probabilità di errore pari a $10^{-13}$ non è ritenuta sufficiente, si possono effettuare più test con diversi valori di $a$: si ha che la $Pr \{falso \, positivo \, dopo \, k \, test \} = (10^{-13})^k$. La probabilità di errore può essere resa arbitrariamente piccola, ma non sempre è facile! Infatti, ci possono essere dei casi veramente sfortunati non rilevabili dal test, ad esempio se $n$ è un numero di Carmichael: un numero $n$ è detto di Carmichael se non è primo e se per ogni $a \le n$ risulta $a^{n-1} = 1 \, mod \, n$. Tuttavia, i numeri di Carmichael sono sufficientemente rari che è estremamente improbabile estrarli a caso. \\ \\
\textbf{Calcolo di d ed e}. Gli interi $d$ ed $e$ sono definiti nel seguente modo: \begin{itemize}
\item $e$ è un qualunque numero relativamente primo rispetto all'intero $\phi(n) = (p-1)(q-1)$;
\item $d$ è l'intero tale che $ed \, mod \, \phi(n) = 1 \Rightarrow$ noto $e$, $d$ si calcola con l'algoritmo di Euclide. 
\end{itemize}
Esistono due strategie per il calcolo di $e$:
\begin{enumerate}
\item una volta ottenuti $p$ e $q$, si sceglie randomicamente $e$ e si testa se esso è relativamente primo con $(p-1)(q-1)$; in caso negativo si ritenta con un altro valore di $e$.
\item Non selezionare prima $p$ e $q$, al contrario, si sceglie prima $e$, per poi scegliere $p$ e $q$ tali che la quantità $(p-1)(q-1)$ sia relativamente prima con $e$.
\end{enumerate}
La sicurezza di RSA non viene messa in crisi se $e$ è scelto sempre allo stesso modo: $d$ continua ad essere imprevedibile se $p$ e $q$ non sono noti. Se $e$ è un intero piccolo o facile da calcolare, le
operazioni di cifratura e di verifica della firma diventano più efficienti, cioè le operazioni che richiedono l'uso della chiave pubblica $PU = <e,n>$ sono più veloci, mentre risulta invariata l'efficienza delle operazioni che richiedono la chiave privata $PR = <d,n>$. Chiaramente, diversamente da $e$, non si può assegnare
a $d$ un valore piccolo, sebbene ciò renderebbe molto più veloci le operazioni che usano la chiave privata $PR$. Infatti, la sicurezza di RSA verrebbe meno: così si renderebbe l'informazione vulnerabile ad attacchi a forza bruta, poichè $d$ è l'esponente privato, a differenza di $e$ che è esponente pubblico.
\\ Di solito si usa $e=3$, poichè è comodo lavorare con esponenti piccoli, in modo che il calcolo di $m^e \, mod \, n$ non sia computazionalmente costoso(il calcolo di $ m^3 \, mod \, n$ richiede soltanto due moltiplicazioni) e la cifratura sia efficiente(così come la verifica della firma). Non si può scegliere $e=2$, in quanto non è relativamente primo con $(p-1)(q-1)$, che è un numero pari. \\
La scelta $e=3$ comporta alcune vulnerabilità: \begin{itemize}
\item se il messaggio $m$ da cifrare rappresenta un intero piccolo, in particolare se $m < n^{1/3} \Rightarrow  c = m^e \, mod \, n = m^3 \, mod \, n = m^3 \Rightarrow$ un avversario può decifrare $c$ senza conoscere la chiave privata semplicemente estraendo la radice
cubica ordinaria di $c \Rightarrow m = c^{1/3}$. Tale vulnerabilità può essere rimossa eseguendo un padding random del messaggio tale che $m^3 > n$. Ciò garantisce che $m^3$ viene sempre ridotto $ mod \, n $.
\item se uno stesso messaggio $m$ viene inviato cifrato a tre o più destinatari aventi un esponente pubblico $e=3$, il messaggio in chiaro $m$ può essere decifrato conoscendo soltanto i tre messaggi cifrati $c_{1}$, $c_{2}$ e $c_{3}$ e le tre chiavi pubbliche $<3,n_{1}>$, $<3,n_{2}>$ e $<3,n_{3}>$: si supponga infatti che un avversario intercetti tre cifrature dello stesso messaggio $m$, cioè $c_{1}$, $c_{2}$ e $c_{3}$. Conoscendo anche le tre chiavi pubbliche $<3,n_{1}>$, $<3,n_{2}>$ e $<3,n_{3}>$ e utilizzando il teorema cinese del resto, l'avversario può calcolare  $m^3 \, mod \, n_{1} \, n_{2} \, n_{3}$. Essendo $m<n_{i}$, per $i=1,2,3 \Rightarrow m^3<n_{1},n_{2},n_{3}$, da cui si ricava che $m^3 \, mod \, n_{1} \, n_{2} \, n_{3} = m^3 \Rightarrow$ l'avversario può risalire ad $m$ estraendo una radice cubica ordinaria. Anche questa vulnerabilità può essere rimossa
mediante un padding random, così si evita che uno stesso messaggio cifrato venga inviato a piu destinatari. 
\end{itemize}
Si osservi che nelle applicazioni pratiche di RSA,il messaggio $m$ è generalmente una chiave di un algoritmo di cifratura a chiave segreta e in ogni caso $m$ è molto più piccolo di $n$, per cui è sempre possibile aggiungere dei bit di riempimento(padding) in modo tale che il messaggio risultante presenti delle caratteristiche desiderate. Se per ogni destinatario il padding scelto è random, la precedente vulnerabilità viene rimossa; la vulnerabilità può essere rimossa anche usando come padding gli identificatori univoci (ID) dei
destinatari. \\
Un'altra scelta possibile è $e=65537$. Infatti esso è pari a $2^{16}+1$, che è un numero primo, e rimuove o riduce del tutto le vulnerabilità viste nel caso $e=3$: la prima vulnerabilità con $e=3$ si ha se $m^3<n$ e nel caso $e=65537$ non ci sono molti valori di $m$ tali che $m^{65537} < n$, a meno che $n$ non sia molto più lungo di 512 bit, quindi l'estrazione della 65537-esima radice ordinaria di $m$ non costituisce una vulnerabilità seria; la seconda vulnerabilità con $e=3$ si ha se uno stesso messaggio $m$ cifrato è inviato a 3 destinatari e nel caso $e=65537$, lo stesso tipo di vulnerabilità si ha quando $m$ viene inviato a 65537 destinatari e non si può dire certo che si tratti di un messaggio segreto!. \\ Infine, la scelta di fissare a priori $e=3$ ha richiesto di scegliere $n$ in modo tale che $\phi(n)$ e 3 fossero relativamente primi. Nel caso $e=65537$ conviene generare $p$ e $q$ come se $e$ non fosse prefissato e rigettare ogni valore di $p$ o $q$ che è uguale a $1 \, mod \: 65537$. Tale evento si verifica con una probabilità molto piccola, cioè $2^{-16}$.\\
Sono presenti altri tipi di vulnerabilità: nel caso della firma digitale risulta che, per ogni numero $x < n$, $x$ è la firma digitale del messaggio $m_{x} = x^e \,  mod \, n$, infatti, $m_{x}^d \, mod \, n = (x^e \, mod \, n)^d \, mod \, n = x^{ed} \, mod \, n = x^{1 \, mod \, \phi(n)} \, mod \, n = x \, mod \, n = x \Rightarrow$ è banale falsificare la firma di qualcuno se il messaggio $m$ da firmare non interessa. La difficoltà sta però nel falsificare la firma di uno
specifico messaggio. \\ Generalmente, ciò che viene firmato(messaggio + padding) ha una struttura sufficientemente vincolata: vengono inseriti dei bit di riempimento organizzati in pattern regolari;la probabilità che un numero random costituisca un messaggio (padding incluso) valido è trascurabile, cioè è estremamente improbabile che un numero random contenga i pattern regolari di bit. Tuttavia, visto che i numeri in RSA sono molto grandi un avversario ha a disposizione molti tentativi, dunque i pattern di riempimento vanno scelti in modo opportuno. \\ Si fa utilizzo dunque degli \textit{smooth numbers}. Intuitivamente, uno smooth number è un numero scomponibile nel prodotto di (molti) numeri primi ragionevolmente piccoli(non conviene usare una definizione assoluta, ovvero un numero è piccolo o grande in base alle capacità di calcolo dell'avversario). Ad esempio, il numero 6056820 è più smooth del numero 6567587, poiché $6056820 = 22 \cdot 32 \cdot 5 \cdot 7 \cdot 11 \cdot 19 \cdot 23$, mentre $ 6567587 = 13 \cdot 557 \cdot 907$. Si tratta di una vulnerabilità prevalentemente teorica, nella pratica difficilmente realizzabile, poichè richiede un'enorme capacità di calcolo, la raccolta di un numero elevato di messaggi firmati e molta fortuna(per l'avversario). \\ \\
\textbf{Idea base}: dalle firme $s_{1}$ e $s_{2}$ dei messaggi $m_{1}$ ed $m_{2}$, è possibile calcolare le firme dei messaggi $m_{1} \cdot m_{2}$, $m_{1}/m_{2}$, $m_{1}^j$, $m_{2}^k$ e $m_{1}^j \cdot m_{2}^k$. Ad esempio, conoscendo la firma $s_{1} = m_{1}^d \, mod \, n $, è possibile ottenere la firma di $m_{1}^2$ senza conoscere $d$(chiave privata): infatti, $(m_{1}^2)^d \, mod \, n = (m_{1}^d)^2 \, mod \, n = (m_{1}^d \, mod \, n)^2 \, mod \, n $, ottenendo quindi la firma di $m_{1}^2$. Se un avversario riesce a collezionare molti messaggi firmati, può ottenere la firma di ogni messaggio $m$ esprimibile come prodotto e/o divisione di messaggi della collezione. In particolare, se ottiene le firme di due messaggi $m_{1}$ e $m_{2}$ tali che il rapporto $m_{1}/m_{2}=p$, dove $p$ è un numero primo, l'attaccante può calcolare la firma di $p$. Inoltre, se è abbastanza fortunato da raccogliere molte coppie di questo tipo, egli può calcolare la firma di molti numeri primi, quindi può falsificare la firma di ogni messaggio dato dal prodotto di ogni sottoinsieme di tali numeri primi ciascuno elevato ad una qualunque potenza. Con abbastanza coppie, può falsificare la firma di ogni messaggio rappresentato da uno smooth number. \\ Generalmente, ciò che si firma con RSA è un digest
messaggio con padding $m^{*} = pad(h(m))$. Al digest del messaggio $m$ vengono aggiunti, in modo opportuno, dei bit di riempimento(padding) ottenendo $m^*$. Se i bit di riempimento sono degli zeri, anzichè essere random, è piu probabile che $m^*$ sia uno smooth number. Invece, è estremamente improbabile che un numero random $mod \, n$ sia smooth: \begin{itemize}
\item con un padding a sinistra di soli zeri, l'intero da firmare $m_{p} = h(m)$ rimane piccolo, per cui il padding non riduce la probabilità che $m_{p}$ sia smooth;
\item con un padding a destra di soli zeri, $m_{p} = h(m) \cdot 2^k $ è un intero molto più grande, ma è divisibile per una potenza di due, quindi, analogamente, il padding non riduce la probabilità che $m_{p}$ sia smooth;
\item con un padding a destra random, l'intero da firmare $m_{p}$ è estremamente improbabile che sia smooth. 
\end{itemize}
Tuttavia, si espone RSA alla minaccia nota come il
problema della radice cubica: si assuma che si è optato per padding a destra random per ridurre la probabilità che le firme prodotte siano smooth. Si ha l'inconveniente che, se l'esponente pubblico $e = 3$, allora un attaccante può virtualmente falsificare la
firma di un qualsiasi messaggio. Infatti, supponiamo che un attaccante, Carol, voglia falsificare la firma di un qualche messaggio $m$ avente digest $h_{m}$. Allora Carol applica un padding a destra di $h_{m}$, considerando bit a zero e ottenendo $p_{m} = h_{m}00..00$. Poi calcola la radice cubica ordinaria e la arrotonda all'intero più vicino $r = round(p_{m}^{1/3})$, ottenendo la firma falsificata di $m$(infatti, $r^e = r^3 = p_{m}$, ossia $h_{m}$ con un padding a destra che è apparentemente casuale).

\section{PKCS}

Ogni applicazione di RSA, cifratura, decifratura e firma, può essere soggetta a diversi tipi di attacchi, che possono essere sventati con opportune contromisure, basate sulla scelta di un'opportuna codifica/formato(quindi padding) del messaggio da cifrare/firmare. A tal fine è stato definito uno standard, \textbf{PKCS}(Public-Key Cryptography Standard), che stabilisce le codifiche per la chiave pubblica RSA, chiave privata RSA, firma RSA, cifratura RSA di messaggi corti(cioè chiavi segrete), firma RSA di messaggi corti(tipicamente digest). \\
Esistono 15 standard PKCS per le diverse situazioni in cui la cifratura a chiave pubblica viene utilizzata. Noi esamineremo solo PKCS1. Esso è stato concepito per far fronte alle seguenti minacce:
\begin{itemize}
\item cifratura di messaggi prevedibili;
\item smooth number per le firme;
\item destinatari multipli di un messaggio quando $e=3$;
\item cifratura di messaggi di lunghezza inferiore ad un terzo della lunghezza di $n$ quando $e=3$;
\item firma di messaggi dove l'informazione è posta nei bit più significativi ed $e = 3$.
\end{itemize}
PKCS definisce uno standard per la formattazione di un messaggio da cifrare con RSA. \\
PKCS definisce uno standard per la formattazione di un messaggio da firmare con RSA.

\section{Diffie-Hellman}

Diffie-Hellman è il primo sistema a chiave pubblica utilizzato. Meno generale di RSA, non serve né a cifrare/decifrare né a firmare messaggi ma permette lo scambio di chiavi, in chiaro e su una rete pubblica insicura, tra due entità(che chiameremo Alice e Bob) e quindi di accordarsi su un segreto(chiave) condiviso, senza rivelarlo. Intercettando tutti i messaggi scambiati non si è in grado di risalire al segreto condiviso. Tale segreto non viene generato da una delle due entità, ma è il risultato dello scambio dei messaggi. In particolare, dopo essersi scambiati complessivamente due messaggi(in chiaro), che tutto il mondo può conoscere, Alice e Bob conosceranno il segreto condiviso $K_{AB}$, il quale verrà poi usato per proteggere la confidenzialità con tecniche di cifratura convenzionali. \\ Diffie-Hellman è realmente usato per stabilire una chiave segreta condivisa in alcune applicazioni, ad esempio nell'ambito della cifratura dei dati inviati in una LAN. Si osservi comunque che Diffie-Hellman non
incorpora alcuna forma di autenticazione, senza la quale si rischia di condividere il segreto con un impostore.  \\ \\
\textbf{Algoritmo}: \\

PRECONDIZIONE: 
\begin{itemize}
\item Alice e Bob condividono due numeri $p$ e $g$, dove $p$ è un numero primo grande e $g<p$ con $g$ radice primitiva di $p$
\item $p$ e $g$ sono noti in anticipo e possono essere resi di dominio pubblico in una repository accessibile sia da Alice che da Bob, oppure possono essere generati dall'iniziatore della comunicazione,
diciamo Alice, e trasmessi a Bob nel messaggio che gli invierà;
\end{itemize}
La fase 0, viene eseguita dall'iniziatore della comunicazione qualora i numeri $p$ e $g$ non siano pubblici: Alice genera (o estrae da un suo archivio) una coppia di numeri $p$ e $g$ tali da soddisfare le proprietà sopra elencate; $p$ e $g$ possono essere subito inviati a Bob oppure possono essere trasmessi nella fase 3a.
\begin{enumerate}
\item \begin{enumerate}
<<<<<<< HEAD
\item [a.] Generazione del segreto privato $s_{A}$: Alice(iniziatore comunicazione) genera un numero random $s_{A}<p$ di 512 bit, che non verrà mai inviato a Bob, quindi calcola $T_{A}=g^{s_{A}} \, mod \,p$ e lo invia(su una rete insicura) a Bob;
\item [b.] Generazione del segreto privato $s_{B}$: Bob genera $s_{B}<p$ di 512 bit, che non verrà mai inviato ad Alice, quindi calcola $T_{B}=g^{s_{B}} \, mod \, p$ e lo invia in rete ad Alice;
||||||| merged common ancestors
\item [a.] Generazione del segreto privato $s_{A}$: Alice(iniziatore comunicazione) genera un numero random $s_{A}<p$ di 512 bit, che non verr� mai inviato a Bob, quindi calcola $T_{A}=g^(s_{A}) \, mod \,p$ e lo invia(su una rete insicura) a Bob;
\item [b.] Generazione del segreto privato $s_{B}$: Bob genera $s_{B}<p$ di 512 bit, che non verr� mai inviato ad Alice, quindi calcola $T_{B}=g^{s_{B}} \, mod \, p$ e lo invia in rete ad Alice;
=======
\item [a.] Generazione del segreto privato $s_{A}$: Alice(iniziatore comunicazione) genera un numero random $s_{A}<p$ di 512 bit, che non verrà mai inviato a Bob, quindi calcola $T_{A}=g^(s_{A}) \, mod \,p$ e lo invia(su una rete insicura) a Bob;
\item [b.] Generazione del segreto privato $s_{B}$: Bob genera $s_{B}<p$ di 512 bit, che non verrà mai inviato ad Alice, quindi calcola $T_{B}=g^{s_{B}} \, mod \, p$ e lo invia in rete ad Alice;
>>>>>>> 97170125928e81d41229ce7a4b1ea69a00151de4
\end{enumerate}
\item \begin{enumerate}
\item [a.] Alice calcola $K_{AB}=T_{B}^{s_{A}} \, mod \,p$;
\item [b.] Bob calcola $K_{BA}=T_{B}^{s_{B}} \, mod \, p$;
\end{enumerate}
\item L'aritmetica modulare garantisce che $K_{AB}=K_{BA}$. Infatti si ha $ K_{AB} = T_{B}^{s_{A}} \, mod \,p = ({g^{s_{B}} \, mod \, p})^{s_{A}} \, mod \,p = g^{s_{B}s_{A}} \, mod \, p = g^{s_{A}s_{B}} \, mod \, p = ({g^{s_{A}} \, mod \, p})^{s_{B}} \, mod \,p = T_{A}^{s_{B}} \, mod \, p = K_{BA}$.
\end{enumerate}

<<<<<<< HEAD
Tuttavia, dati $p$, $g$, $T_{A}$ e $T_{B}$, è possibile calcolare $s_{A}$, $s_{B}$ oppure $g^{s_{A}s_{B}}$? Si può ottenere $s_{A}$ da $g^{s_{A}}$ tramite logaritmo discreto $dlog_{g} \, g^{s_{A}}$, ma al momento non sono note tecniche per calcolare tale quantità in un tempo ragionevole, anche conoscendo $g^{s_{A}}$ e $g^{s_{B}}$.
||||||| merged common ancestors
Tuttavia, dati $p$, $g$, $T_{A}$ e $T_{B}$, � possibile calcolare $s_{A}$, $s_{B}$ oppure $g^{s_{A}s_{B}}$? Si pu� ottenere $s_{A}$ da $g^(s_{A})$ tramite logaritmo discreto $dlog_{g} \, g^{s_{A}}$, ma al momento non sono note tecniche per calcolare tale quantit� in un tempo ragionevole, anche conoscendo $g^{s_{A}}$ e $g^{s_{B}}$.
=======
Tuttavia, dati $p$, $g$, $T_{A}$ e $T_{B}$, è possibile calcolare $s_{A}$, $s_{B}$ oppure $g^{s_{A}s_{B}}$? Si può ottenere $s_{A}$ da $g^(s_{A})$ tramite logaritmo discreto $dlog_{g} \, g^{s_{A}}$, ma al momento non sono note tecniche per calcolare tale quantità in un tempo ragionevole, anche conoscendo $g^{s_{A}}$ e $g^{s_{B}}$.
>>>>>>> 97170125928e81d41229ce7a4b1ea69a00151de4
\\
La vulnerabilità di Diffie-Hellman è che tale algoritmo non fornisce alcuna prova di autenticazione: Alice non può essere certa che $T_{B}$ sia stato inviato da Bob e non da un impostore; allo stesso modo, Bob non può essere certa che $T_{A}$ sia stato inviato da Alice e non da
un impostore. E' quindi possibile che un impostore, Mr. X, intercetti e modifichi i messaggi facendo credere a Bob di comunicare con Alice e
viceversa. Alice e Bob non hanno modo di rendersi conto dell'attacco in atto. Un attacco di questo tipo viene detto Man-in-the-Middle(o Bucket Brigade Attack): Alice pensa che $K_{AX}$ sia la chiave segreta $K_{AB}$ che condivide con Bob; Bob pensa che $K_{bX}$ sia la chiave segreta $K_{BA}$ che condivide con Alice; invece, Mr. X ha due chiavi segrete: \begin{itemize}
\item $K_{XA}$ per comunicare con Alice;
\item $K_{XB}$ per comunicare con Bob.
\end{itemize}
Questo perchè Mr. X conosce $g$ e $p$, quindi si calcola $s_{x}<p$ e $T_{x}=g^{s_{x}} \, mod \, p$ e li invia sia ad Alice che a Bob. Tale attacco è un attacco all'integrità. \\
Come risolvere tale problema? 
\begin{itemize}
\item \textbf{Autenticazione via password}: \\ Ipotesi: Alice e Bob si sono preliminarmente accordate su una coppia di password($pwd_{A}$, password che Alice invia a Bob; $pwd_{B}$, password che Bob invia ad Alice). Si consideri allora la seguente procedura di autenticazione, dove $K_{AB}$ è la chiave segreta condivisa ottenuta con Diffie-Hellman: \begin{enumerate}
\item scambio chiavi Diffie-Hellman;
\item Alice invia un messaggio cifrato con $K_{AB}$ e $pwd_{A}$;
\item Bob invia un messaggio cifrato con $K_{BA}$ e $pwd_{B}$.
\end{enumerate}
Il problema è che l'autenticazione via password, se è in corso un attacco Man-in-the-Middle, non funziona: Mr. X può decifrare tutte i messaggi che riceve da Alice con $K_{AX}$, cifrarli con $K_{XB}$ e inviarli a Bob. Viceversa, può decifrare tutte i messaggi che riceve da Bob con $K_{XB}$, cifrarli con $K_{AX}$ e inviarli ad Alice.
\item Anche altri tipi di proposte sono insufficienti(timestamp, domande personali..). Il problema è che Diffie-Hellman effettua delle operazioni invertibili(cifratura/decifratura) ed è sicuro solo nel caso di attacchi passivi(un intruso intercetta i messaggi, ma non li modifica). E' necessario proteggere l'integrità, per cui si adottano due strategie generali: Diffie-Hellman con Numeri Pubblici e Scambio Diffie-Hellman Autenticato.	
\end{itemize}

\textbf{Diffie-Hellman con Numeri Pubblici}. Un possibile modo per sventare attacchi attivi è evitare che $p$, $g$, $s_{A}$, $s_{B}$, $T_{A}$ e $T_{B}$ vengano generati/calcolati ad ogni scambio. $p$, $g$, $T_{A}$ e $T_{B}$ potrebbero essere resi pubblici in una repository fidata, con $p$ e $g$ uguali per tutti gli utenti, mentre ogni utente U pubblica il proprio valore $T_{U}$, mantenendo privato il segreto $s_{U}$. Ne consegue che, se un avversario non è in grado di accedere alla repository e di modificare i valori pubblici, allora Diffie-Hellman diventa sicuro anche nel caso di attacchi attivi. Inoltre, non è più necessario lo scambio dei valori $T_{A}$ e $T_{B}$: consultando la repository ogni utente A può ottenere la chiave $K_{AB}$ che condividerebbe con l'utente B. \\

\textbf{Scambio Diffie-Hellman Autenticato}. Alice e Bob conoscono un qualche tipo di informazione che permette loro di autenticarsi reciprocamente, ovvero una chiave segreta condivisa $K^{AB}$, da non confondere con la chiave concordata con Diffie-Hellman $K_{AB}$, la propria coppia <chiave privata, chiave pubblica> e la chiave
pubblica dell'altro. Si possono usare tale(i) informazione(i) per provare che sono realmente loro, e non un impostore, coloro che generano i valori di Diffie-Hellman $g$, $p$, $T_{A}$ e $T_{B}$. Tale prova può avvenire sia contestualmente che dopo lo scambio Diffie-Hellman esaminato in precedenza. \\ Alcune possibili soluzioni sono:
\begin{itemize}
\item Autenticazione contestuale allo scambio Diffie-Hellman: \begin{enumerate}
\item Cifrare lo scambio Diffie-Hellman con la chiave segreta $K^{AB}$.
\item Cifrare il valore Diffie-Hellman con la chiave pubblica dell'altro interlocutore.
\item Firmare il valore Diffie-Hellman con la propria chiave privata
\end{enumerate}
\item Autenticazione successiva allo scambio Diffie-Hellman:
\begin{enumerate}
\item Dopo lo scambio Diffie-Hellman, trasmettere un hash della chiave concordata $K_{AB}$, del proprio nome e della chiave segreta $K^{AB}$.
\item Dopo lo scambio Diffie-Hellman, trasmettere un hash del valore
Diffie-Hellman trasmesso e della chiave segreta $K^{AB}$.
\end{enumerate}
\end{itemize}
Notazione adottata: \begin{itemize}
\item $K^{AB}$: chiave segreta condivisa tra Alice e Bob prima di effettuare lo scambio Diffie-Hellman.
\item $K_{AB}$: chiave concordata con Diffie-Hellman.
\item $K^{AB}$ \{msg\} : cifratura di msg con la chiave segreta $K^{AB}$, cioè $E(K^{AB}, msg)$.
\item $\{msg\}_{Bob}$ : cifratura di msg con la chiave pubblica di Bob, cioè $E(PU_{Bob}, msg)$.
\item $[msg]_{Bob}$ : firma di msg con la chiave privata di Bob, cioè
$E(PR_{Bob}, msg)$.
\end{itemize}

Oltre alla mancanza di autenticazione, Diffie-Hellman classico presenta anche lo svantaggio che, la comunicazione cifrata con la chiave concordata, può avvenire soltanto dopo l'esecuzione di uno scambio attivo. In pratica, Alice non può inviare un messaggio cifrato a Bob prima di ricevere $T_{B}$. Tale problema puo essere ovviato
introducendo le chiavi pubbliche Diffie-Hellman: una chiave pubblica D-H è una tripla $<p, g, T>$, dove $T = g^s \, mod \, p$, dove $s$ è la corrispondente chiave privata. Le chiavi pubbliche vanno custodite in un luogo fidato e accessibile da tutti in modo sicuro. La chiave pubblica di Bob è $<p_{B}, g_{B}, T_{B}>$. \\

Di seguito mostriamo un esempio di procedura che consenta ad Alice di inviare un messaggio cifrato a Bob, con la chiave concordata $K_{AB}$, anche se Bob risulta essere inattivo, cioè Alice deve essere in grado di cifrare senza dover attendere alcuna risposta da Bob, il quale, una volta attivo, dovrà poter calcolare $K_{AB}$ e decifrare il messaggio:
\begin{enumerate}
\item Alice genera $s_{A}<p_{B}$, calcola $T_{A}^{*}=g_{B}^{s_{A}} \, mod \, p_{B}$ e $K_{AB}=T_{B}^{S_{A}} \, mod \, p_{B}$.
\item Poi cifra il messaggio $msg=E(K_{AB},msg)$ e lo invia a Bob assieme a $T_{A}^{*}$.
\item Bob, una volta attivo, calcola  $K_{BA}=(T_{A}^*)^{S_{B}} \, mod \, p_{B}$.
\item Decifra $E(K_{AB},msg)$, ottenendo $msg=D(K_{BA},E(K_{AB},msg))$.
\end{enumerate}

\section{Zero Knowledge Proof System}

La dimostrazione della conoscenza di un segreto è alla base di molte tecniche di autenticazione. Nelle tecniche di autenticazione a chiave segreta, il segreto è appunto la chiave condivisa tra i due principal. Nei protocolli a chiave pubblica il segreto è noto solo ad un principal, il quale deve dimostrare all'altro che detiene il segreto senza fornire delle informazioni che possano consentire ad un impostore di eseguire la prova. \\
Una dimostrazione quindi è a \textit{conoscenza zero}(\textbf{Zero Knowledge Proof, ZKP}) se permette di provare la conoscenza di un segreto, che deve essere associato alla chiave pubblica, senza fornire delle informazioni che permettano ad un impostore di eseguire la prova: \begin{itemize}
\item la prova non deve rivelare il segreto;
\item la prova non deve rivelare eventuali informazioni, che pur non essendo il segreto, consentano comunque ad un impostore di effettuare la prova.
\end{itemize}
Le dimostrazioni a conoscenza zero sono impiegate nei protocolli/sistemi di autenticazione(detti \textbf{Zero Knowledge Proof Systems, ZKPS}). RSA è un esempio di ZKPS: è possibile provare la conoscenza di un segreto associato alla chiave pubblica(si pensi alla firma di una sfida), senza rivelare la chiave privata o altre informazioni che
permettano ad un impostore di impersonare il proprietario
della chiave privata. Tuttavia, esistono ZKPS molto più efficienti di RSA, anche se non permettono di cifrare e/o di firmare. \\

Uno schema di autenticazione a conoscenza zero(\textbf{Zero Knowledge Authenication Scheme, ZKAS}) consiste in un'autenticazione che sfrutta una \textit{ZKP}. Non si tratta di una tecnica deterministica, ma
probabilistica(anche RSA in fondo è probabilistica). Deve poter essere resa arbitrariamente piccola la probabilità che un dimostratore onesto fornisca una prova errata; un verificatore onesto fornisca una verifica errata quando il dimostratore è onesto; un dimostratore disonesto fornisca una prova corretta. 

Uno schema di autenticazione a conoscenza zero deve soddisfare i seguenti requisiti: \begin{itemize}
\item[a.] ad ogni entità è associato un segreto privato $s$ e
una chiave pubblica $k_{s}$, cioè una coppia $<s, k_{s}>$. Ovviamente, $k_{s}$ non deve esporre $s$;
\item[b.] l'autenticazione consiste nel provare la conoscenza del segreto $s$;
\item[c.] la prova deve essere a conoscenza zero, cioè le informazioni addotte dal dimostratore non devono poter essere riutilizzate con successo(in seguito) da un impostore, quindi la prova non consente di rivelare $s$;
\item[d.] chi non conosce il segreto $s$ non deve poter eseguire la prova con successo;
\item[e.] chi non conosce il segreto $s$ deve poter verificare la correttezza della prova utilizzando la chiave pubblica $k_{s}$ dell'entità che si sta autenticando(senza la chiave pubblica non deve essere possibile verificare la correttezza della prova).
\end{itemize}

\subsection{ZKAS basato su MSR}

Di seguito mostriamo un protocollo di autenticazione, estremamente efficiente, che sfrutta un problema difficile nell'ambito dell'aritmetica modulare. A rigore tale schema di autenticazione non è
completamente a conoscenza zero, anche se nella pratica può considerarsi tale. \\
Il problema che viene sfruttato è il problema della radice quadrata modulare(\textbf{Modular Square Root, MSR}): dati un numero intero semiprimo grande $n=pq$, con $p$ e $q$ numeri primi grandi, $m<n$ intero assegnato(avente una radice quadrata ordinaria non intera), trovare un numero intero $s$ tale che $s^2 \, mod \, n=m$ è un problema difficile almeno quanto fattorizzare un numero intero.
Dunque, tale protocollo consiste dei seguenti passi: \begin{enumerate}
\item \textbf{Generazione delle chiavi}. Peggy(il dimostratore) calcola la chiave pubblica $<n,v>$, dove $n=pq$ come in RSA, $v$ è un numero di cui Peggy conosce la radice quadrata modulare(ottenere $v$ è semplice, basta scegliere un numero random $s$ e porre $v = s^2 \, mod \, n$; $s$ è la chiave privata di Peggy da non rivelare; $<n,v>$ va divulgata a tutto il mondo.
\item \textbf{Autenticazione}. \begin{itemize}
\item Peggy sceglie $k$ numeri random $r_{1},r_{2},..,r_{k}$;
\item Per ogni $r_{i}$ invia a Victor $r_{i}^2 \, mod \, n$;
\item Victor attribuisce a ciascun $r_{i}^2$ che vale 0 o 1 e la comunica a Peggy;
\item Peggy invia a Victor $sr_{i} \, mod \, n$ per ciascun $r_{i}^2$ etichettato con 1 e $r_{i} \, mod \, n$ per ciascun $r_{i}^2$ etichettato con 0;
\item Victor eleva al quadrato ciascun numero della risposta di Peggy e verifica che tale quadrato valga $vr_{i}^2 \, mod \, n$ se il corrispondente $r_{i}^2$ aveva etichetta 1, oppure $r_{i}^2 \, mod \, n$ se il corrispondente $r_{i}^2$ aveva etichetta 0.
\end{itemize}
\end{enumerate}

Supponiamo che Fred voglia impersonare Peggy. Allora egli è in grado di rispondere in modo corretto agli $r_{i}^2$ che Victor etichetta con 0(Fred può scegliere a suo piacimento gli $r_{i}$), ma non è in grado di rispondere agli $r_{i}^2$ etichettati con 1. In assenza di etichette 0, infatti, il protocollo si semplificherebbe: Peggy si limiterebbe ad inviare delle coppie $<r_{i}^2, sr_{i} \, mod \, n>$, tuttavia non si avrebbe più un protocollo a conoscenza zero, poichè Fred potrebbe usare una precedente sequenza inviata di Peggy ed impersonarla con successo. Invece, l'etichettatura scelta in modo casuale da Victor implica che Fred ha una probabilità del 50\% di
rispondere in modo corretto ad ogni $r_{i}^2$. Fred potrebbe generarsi autonomamente gli $r_{i}$, ma in tal caso non saprebbe rispondere agli $r_{i}^2$ con etichetta 1 oppure potrebbe usare un insieme di $r_{i}^2$ etichettati in passato con 1 in una precedente autenticazione, ma allora non conoscerebbe i corrispondenti $r_{i}$ e non saprebbe rispondere nel caso in cui l'etichetta è 0. Se $k$ è sufficientemente grande la probabilità che Fred impersoni correttamente Peggy tende a 0. \\
ZKAS basato su MSR è molto più efficiente di RSA. Infatti, assumendo $k = 30$, Peggy deve effettuare 45 operazioni modulari(30
quadrati più una media di 15 moltiplicazioni per $s$) e Victor deve effettuare lo stesso numero di operazioni di Peggy; usando RSA Peggy deve eseguire una esponenziazione modulare che consiste in una media di 768 moltiplicazioni modulari mentre Victor se la cava con 3 moltiplicazioni nel caso in cui $e = 3$.
\chapter{Sistemi di Autenticazione}

[INSERIRE SLIDES 1-48]

\section{Intermediari fidati}

\subsection{Distribuzione delle chiavi segrete}
Si assuma che la sicurezza di una rete si basi sulla tecnologia a chiave segreta. Se la rete ha \textit{n} nodi, e
ogni computer \textit{c} deve poter autenticare ogni altro nodo, allora \textit{c} deve memorizzare \textit{n – 1} chiavi (una per ogni altro sistema della rete): se un nuovo nodo viene aggiunto alla rete dovrebbero essere generate \textit{n} nuove chiavi per avere una chiave segreta condivisa con tutti gli \textit{n} nodi pre-esistenti. \\
Sarebbe pertanto necessario distribuire tali \textit{n} chiavi in modo sicuro a tutti gli altri nodi della rete, ma tale strategia di gestione delle chiavi può aver senso solo per reti molto piccole!
[ins figura 50]
Un centro di distribuzione delle chiavi, \textbf{K}ey \textbf{D}istribution \textbf{C}enter (KDC), agevola la gestione/distribuzione delle chiavi segrete: conosce le chiavi di tutti i nodi e se un nuovo nodo si aggiunge alla rete, sola una chiave segreta, condivisa tra quel nodo e il KDC, deve essere generata.
[ins figura 52]

\subsubsection{Centri di distribuzione delle chiavi}
Se un nodo $\alpha$ deve comunicare con un nodo $\beta$, $\alpha$ comunica con il KDC in modo sicuro, e usando la loro chiave segreta condivisa $K_{\alpha KDC}$ gli chiede di inviargli una chiave per comunicare con $\beta$; il KDC autentica $\alpha$, sceglie un numero random $R_{\alpha\beta}$ da usare come chiave segreta condivisa tra $\alpha$ e $\beta$; cifra $R_{\alpha\beta}$ con la chiave segreta $K_{\beta KDC}$ che condivide con $\beta$, e trasmette a $\beta$ $R_{\alpha\beta} $ cifrata insieme alle istruzioni che $\beta$ deve usare per comunicare con $\alpha$ (in genere, il KDC non trasmette direttamente $ R_{\alpha\beta} $ cifrato a $\beta$, per confinare il suo intervento all'interazione con un solo nodo e prevenire attacchi di DoS, ma lo invia ad $\alpha$ che poi lo inoltrerà a $\beta$:
il messaggio cifrato per $\beta$ che il KDC invia ad $\alpha$, e che $\alpha$ dovrà inoltrare a $\beta$ e detto \textbf{ticket}, che, oltre a contenere $ R_{\alpha\beta} $ contiene altre informazioni utili, come data di scadenza, nome del nodo $\alpha$, ecc.). \\ \\
I KDC semplificano la distribuzione delle chiavi: quando un nuovo utente deve essere aggiunto alla rete, o quando si sospetta che una chiave d’utente sia stata compromessa, c’è un singolo punto della rete, il KDC, la cui configurazione deve essere aggiornata. L’alternativa al KDC è installare le informazioni di un utente in ciascun server ove potrebbe accedere. \\
L'uso dei KDC comporta però i seguenti svantaggi: 
\begin{itemize}
\item contiene tutte le informazioni per impersonare un utente ad un qualsiasi altro utente, perciò, se compromesso, tutte le risorse di rete risultano vulnerabili
\item è un "single point of failure", quindi in caso di guasto nessuno può iniziare una comunicazione con nuovi utenti (le chiavi precedentemente distribuite continuano a funzionare)
\item è possibile avere più KDC (KDC multipli) con lo stesso database di chiavi, ma ciò comporta una maggior complessità di gestione, costi extra per le macchine e per la replicazione dei protocolli e maggiori vulnerabilità (è necessario proteggere più target da eventuali attacchi)
\item può costituire un collo di bottiglia (in questa caso, avere più di un KDC può alleviare tale problema)
\end{itemize}  

\subsubsection{Autorità di certificazione}
La distribuzione delle chiavi è più facile con la crittografia a chiave pubblica: ogni nodo deve custodire soltanto la propria chiave privata e tutte le chiavi pubbliche possono essere accessibili in un unico punto. \\
Anche in questo caso continuano ad esserci dei problemi: chi garantisce che le chiavi pubbliche disponibili in un dato punto siano corrette (corrispondano realmente alle entità a cui sono associate)? \\
Per evitare la falsificazione delle chiavi pubbliche si ricorre alle Autorità di Certificazione. \\
Una Autorità di Certificazione, \textbf{C}ertification \textbf{A}uthority CA, è un intermediario fidato che genera i certificati, cioè dei messaggi firmati dalla CA contenenti il nome, la chiave pubblica ed altre informazioni di uno specifico nodo. Tutti i nodi dovranno essere pre-configurati con la chiave pubblica della CA, in modo tale da poter verificare la sua firma sui certificati rilasciati: si tratta dell’unica (in questo specifico esempio) chiave pubblica che devono conoscere a priori. \\ \\
Lo \textbf{standard X.509} per le infrastrutture a chiave pubblica ha definito il formato standard di un certificato. \\
Un \textbf{certificato} contiene:
\begin{itemize}
\item il nome utente (persone fisica, organizzazione, server, applicazione, …)
\item la chiave pubblica dell’utente
\item la data di scadenza
\item un numero seriale
\item la firma, della CA che lo ha emesso, dell’intero contenuto del certificato
\end{itemize}
[ins figura 61]
I certificati possono essere memorizzati nel luogo ritenuto più conveniente (e.g. in un directory service), oppure
ciascun nodo può memorizzare i certificati d’interesse e trasmetterli durante lo "scambio di autenticazione". 
In un certo senso le CA sono la controparte dei KDC nelle infrastrutture a chiave pubblica: CA e KDC costituiscono l'intermediario fidato la cui compromissione può arrecare seri danni all'integrità della rete. \\ \\
Si noti che non è necessario che la CA sia on-line: può risiedere in una stanza fisicamente ben protetta (magari con una guardia) e si può limitare l'accesso alla CA ad una sola persona di grande fiducia. Questa persona, interagendo con la CA, genera il certificato di un dato utente, lo memorizza su un qualche supporto di memoria esterna che può consegnare "a mano" al diretto interessato. Se la CA non è on-line, nessun potenziale intruso può accedervi per ottenere informazioni utili, non deve implementare protocolli di rete che richiedono elevata efficienza computazionale, quindi può essere implementata da un dispositivo estremamente semplice e pertanto molto più sicuro.\\ 
Se la CA dovesse "crashare", la rete continuerebbe a funzionare, ma non sarebbe possibile aggiungere nuovi utenti o
revocare certificati compromessi o di utenti sospetti, pertanto non è essenziale avere CA multiple. \\
I certificati sono poco sensibili ad eventuali attacchi: se memorizzati in modo conveniente, ma potenzialmente
insicuro (ad es. in un servizio di directory), un sabotatore può cancellare i certificati, impedendo l’accesso ai corrispondenti nodi della rete (attacco DOS) ma non può creare dei certificati fasulli o modificarli in qualche modo se non dispone della chiave privata della CA.\\
Una CA compromessa non puo decifrare le conversazioni tra due nodi (reali) da lei serviti (mentre un KDC compromesso può decifrare tutte le conversazioni tra coppie di nodi da lui serviti), pertanto una CA compromessa può ingannare un utente, diciamo Alice, inviandogli una falsa chiave pubblica di Bob e riuscire ad impersonare Bob in una comunicazione con Alice, ma non può decifrare una comunicazione tra la vera Alice e il vero Bob: quindi la compromissione di una CA rimane un fatto molto grave, ma non quanto la compromissione di un KDC. \\ \\
Le CA presentano comunque il seguente svantaggio rispetto i KDC: \\
Supponiamo che Fred offra dei servizi di hosting per conto dell'azienda X SpA; la società X SpA fornisce a Fred un certificato (rilasciato a favore della X SpA) necessario ad autenticare il server, pertanto Fred conosce la chiave privata della chiave pubblica certificata. Supponiamo inoltre che a causa di un contrasto con Fred, la X SpA decida di interrompere il rapporto di lavoro; se il certificato è ancora valido Fred può continuare a rendere dei servizi per conto della X SpA, magari anche danneggiandola. Per la X SpA sarebbe importante informare gli utenti di non accettare il certificato generato per Fred e ancora in corso di validità. \\
Con i KDC tale problema è di facile soluzione: basta rimuovere $K_{Fred}$ dal KDC.\\ 
Ovviamente si ha un problema analogo anche quando viene smarrita, o peggio rubata, la chiave privata associata alla chiave pubblica certificata.\\
Nel caso delle CA non è facile estromettere qualcuno che detiene una chiave privata la cui chiave pubblica è certificata (si noti che potrebbe essere molto rischioso attendere la scadenza naturale del certificato); in queste situazioni conviene revocare il certificato: si revoca un certificato quando, a partire da un certo momento, non devono essere più considerate valide le firme generate con la chiave privata abbinata alla chiave pubblica contenuta nel certificato. \\ \\
La \textbf{revoca} di un certificato si attua inserendo un riferimento al certificato all'interno di una lista di revoca (\textbf{C}ertificate \textbf{R}evocation \textbf{L}ist CRL), che elenca i numeri seriali di certificati da non onorare. Ogni CA pubblica periodicamente una nuova CRL che contiene tutti i certificati revocati e non scaduti.\\
Un certificato è valido se:
\begin{itemize}
\item la firma della CA è valida,
\item non è scaduto
\item non è inserito nella CRL più recente della CA che lo ha emesso
\item la CRL ha una data e ora di emissione
\end{itemize}
Se un'applicazione vuole assicurarsi che nessuno dei certificati che onora sia stato revocato entro un'ora prima
della verifica, allora deve essere trascorsa al più un ora dal momento in cui la CRL, consultata da tale applicazione, è stata emessa; quindi, in questo caso, una nuova CRL deve essere pubblicata con una cadenza di un'ora.\\
Un intruso potrebbe cancellare l'ultima CRL, in tal caso le applicazioni configurate per consultare esclusivamente CRL pubblicate nell'ultima ora si rifiuteranno di onorare tutti i certificati, ma l'intruso non può impersonare un utente valido distruggendo la CRL o sovrascrivendola con una CRL più vecchia.\\
Lo standard X.509 ha definito anche il formato delle CRL, in base al quale una CRL deve includere:
\begin{itemize}
\item una lista di numeri seriali di certificati revocati e non scaduti
\item e una data e ora di emissione della CRL
\item la firma dell'intera lista con la chiave privata della CA
\end{itemize}
\paragraph{Problema}
Supponendo che Bob sia un'applicazione che deve autenticare l'utente Alice, illustrare un possibile protocollo di
autenticazione a chiave pubblica, che includa una verifica dell'appartenenza della chiave pubblica all'utente Alice.
\\ \\
Chiaramente Bob ha bisogno del certificato di Alice, \textit{$Cert_{Alice}$}, e di una CRL recente: Bob può ottenerli da un servizio di directory, oppure Alice li trasmette a Bob. 
In riferimento alla FIGURA, una possibile soluzione è:
\begin{itemize}
\item Bob recupera il certificato di Alice \textit{$Cert_{Alice}$} e una CRL recente
\item Consultando il certificato \textit{$Cert_{Alice}$} individua la chiave pubblica di Alice \textit{$PU_{Alice}$}
\item Verifica la correttezza di \textit{$PU_{Alice}$} come segue:
\begin{itemize}
\item se il certificato \textit{$Cert_{Alice}$} è correttamente firmato dalla CA che lo ha emesso e non è scaduto, e
la CRL è correttamente firmata dalla CA ed è sufficientemente recente e non contiene il certificato \textit{$Cert_{Alice}$}, allora Bob conclude che \textit{$PU_{Alice}$} è corretta (quindi è quella nel certificato)
\end{itemize}
\item Bob ed Alice iniziano uno scambio di messaggi seguendo uno schema di autenticazione a chiave pubblica, ad esempio:
\begin{itemize}
\item Bob invia una sfida \textit{R} ad Alice
\item Alice firma \textit{R} con la propria chiave privata ed invia [\textit{R}]\textit{$NONLOSO_{Alice}$} a Bob
\item Bob verifica la firma di Alice ed in caso affermativo autentica Alice
\end{itemize}
\end{itemize}
[ins fig 76]


\chapter{Sicurezza dei Sistemi Operativi}
Un SO (Sistema Operativo), o OS (Operating System), gestisce il modo in cui le applicazioni software accedono alle risorse hardware del calcolatore:
\begin{itemize} 
  \item CPU
  \item Memoria principale
  \item Memoria secondaria
  \item Periferiche di I/O
  \item Interfacce di rete
\end{itemize}
Un OS fornisce un’interfaccia semplificata e consistente a utenti e applicazioni al fine di interagire con i componenti hardware. Grazie a questa astrazione è possibile sviluppare programmi software senza preoccuparsi della particolare tipologia di hardware sul quale saranno eseguiti. Grazie a questa astrazione è possibile sviluppare programmi software senza preoccuparsi della particolare tipologia di hardware sul quale saranno eseguiti. Gli OS svolgono numerose funzioni, alcune delle quali strettamente legate a problemi di sicurezza, in particolare vedremo:
\begin{itemize} 
  \item meccanismi di autenticazione
  \item sicurezza dei processi
  \item sicurezza del filesystem
  \item sicurezza della memoria
\end{itemize}
\section{Meccanismi di Autenticazione}
Un OS deve poter identificare i propri utenti in modo sicuro, i.e. utenti diversi potrebbero avere permessi di accesso alle risorse diversi. Un meccanismo di autenticazione standard ampiamente usato consiste nell’inserimento di un username e di una password: se la password inserita coincide con la password memorizzata dall'OS per il dato username allora l'utente viene autenticato. Un OS deve dunque memorizzare la password di ogni utente che può accedere al sistema. Generalmente gli OS memorizzano le password criptate attraverso funzioni hash in un file o in un apposito database. Grazie alla proprietà one-way delle funzioni hash, un attaccante che riesce ad accedere al file dove sono memorizzate le password non può ricostruire facilmente il loro valore. Tali file si trovano:
\begin{itemize} 
  \item per Windows in system32 - config - SAM
  \item per Linux in $/etc/passwd$ e in $/etc/shadow$ (solo l’utente root può leggerle)
\end{itemize}

\subsection{Possibili attacchi}
\begin{itemize} 
  \item \textbf{Brute Force}: attacco offline, tutte le possibili password per un dato alfabeto vengono generate automaticamente, criptate con la funzione hash usata dal sistema di autenticazione e confrontate con le password memorizzate
  \item \textbf{Dizionario}: attacco offline, liste di parole comuni (es: nomi) che vengono criptate con la funzione hash usata dal sistema di autenticazione e confrontate con le password memorizzate
  \item \textbf{Rainbow tables}
\end{itemize}
\subsection{Password Robuste}
Per la definizione di password robuste a tali attacchi vanno seguite delle ben definite linee guida:
\begin{itemize}
  \item evitare parole comuni (e.g.: nomi)
  \item evitare password brevi
  \item usare caratteri maiuscoli E minuscoli
  \item usare caratteri speciali
  \item usare i numeri
\end{itemize}
Un esempio di password robusta è: "Voglio compr@re 11 Cani!": per scoprire questa password con un attacco brute-force in 60 giorni dovrei disporre di un computer in grado di generare circa $3,86 x 10^{44} PW/sec$. Per rendere la struttura ancora più sicura si può impostare una scadenza alla password.
\subsection{Password Salt}
Il \textbf{Salt} consiste nell'aggiunta di bit random all'input di una funzione hash (o di un algoritmo di crittografia) al fine di aumentare la randomicità dell'output. Nel caso dell’autenticazione è possibile associare un numero random all'userID dell’utente, seguendo la forma: \newline \newline
\textbf{salt = numero random + userID} \newline \newline
\begin{figure}[htbp]
	\centering%
	\subfigure%
	{\includegraphics[height=8cm, width=13cm, keepaspectratio]{Immagini/Capitolo9/password.png}}
	\caption{Esempi di Salt \label{fig:salt}} 	
\end{figure}
\newpage
Il processo di autenticazione funziona nel seguente modo:
\begin{itemize}
  \item L’utente inserisce il suo userID X e la password P
  \item Il processo di autenticazione dell’OS recupera il salt S per l’userID X e l’hash H del salt concatenato alla password associata a X
  \item OS verifica se $HASH(S \vert \vert P) == H$
\end{itemize}
\begin{figure}[htbp]
	\centering%
	\subfigure%
	{\includegraphics[height=8cm, width=8cm, keepaspectratio]{Immagini/Capitolo9/password2.png}}
	\caption{Processo di autenticazione con Salt \label{fig:salt}} 	
\end{figure}
Tale meccanismo comporta evidenti benefici. Se l’attaccante non può trovare il salt associato con l’userID, allora lo spazio di ricerca per un attacco con dizionario cresce notevolmente, diventando $2^{B} x D$, dove B è il numero di bit del Salt, mentre D lo spazio del dizionario. \newline
Inoltre, anche se l'attaccante fosse in grado di recuperare il salt memorizzato in forma criptata dall’OS, questo meccanismo consente di rallentare notevolmente l’attacco con dizionario, rendendolo valido per un userID alla volta. Senza salt è possibile crackare molte password nello stesso momento, in quanto si ha solo bisogno dell'hash di ogni possibile password e di confrontarlo con tutti gli hash memorizzati. Con il meccanismo di salt, invece, ogni possibile password va
concatenata al salt ad essa associato prima di calcolarne l'hash.
\input{Capitolo_10}
\chapter{Sicurezza delle reti}
\section{Address Resolution Protocol Spoofing}
The address resolution protocol (\textbf{ARP}) connects the network layer to the data layer by converting IP addresses to MAC addresses. \\
ARP works by broadcasting requests and caching responses for future use. The protocol begins with a computer broadcasting a message of the form: 
\begin{lstlisting}
who has <IP address1> tell <IP address2>
\end{lstlisting}
When the machine with \textit{<IP address1>} or an ARP server receives this message, its broadcasts the response:
\begin{lstlisting}
<IP address1> is <MAC address>
\end{lstlisting}
The requestor's IP address \textit{<IP address2>} is contained in the link header.\\
The following Linux and Windows command displays the ARP table:
\begin{lstlisting}
arp-a
\end{lstlisting}
\begin{figure}[htbp]
	\centering%
	\subfigure%
	{\includegraphics[height=2cm, width=10cm, keepaspectratio]{Immagini/Capitolo11/ARP_Caches.png}}
	\caption{ARP Caches\label{fig:ARP_Caches}} 	
\end{figure}
The ARP table is updated whenever an ARP response is received. Notiche that requests are not tracked, ARP announcements are not authenticated and machines trust each other, so s rogue machine can spoof other machines.\begin{figure}[htbp]
	\centering%
	\subfigure%
	{\includegraphics[height=3cm, width=10cm, keepaspectratio]{Immagini/Capitolo11/Poisoned_ARP_Caches.png}}
	\caption{Poisoned ARP Caches\label{fig:Poisoned_ARP_Caches}} 	
\end{figure}
According to the standard, almost all ARP implementations are stateless, so an ARP cache (\figurename ~\ref{fig:ARP_Caches}) updates every time that it receives an arp reply… even if it did not send any arp request!

It is possible to "poison" an ARP cache by sending gratuitous ARP replies (\figurename ~\ref{fig:Poisoned_ARP_Caches})

Notice that using static entries solves the problem but it is almost impossible to manage!

\section{Denial of Service Attack}
The DoS Attack (\figurename ~\ref{fig:DoS_Attack}) consists in sending large number of packets to an host providing service. This attack, often executed by botnet, slows down or crashes the host.\\ 
\begin{figure}[htbp]
	\centering%
	\subfigure%
	{\includegraphics[height=4cm, width=12cm, keepaspectratio]{Immagini/Capitolo11/DoS_Attack.png}}
	\caption{DoS Attack\label{fig:DoS_Attack}} 	
\end{figure}
The attack starts at zombies, travels through tree of internet routers rooted and ends at victim.\\
The \textbf{IP source spoofing} hides attacker and scatters return traffic from victim.\\ \\
An important problem is to identify leaves of DoS propagation tree (i.e. routers next to attacker). Notiche that there are more than 2M internet routers, attacker can spoof source address and knows that traceback is being performed.\\
Some approaches to this problem could be:
\begin{itemize}
\item Filtering and tracing (immediate reaction)
\item Messaging (additional traffic)
\item Logging (additional storage)
\item Probabilistic marking
\end{itemize}
In particular, probabilistic marketing performs random injection of information into packet header, changes seldom used bits, forwards routing information to victim and uses redundancy to survive packet losses. This approach has benefits as:
\begin{itemize}
\item No additional traffic
\item No router storage
\item No packet size increase
\item Can be performed online or offline
\end{itemize}

\subsection{SYN Flood}
SYN Flood is typically DOS attack, though can be combined with other attack such as TCP hijacking.\\
Rely on sending TCP connection requests faster than the server can process them. Attacker creates a large number of packets with spoofed source addresses and setting the SYN flag on these. The server responds with a SYN/ACK for which it never gets a response (waits for about 3 minutes each); eventually the server stops accepting connection requests, thus triggering a denial of service.
The problem of avoiding this attack can be solved in multiple ways: one of the common way to do this is to use SYN cookies.

\input{Capitolo_13}
\appendix
\chapter{Protezione Dati Sensibili}

\section{Introduzione}
Molti DB contengono \textbf{dati sensibili}, che si definiscono come quei dati che \underline{non dovrebbero essere resi pubblici}.
La identificazione dei dati sensibili dipende molto dal \textbf{DB} e dalla sua specifica \underline{applicazione}.
I due casi estremi di questa categorizzazione sono : 
\begin{itemize}
\item \underline{Nessun} dato sensibile (ad es. il catalogo di una biblioteca)
\item \underline{Tutti} dati sensibili (ad es. un database della difesa militare)
\end{itemize}
Questi sono, inoltre, i casi più semplici da gestire, in quanto si \underline{garantisce} accesso a tutto il DB o si \underline{nega} accesso a tutto il DB.
In generale si ha che solo \textbf{alcuni dati sono sensibili} e che, inoltre, ognuno presenta un \textbf{diverso livello di sensibilità} necessario da applicare.

\begin{figure}[htbp]
	\centering

	{\includegraphics[height=15cm, width=12cm, keepaspectratio]{Immagini/Appendice1/prot_dati_01.jpg}}
	\caption{Esempio di una tabella di un DB \label{fig:tabella_db}}
	
\end{figure}

Nella tabella di fig: \ref{fig:tabella_db} i Campi Non Sensibili risultano essere Nome e Dormitorio, quelli Sensibili sono Aiuto, Multe e Droghe mente quelli Parzialmente Sensibili sono Sesso e Razza.
I dati possono essere sensibili secondo i seguenti parametri:
\begin{itemize}

	\item \textbf{Sensibili per Contenuto}, in cui è il valore dell'attributo ad essere sensibile.
	\item \textbf{Sensibili per Provenienza}, in questo caso è la fonte dei dati a necessitare riservatezza.
	\item \textbf{Dichiarati Sensibili}, è lo stesso Amministratore del Sistema a dichiarare che i dati sono sensibili.
	
\end{itemize}

Le parti sensibili di un dato possono essere un \textbf{attributo o record}, un intero record o solo un certo attributo possono risultare sensibili; le parti sensibili lo possono essere sempre o \textbf{diventare sensibili in funzione alla divulgazione di altri dati}, ad esempio la longitudine di un sito segreto non è sensibile finché non è fornito insieme alla latitudine.

Il come non divulgare dati sensibili può non essere semplice, infatti si possono ottenere dati sensibili anche se i dati in sè non vengono mostrati, perciò anche alcune delle caratteristiche dei dati devono essere considerate sensibili.

\begin{figure}[htbp]

	\centering

		{\includegraphics[height=15cm, width=12cm, keepaspectratio]{Immagini/Appendice1/prot_dati_02.jpg}}
			\caption{Esempio di come ottenere dati sensibili senza leggerli effettivamente \label{fig:query_db}}
\end{figure}

Nella fig. \ref{fig:query_db} si nota come, anche se non si accede fisicamente al campo sensibile, si riesce a dedurlo.
Per poter proteggere i dati si potrebbe pensare di attuare delle politiche sempre più restrittive, ad esempio rifiutare ogni tipo di query che faccia riferimento al campo sensibile...ma ciò va a limitare anche le interrogazioni legittime.
Perciò abbiamo due esigenze contrastanti:
\begin{itemize}

	\item Nascondere i dati per evitare di fornire dati sensibili
	\item Facilitare la divulgazione dei dati non sensibili per consentire l'uso corretto dei dati da parte degli utenti autorizzati
	
\end{itemize}

\begin{figure}[htbp]

	\centering

	{\includegraphics[height=15cm, width=12cm, keepaspectratio]{Immagini/Appendice1/prot_dati_03.jpg}}
				\caption{Rappresentazione grafica di accessibilità vs. protezione dei dati \label{fig:protezione_vs_accessibilita}}

\end{figure}

\subsection{Tipi di Divulgazione}
La divulgazione può essere suddivisa nei seguenti gruppi:

\begin{itemize}

	\item\textbf{Dati Esatti}: consiste nella divulgazione del valore esatto del campo sensibile, può avvenire per richiesta esplicita o vengono restituiti come parte di una query insieme ad altri dati non sensibili o, addirittura, per errore del sistema
	\item \textbf{Limiti}: si rendono noti gli estremi di variabilità di un dato sensibile, con una ricerca dicotomica si potrebbe arrivare a dedurre un valore molto prossimo a quello esatto. In certi casi, comunque, la semplice conoscenza dei limiti rappresenta una violazione alla sicurezza stessa.
	\item \textbf{Risultato Negativo}: si può dedurre un certo valore anche dal fatto che è uguale ad un altro valore con cui si è fatta la query (se il numero di condanne di una persona non è zero, si sa che quest'ultima è stata condannata almeno una volta)
	\item \textbf{Esistenza}: anche la sola conoscenza di esistenza di un attributo è sensibile
	\item \textbf{Valore Probabile}: tramite l'utilizzo di diverse query e interpolando i dati ottenuti si possono ottenere informazioni non precise, ma attendibili, riguardo i valori di alcuni dati sensibili
\end{itemize}

\section{Inferenza}
L'\textbf{Inferenza} è un metodo per dedurre o derivare dati, sia sensibili che non. Ricordando la tabella di fig. \ref{fig:tabella_db}, una query del tipo è un \textbf{Attacco Diretto}.
\begin{figure}[htbp]


	{\includegraphics[height=5cm, width=8cm, keepaspectratio]{Immagini/Appendice1/prot_dati_04.jpg}}
				\caption{Query che sfrutta inferenza per ottenere dati sensibili e viene bloccata \label{fig:query_inferenza}}

\end{figure}

Potrebbe essere rifiutata dal DBMS in quanto le tuple del risultato sono quelle con uno specifico valore su un attributo sensibile.
La seguente query, invece, intuitivamente risulta legale ma va a selezionare solo tuple sensibili, in quanto la seconda e la terza condizione non sono mai soddisfatte.

\begin{figure}[htbp]


	{\includegraphics[height=5cm, width=8cm, keepaspectratio]{Immagini/Appendice1/prot_dati_05.jpg}}
				\caption{Query che sfrutta inferenza per ottenere dati sensibili e non viene bloccata \label{fig:query_inferenza1}}

\end{figure}

Per scovare l'accesso improprio ai dati, il DBMS dovrebbe capire che ci sono solo due valori ammessi per il campo Sesso e che non c'è alcuna tupla con valore Dormitorio uguale ad Ayres.
Per limitare gli Attacchi Diretti è applicata, alle volte, una regola ``\textit{n elementi sul k percento}``, che consiste nel non divulgare se un piccolo numero di elementi divulgati costituisce una grande percentuale del valore restituito dalla query (nel caso precedente un unico elemento era il 100\%).

Un attacco inferenziale \textbf{Indiretto} si basa sulla deduzione di dati sensibile a partire da uno o più dati statistici intermedi.

\subsection{Somma e Conteggio}

Una query del genere \ref{fig:query_indiretta_result} potrebbe sembrare innocua poichè restituisce solo valori aggregati, ma il risultato è una \textbf{Divulgazione di Tipo Negativo}, in quanto posso dedurre dal valore nullo nel dormitorio Holmes che le donne in quel dormitorio nn ricevono aiuti finanziari (che è un dato decretato sensibile). 

\begin{figure}[htbp]
	\centering
	
	\subfigure[Query di attacco indiretto di somma]
		{\includegraphics[height=5cm, width=8cm, keepaspectratio]{Immagini/Appendice1/prot_dati_06.jpg}}
				
	\subfigure[Risultato della query]
		{\includegraphics[height=5cm, width=8cm, keepaspectratio]{Immagini/Appendice1/prot_dati_09.jpg}}
							\caption{Esempio di attacco Indiretto \label{fig:query_indiretta_result}}                           
	
\end{figure}

Inoltre, il conteggio può essere combinato con la somma per produrre risultati ancora più rivelatori!!

\begin{figure}[htpb]
	\centering
	
	\subfigure[Query di attacco indiretto di conteggio]
		{\includegraphics[height=5cm, width=8cm, keepaspectratio]{Immagini/Appendice1/prot_dati_07.jpg}}
				
	\subfigure[Risultato della query]
		{\includegraphics[height=5cm, width=8cm, keepaspectratio]{Immagini/Appendice1/prot_dati_08.jpg}}
							\caption{Esempio di attacco Indiretto di Somma e Conteggio \label{fig:query_indiretta_result1}}                           
	
\end{figure}

Il risultato della query di \textbf{Attacco di Somma} \ref{fig:query_indiretta_result} combinata con la query di \textbf{Attacco di Conteggio} \ref{fig:query_indiretta_result1} rivela che i due uomini nel dormitorio Holmes e West ricevono un aiuto finanziario di 5000 e 4000.

\subsection{Media}

La media può permettere di ottenere informazioni esatte se utilizzate in combinazione con il conteggio. Usata in modo opportuno riesce anche ad aggirare eventuali controlli sulla numerosità dell'insieme su cui viene calcolata.
Ad esempio, supponiamo che la media non venga divulgata se la somma degli elementi su cui è calcolata è pari ad 1 (legge interna del DBMS). 

\begin{figure}[htpb]
	\centering
		\subfigure[Query di Media per ottenere informazioni sugli studenti di Sesso Femminile]

		{\includegraphics[height=2cm, width=8cm, keepaspectratio]{Immagini/Appendice1/prot_dati_10.jpg}}
		
		\subfigure[Query di Count da combinare con la Query di Media per ottenere un Attacco Inferenziale]
		{\includegraphics[height=2cm, width=8cm, keepaspectratio]{Immagini/Appendice1/prot_dati_11.jpg}}
		
		\subfigure[Combinazione delle informazioni ottenute]
				{\includegraphics[height=2cm, width=8cm, keepaspectratio]{Immagini/Appendice1/prot_dati_12.jpg}}
		
		\caption{Esempio di query di media per ottenere indirettamente informazioni 
		  \label{fig:query_media}}  

\end{figure}

Dalla procedure di \ref{fig:query_media} si trova che $ a_{3} = 5000 $ che sarebbe un dato sensibile, in quanto media di un valore su un unico elemento!

\subsection{Attacchi di Tracker}
Abbiamo visto che una delle tecniche di protezione dall'inferenza consiste nel non divulgare i dati se solo un piccolo numero di elementi fornisce molta informazione; è possibile aggirare questa limitazione effettuando più query lecite e combinandole insieme. 

\begin{figure}[htpb]
\centering

	{\includegraphics[height=5cm, width=8cm, keepaspectratio]{Immagini/Appendice1/prot_dati_13.jpg}}
		\caption{Esempio di query di tracker non lecita du DB
				  \label{fig:query_tracker_sbagliata}}  
\end{figure}

La query \ref{fig:query_tracker_sbagliata} è, ovviamente, bloccata dal DBMS in quanto va a richiedere un unico risultato sulla tabella del DB in fig. \ref{fig:tabella_db}.
Il dato non viene divulgato in quanto \underline{dominato da una sola tupla}, ma osserviamo che il dato in questione pò essere calcolato contando partendo dal fatto che tutte le donne non sono di razza \textit{C} o che non risiedono nel dormitorio Holmes.

\begin{figure}[htpb]
\centering
	{\includegraphics[height=5cm, width=8cm, keepaspectratio]{Immagini/Appendice1/prot_dati_14.jpg}}
		\caption{Esempio di query di tracker non lecita du DB
				  \label{fig:query_tracker}}  
\end{figure}

\subsection{Vulnerabilità del Sistema Lineare}
L'attacco di Tracker è un caso specifico di una vulnerabilità più generica in cui sfruttando l'algebra, la logica e un po' di fortuna si riesce ad ottenere tramite diverse query dei valori protetti.
\newpage
Nell'esempio \ref{fig:query_tracker} la query protetta è del tipo \textit{q = COUNT\{(Sesso = F)}$\wedge$\textit{(Razza = C)}$\wedge$\textit{(Dormitorio = Holmes)\}}.
\newline
In base all'algebra, possiamo riscrivere la query nel seguente modo: 
\textit{q = COUNT(Sesso = F)}$-$\textit{COUNT\{(Sesso = F)}$\neg$\textit{(Razza = C)}$\wedge$\textit{(Dormitorio = Holmes)\}}.
\newline
Più in generale potremmo avere le seguenti query:
\newline
$q_{1} = c_{1} + c_{2} + c_{3} + c_{42}$
\newline
$q_{2} = c_{3} + c_{2} $
\newline
$q_{3} = c_{5} + c_{1} + c_{3}$
\newline
$q_{4} = c_{4} + c_{7} + c_{2}$
\newline
$q_{5} = c_{8} + c_{3}$
\\
Nessuna delle query rivela un qualsiasi valore $c_{i}$ ma, risolvendo il sistema, possiamo conoscerli tutti.

\subsection{Aggregazione}
L'aggregazione è un altro particolare tipo di inferenza, in cui si cerca di ottenere dati sensibili mettendo insieme dati non sensibili (ad esempio il nome di un impiegato e il suo stipendio, divisi non sono dati sensibili ma insieme si). 
\newline
Evitare l'aggregazione è difficile perché bisogna tenere conto delle informazioni giù in possesso dell'utente. Se un utente conosce il nome di un dipendente non devo comunicargli lo stipendio e viceversa. Purtroppo tenere traccia di tutti i dati rivelati ad ogni utente è proibitivo, soprattutto per il fatto che utenti diversi potrebbero condividere tutti le informazioni in loro possesso.

\section{Protezione dall'Inferenza}
Esistono alcune tecniche per combattere l'inferenza: 
\begin{itemize}
\item Analisi delle query
\item Soppressione dei valori sensibili per non fornirli
\item Occultamento del valore reale, con un altro simile
\end{itemize}

\subsection{Soppressione e Occultamento}
Un esempio di soppressione è la già citata regola del ``\textit{n elementi sul k percento}``; purtroppo, per essere efficace, può essere necessario sopprimere altri dati per evitare che quelli sensibili siano ottenibili per differenza.

\begin{figure}[htpb]
	\centering
		\subfigure[Numero di residenti per Sesso e Dormitorio]
		{\includegraphics[height=2cm, width=8cm, keepaspectratio]{Immagini/Appendice1/prot_dati_15.jpg}}
		
		\subfigure[Tabella dopo la soppressione]
		{\includegraphics[height=2cm, width=8cm, keepaspectratio]{Immagini/Appendice1/prot_dati_16.jpg}}
				
		\caption{Esempio di protezione per soppressione
		  \label{fig:query_soppressione}}  

\end{figure}
Per evitare di divulgare dati sensibili è possibile combinare più righe o colonne, implementando una protezione dei dati di Occultamento.

\begin{figure}[htpb]
	\centering
		\subfigure[Numero di residenti per Sesso e Uso di Droghe]
		{\includegraphics[height=2cm, width=8cm, keepaspectratio]{Immagini/Appendice1/prot_dati_17.jpg}}
		
		\subfigure[Tabella dopo l'occultamento]
		{\includegraphics[height=2cm, width=8cm, keepaspectratio]{Immagini/Appendice1/prot_dati_18.jpg}}
				
		\caption{Esempio di protezione per occultamento
		  \label{fig:query_occultamento}}  

\end{figure}


Nel caso \ref{fig:query_occultamento} la query viene eseguita su di un \textbf{Campione Casuale} dei dati, il campione deve essere abbastanza ampio da rappresentare l'intera popolazione. In questo modo i dati sono sempre rappresentativi ma si diminuisce la possibilità di attacchi statistici.

\subsection{Perturbazione Casuale dei dati}
A volte può essere utile perturbare i valori del DB con un piccolo errore, per ogni $x_{i}$ che rappresenta il valore vero dell'attributo i si può generare un piccolo errore $e_{i}$ e aggiungerlo a $x_{i}$ per calcolare dati statistici.
Valori di query aggregate, quali la somma o la media, produrranno valori vicini a quelli veri ma non esatti.
\section{Esercitazione sull'inferenza}
Con riferimento alla tabella dipendenti, in \figurename ~\ref{fig:Tabella_dipendenti}, trovare lo stipendio di Mister X con un attacco indiretto di tipo media.\\
\begin{figure}[htbp]
	\centering%
	\subfigure%
	{\includegraphics[height=6cm, width=12cm, keepaspectratio]{Immagini/Appendice1/Tabella_dipendenti.png}}
	\caption{Tabella dipendenti\label{fig:Tabella_dipendenti}} 	
\end{figure}
Si hanno a disposizione le seguenti informazioni:
\begin{enumerate}
\item [a.] Mister X è un uomo con più di 30 anni e non è di Roma;
\item [b.] i rimanenti dipendenti maschi con più di 30 anni sono tutti romani;
\item [c.] nessun dipendente donna è di Roma.
\end{enumerate}
Si tenga conto, inoltre, che viene applicata la seguente politica di protezione dei dati sensibili:
\begin{enumerate}
\item il campo stipendio non può essere divulgato direttamente e non può essere utilizzato come criterio di filtraggio;
\item i criteri di filtraggio possono essere età, sesso, e città;
\item non possono essere utilizzati più di due criteri di filtraggio in una singola interrogazione;
\item è consentita la divulgazione di medie e conteggi a patto che si riferiscano ad almeno tre tuple.
\end{enumerate}
\subsection{Svolgimento}
\subsubsection{Idea base}
Una possibile strategia di attacco prevede l'individuazione di due insiemi di tuple U e V tali che:
\begin{enumerate}
\item [I.] ciascun insieme contiene singolarmente almeno tre tuple (vincolo 4.);
\item [II.] ciascun insieme deve poter essere individuato usando al più due condizioni di filtraggio (vincolo 3.) basate sugli attributi eta, sesso, e città (vincolo 2.);
\item [III.] la differenza insiemistica $U \ V$ deve contenere soltanto mister X; ovvero ${Mister X} = U \ V$;
\item [IV.] l'insieme $V$ deve essere un sottoinsieme di $U$, i.e. $V \subset U$
\end{enumerate}
Una volta individuati due insiemi $U$ e $V$, soddisfacenti le condizioni I, II, III e IV, ponendo:
\begin{itemize}
\item $m_{U}$ = media degli stipendi dei dipendenti in $U$;
\item $c_{U}$ = numero dei dipendenti in $U$;
\item $m_{V}$ = media degli stipendi dei dipendenti in $V$;
\item $c_{V}$ = numero dei dipendenti in $V$;
\end{itemize}
si ottiene che: $stipendio_{Mister X} = m_{U} \times c_{U} – m_{V} \times c_{V}$.\\
\subsubsection{Individuazione di $U$ e $V$}
A questo punto si considerino i seguenti insiemi:
\begin{itemize}
\item $A$: insieme dei dipendenti di sesso maschile (sesso = M);
\item $B$: insieme dei dipendenti con più di 30 anni (eta > 30);
\item $C$: insieme dei dipendenti romani (citta = Roma);
\end{itemize}
Si osservi che dalla (a) e dalla (b) risulta che: ${Mister X} = (A \cap B) \ C = (A \cap B) \ (B \cap C)$\\
mentre dalla (c) si ha che: $C \subset A$\\
pertanto risulta che: ${Mister X} = (A \cap B) \ (B \cap C)$ e $(B \cap C) \subset (A \cap B)$\\
quindi gli insiemi $U$ e $V$ sono:
\begin{itemize}
\item $U = A \cap B$
\item $V = B \cap C$
\end{itemize}
\subsubsection{Query SQL}
Infine vediamo quali sono le query  SQL per ottenere le informazioni che ci servono:\\ \\
Individuazione di $m_{U}$ e $c_{U}$:
\begin{lstlisting}
SELECT avg(stipendi), count(stipendi) 
FROM dipendenti 
WHERE sesso = 'M' AND eta > 30;
\end{lstlisting}
Individuazione di $m_{V}$ e $c_{V}$:
\begin{lstlisting}
SELECT avg(stipendi), count(stipendi) 
FROM dipendenti 
WHERE eta > 30 AND citta = 'Roma';
\end{lstlisting}




%%BIBLIOGRAFIA============================================
\normalsize
\newpage
\bibliography{bibliografia}
\bibliographystyle{plain}

\end{document}
